\documentclass[10pt,letterpaper]{article}
\usepackage[english]{babel}
\usepackage{graphicx}
\usepackage[margin=2cm]{geometry}

\usepackage{amsfonts}
\usepackage{amsmath}
\usepackage{amsthm}
\newcommand{\dif}[1][]{\mathrm{d} {#1}\,}
\newcommand{\rb}[1]{ \left(  {#1} \right) }
\newcommand{\frb}[1]{ \left(  {#1} \right) }

\newtheoremstyle{break}
{\topsep}
{\topsep}%
{\normalfont}
{}%
{\bfseries}
{}%
{\newline}
{}%
\theoremstyle{break}
\newtheorem{exercise}{Exercise}
\newtheorem*{information}{Information}
\newtheorem{solution}{Solution}
\newtheorem*{solutioninformation}{Solution Information}

%%%%%%%%%%%%%%%%%%%%%%%%%%%%%%%%%%%%%%%%%%%%%%%%%%%%%%%%%%%%%%%%%%%%%%
%%%%%%%%%%%%%%%%%%%%%%%%%%%%%%%%%%%%%%%%%%%%%%%%%%%%%%%%%%%%%%%%%%%%%%



%========================================================================
\title{Background and characteristics}
\date{}

\begin{document}

\maketitle



\begin{exercise}
    Suppose that $u$ is the density of a quantity over the real line.
    We assume that its velocity over the real line is given by
    $$
        v(x,t) = u(x,t)^3.
    $$
    Describe the change of the integral of $u$ over a subinterval $[x_1,x_2]$ in terms of the velocity $v$ and $u$ itself. Use the fundamental theorem of calculus to derive a conservation law, and find its characteristics. 
    Discuss the similarity to Burgers' equation, and how this example can be generalized.
\end{exercise}

\begin{solution}
    We consider the time-dependent integral 
    $$
        M(t) = \int_{x_1}^{x_2} u(x,t) dx.
    $$
    The influx/outflux of the endpoint is described by the velocity 
    $$
        \partial_t M(t) 
        =
        \partial_t \int_{x_1}^{x_2} u(x,t) dx
        =
        u(x_1,t)^3 u(x_1,t) - u(x_2,t)^3 u(x_2,t) 
        =
        u(x_1,t)^4 - u(x_2,t)^4 
        .
    $$
    Via the fundamental theorem of calculus,
    $$
        \partial_t \int_{x_1}^{x_2} u(x,t) dx
        =
        - \int_{x_1}^{x_2} \partial_x u(x,t)^4 dx
        .
    $$
    We take the time-derivative under the integral. 
    If we take the intervals infinitesimally small, then we derive the conservation law
    $$
        \partial_t u(x,t)
        +
        \partial_x \left( u(x,t)^4 \right)
        =
        0
        .
    $$
    The solution is constant along the lines in directions $(4u^3,1)$.
    The characteristics have the form 
    $$
        \partial_t x(t) = 4u(x,t)^3.
    $$
    This has the general form of a conservation law with flux $f(u) = u^4$.
    We get Burgers' equation if we replace the exponent by $2$,
    and we get a transport equation if we replace the exponent by $1$.
\end{solution}





\begin{exercise}
    An example for a nonlinear wave equation is
    $$
        \partial_{tt} E = \partial_{x} \left( |\partial_x E|^{p-2} \partial_x E \right)
    $$
    where $p \geq 1$. Reformulate this as a system of conservation laws.
\end{exercise}

\begin{solution}
    Similar as in the lecture, we introduce the variables $v = \partial_t E$ and $s = \partial_s E$, which represent the velocity and the slope.
    The system of conservation laws consists of 
    $$
        \partial_t s = \partial_s v,
        \qquad
        \partial_t v = \partial_s \left( s^{p-2} s \right).
    $$
\end{solution}




\begin{exercise}
    Consider the transport equation 
    $$
        \partial_t u(x,t) + \partial_x\left( a(t) u(x,t) \right) = 0
    $$
    where the velocity depends only one the time variable $t$.
    Find the characteristics and show that the characteristics do not intersect.
    \textit{This simple PDE is only for practice purposes.}
\end{exercise}

\begin{solution}
    The characteristic originating at $x_0$ satisfies 
    $$
        \partial_t x(t) = a(t), \quad x(0) = x_0.
    $$
    Consequently, it is given by 
    $$
        x(t) = x_0 + \int_0^t a(s) ds,
    $$
    Suppose that $x_0(t)$ and $x_1(t)$ are two characteristics that start at $x_0$ and $x_1$, respectively, and suppose that $x_1 > x_0$.

    We then have 
    $$
        x_1(t) - x_0(t)
        =
        x_1 + \int_0^t a(s) ds - x_0 + \int_0^t a(s) ds
        =
        x_1 - x_0
        .
    $$
    So the distance between $x_1(t)$ and $x_0(t)$ remains constant over time.
\end{solution}







\begin{exercise}
    (Hard) Consider the conservation law (transport equation)
    $$
        \partial_t u(x,t) + \partial_x\left( a(x) u(x,t) \right) = 0
    $$
    We have seen that the characteristics satisfy the following ordinary differential equation with initial values:
    $$
        \partial_t x(t) = a(x(t)), \quad x(0) = x_0.
    $$
    Show that the characteristics do not intersect if the velocity is always positive. \footnote{This is often noted casually in the literature, but can be proven rigorously.}
\end{exercise}

\begin{solution}
    Suppose that $x_0$ and $x_1$ are two points with $x_0 < x_1$. 
    We write $x_0(t)$ and $x_1(t)$ for the two associated characteristics. 
    They satisfy the equations 
    $$
        x_0(t) = x_0 + \int_0^t a( x_0(s) ) ds,
        \qquad
        x_1(t) = x_1 + \int_0^t a( x_1(s) ) ds.
    $$
    If the two characteristics intersect, 
    then there is some $T$ such that $x_0(T) = x_1(T)$.

    Since the velocity is always positive, $x_0(t)$ must have have passed $x_1$ at some time $t_0$, where we have $x_0(t_0) = x_1$.
    At the intersection time $T$ we then must have 
    $$
        x_0 + \int_0^T a( x_0(s) ) ds
        =
        x_1 + \int_0^T a( x_1(s) ) ds.
    $$
    We split the first integral
    $$
        x_0
        +
        \underbrace{ \int_0^{t_0} a( x_0(s) ) ds }_{x_1 - x_0}
        +
        \underbrace{ \int_{t_0}^T a( x_0(s) ) ds }_{\int_0^{T-t_0} a( x_1(s) ) ds}
        =
        x_1
        +
        \int_0^T a( x_1(s) ) ds.
    $$
    Here, the first underbraced integral uses $x_0(0)=x_0$ and $x_0(t_0)=x_1$.
    The second underbraced integral uses that the characteristic of $x_0(t)$ follows the same trajectory of $x_1(t)$ (but with a time delay).
    So this becomes 
    $$
        x_1
        +
        \int_0^{T-t_0} a( x_1(s) ) ds 
        =
        x_1
        +
        \int_0^T a( x_1(s) ) ds.
    $$
    But this cannot be, because $a$ is always positive. 
\end{solution}
    





\end{document}
