\documentclass{article}

\usepackage{../auxFiles/ExStyPac}


\usepackage[framed]{../auxFiles/mcode}





\input ../auxFiles/mac.tex







%========================================================================
\begin{document}

\Head{4}{Finite Difference Schemes (Solution)}






\begin{exerciseList}



\item
\begin{enumerate}
\item
{\textit{Consistency.}}\quad 
A method is consistent if its local truncation error $T_k$ satisfies 
\begin{gather} 	
\tau_k(x,t)=O\frb{k^p}+O\frb{h^q}
\qquad
\text{where}\qquad p,q>0\ . 
\end{gather} 
Essentially, consistency tells us that we are approximating the solution of the correct PDE. 

\item
{\textit{Stability.}}\quad 
A method $v^{n+1}=\cH_kv^n$ is stable if for each $T>0$ there exist constants $C$ and $k_0$, such that 
\begin{gather} 	
	\|\cH_k^n\|\le C \qquad\qquad 0\le 
nk\le T\ , \quad 0<k<k_0\ . 
\end{gather}

\item
{\textit{Convergence.}}\quad
A method is convergent if the error $E_k$ satisfies 
\begin{gather} 	
	\lim_{k\to0} \max_{0\le kn\le T}\|E_k\frb{\cdot,kn}\| = 0 \ . 
\end{gather} 
\end{enumerate}

The importance of stability is made clear by the Lax equivalence theorem. In class you have seen a proof of one direction of this equivalence: if a method $v^{n+1}=\cH_kv^n$ is consistent and stable, then it is convergent. 



\item
\begin{enumerate}\setcounter{enumii}{1}
\item
Let $u$ be a smooth solution of $u_t+au_x=0$. The local truncation error for the Leapfrog scheme is given by 
\begin{gather} \label{LeapfTrunc}
	k\tau_k(x,t)=u\frb{x,t+k}-u\frb{x,t-k}+\frac{ak}{h}\rb{u\frb{x+h,t}-u\frb{x-h,t}}\ . 
\end{gather}
Next, we expand all the terms on the right hand side of \deq{LeapfTrunc} about $(x,t)$. For example, we have $u\frb{x,t+k}$ and $u\frb{x,t-k}$ given by 
\begin{gather}
	u\frb{x,t+k}=u +u_tk +\frac{1}{2}u_{tt}k^2+O\frb{k^3} 
\end{gather}
and
\begin{gather}
	u\frb{x,t-k}=u -u_tk +\frac{1}{2}u_{tt}k^2+O\frb{k^3}\ , 
\end{gather}
where, for simplicity of notation, $u$, $u_t$ and $u_{tt}$ stands for $u(x,t)$, $u_t(x,t)$ and $u_{tt}(x,t)$, respectively. It is, thus, clear that 
\begin{gather}
	u\frb{x,t+k}-u\frb{x,t-k}=2ku_t+O\frb{k^3}\ .
\end{gather}
By repeating the calculation also for the translations in space $u(x\pm h,t)$, we get 
\begin{gather}
	k\tau_k(x,t)=2ku_t+O\frb{k^3} +\frac{ak}{h}\rb{2hu_x+O\frb{h^3}}\ , 
\end{gather}
which implies 
\begin{gather}
	\tau_k(x,t)=2\rb{u_t+au_x}+O\frb{k^2}+O\frb{h^2}\ . 
\end{gather}
Since $u$ satisfies $u_t+au_x=0$, the local truncation error is simply $\tau_k(x,t)=O\frb{k^2}+O\frb{h^2}$. 

\item
Notice that at the $n$-th time step, to calculate $v^{n+1}$, the Leapfrog scheme requires the values, not only $v^n$, but also of $v^{n-1}$.
Compared to the LF and LW schemes which use only $v^n$ to calculate $v^{n+1}$, this is a clear disadvantage. In computations, keeping the numerical solution at more than one time level multiplies the size of memory required for the program.

Another issue is obtaining the first values. In our example, the values of $v^0$ are obtained directly from the initial data, however the Leapfrog scheme requires another time level to work.
So we must use other means to get $v^1$, and only then can we use the Leapfrog method to advance.
\end{enumerate}






\item
To show that the Lax-Friedrichs (LF) scheme is stable provided 
\begin{gather} \label{LF_stab_cond}
	\frac{|a|k}{h}\le 1\ , 
\end{gather} 
we show that $\|\cH^n\|_\infty$ is bounded for all $n$, where $\cH$ is the operator defined by
\begin{gather}
	\cH v_j=\frac{1}{2}\rb{v_{j+1}+v_{j-1}} -\frac{ak}{2h}\rb{v_{j+1}-v_{j-1}}\ .
\end{gather}
(That is, the LF scheme is given by $v^{n+1}=\cH v^n$.)
To do so, we suppose $v$ is some grid function, and calculate the norm of $\cH v$.
We have
\begin{gather}a	
	\norm{\cH v^n}_\infty &=\norm{ \frac{1}{2}\rb{v_{j+1} +v_{j-1}} 
			-\frac{ak}{2h}\rb{v_{j+1} -v_{j-1}}}_\infty, \\
		&\le \frac{1}{2} \rb{\abs{1-\frac{ak}{h}}
			+\abs{1+\frac{ak}{h}}}\ \norm{v^n}_\infty, \\
		&\le	\norm{v^n}_\infty,	 
\end{gather}a
whenever (\ref{LF_stab_cond}) holds. Thus, 
\begin{gather} 	
	\|\cH^n\|_\infty \le \|\cH\|_\infty^n \le1
	\qquad\qquad \forall n\in\N\ . 
\end{gather}

%Another approach uses the Fourier transform.
%The following is a formal presentation of the method, however the arguments can be made rigorous.
%By formally substituting
%\begin{gather}
%	v_j^n=e^{ij\tta}\hat v^n
%\end{gather}
%into the scheme, we find that $\hat v^n$ satisfies
%\begin{gather}
%	\hat v^{n+1}=\sqb{\frac{1}{2}\rb{e^{i\tta}+e^{-i\tta}} -\frac{ak}{2h}\rb{e^{i\tta}-e^{-i\tta}}}\hat v^n=:g\frb{\tta}\hat v^n\ .
%\end{gather}
%Since the Fourier transform preserves the 2-norm, the 2-norms of $v^n$ and $\hat v^n$ are equal.
%Thus, to prove stability it suffices to show $|g\frb{\tta}|\le 1$, for all $\tta$.
%Observing that
%\begin{gather}
%	g\frb{\tta}=\cos\tta-i\frac{ak}{2h}\, \sin\tta\ ,
%\end{gather}
%we have
%\begin{gather}
%	|g\frb{\tta}|^2=\cos^2\tta+\rb{\frac{ak}{2h}}\, \sin^2\tta=1-\rb{1-\rb{\frac{ak}{2h}}^2}\sin^2\tta\ ,
%\end{gather}
%which is less than or equal to one, for all $\tta$, if and only if \deq{LF_stab_cond} holds.



\item

Let us consider the scheme
\begin{gather} \label{implicit} 	
	v_j^{n+1}=v_j^n -\lambda \rb{v_{j}^{n+1} -v_{j-1}^{n+1}}, \quad \lambda = \frac{ak}{h}. 
\end{gather} 
Before trying to show stability, we first re-write the scheme (assuming periodic boundaries) as 
\begin{gather} \label{implicit_1} 	
	\rb{1 + \lambda} v_{j}^{n+1} -\lambda v_{j-1}^{n+1} =v_j^n \quad \implies \quad A v^{n+1} = v_n,
\end{gather} 
where 
\begin{gather}
A = \begin{bmatrix}
1+\lambda & 0 & \dots & 0 & -\lambda \\
-\lambda & 1+\lambda & 0 & \dots & 0 \\
\ddots & \ddots & \ddots & 0 & 0\\
0 & \ddots & \ddots & \ddots & 0 \\
0 & \dots & 0 & -\lambda & 1 + \lambda
\end{bmatrix}
\end{gather}
Note that we can now write the scheme in the form $v^{n+1} = \cH v^n$, where $\cH = A^{-1}$, provided $A$ is non-singular. Since $A$ is strictly diagonally dominant, its eigenvalues are non-zero by Gershgorin's theorem. This proves that $A$ is invertible. Furthermore, there is a nice theorem \cite{Varah} about a diagonally dominant matrix A , which says 
\begin{gather}
\|A^{-1}\|_\infty \leq \frac{1}{\alpha}, \quad \alpha = \min_k\left(|A_{kk}| - \sum_{j\neq k} |A_{kj}| \right).
\end{gather}
In our case, $\alpha = 1$. Thus, $\norm{\cH}_\infty = \norm{A^{-1}}_\infty \leq 1$. Since this condition holds independent of any CFL restrictions on $h$ and $k$, the scheme is unconditionally stable.



\item
\begin{enumerate}
\item
Matlab code for implementing the schemes to solve the advection equation can be found on the last two pages of this solution manual.

\item
In Figure 1, $u\frb{x,0.5}$ is plotted as solved by the Upwind, Lax-Friedrichs, Lax-Wendroff and Beam-Warming methods.

\item
Both the upwind and Lax-Friedrichs schemes capture the discontinuity.
The upwind scheme, however, seems to produce less numerical dissipation.
This is because the upwind scheme exploits that fact that information travels only in one direction.
The higher order methods Lax-Wendroff and Beam-Warming both introduce oscillations around the discontinuities.

\item[(d),(e)]
\addtocounter{enumii}{2}
See Figure 2.


\item
Notice how, on this problem with non-smooth solutions, the rate of convergence of the first order methods is now $O\frb{h^{1/2}}$, while the rate of convergence of the second order methods seem to be $O\frb{h^{0.6}}$ at best.
\end{enumerate}





\newpage
\begin{figure}[H] 
\begin{center}
\subfigure[]{\includegraphics[width=8cm, height=6cm]{Upwind_Profile}}
\subfigure[]{\includegraphics[width=8cm, height=6cm]{Lax-Friedrichs_Profile}}
\subfigure[]{\includegraphics[width=8cm, height=6cm]{Lax-Wendroff_Profile}}
\subfigure[]{\includegraphics[width=8cm, height=6cm]{Beam-Warming_Profile}}
\caption{ The analytical solution and the numerical solution $u(x,t=0.5)$ for the advection equation on Riemann initial data with $a=1$, $h=0.005$, $\frac{k}{h}=0.5$.} 
\end{center}
\end{figure}



\begin{figure}[H] 
\begin{center}
\subfigure[]{\includegraphics[width=8cm, height=6cm]{Upwind_Accuracy}}
\subfigure[]{\includegraphics[width=8cm, height=6cm]{Lax-Friedrichs_Accuracy}}
\subfigure[]{\includegraphics[width=8cm, height=6cm]{Lax-Wendroff_Accuracy}}
\subfigure[]{\includegraphics[width=8cm, height=6cm]{Beam-Warming_Accuracy}}
\caption{ The error as a function of resolution. The error is measured in 1-norm, the resolution is measured as the number of discretization points in the x dimension. }
\end{center}
\end{figure}




\newpage



\begin{lstlisting}
% Solution03: Problem 5
% This script was written for EPFL MATH459, Numerical Methods for
% Conservation Laws. The scalar advection equation du/dx + du/dt = 0
% with riemann initial data: u = 1 if x<0,  0 if x>0.
% The problem is solved using the following schemes:
%       1. Upwind
%       2. Lax-Friedrichs
%       3. Lax-Wendroff
%       4. Beam-Warming
% The solution is visualized and the accuracy of the scheme tested.

clc
clear all
close all

% Scheme Options: UW --> Upwind
%                 LF --> Lax-Friedrichs
%                 LW --> Lax-Wendroff
%                 BW --> Beam-Warming
% Task Options  : Solve --> To simply run the scheme with h=0.0025 and k/h = 0.5
%                 Acc   --> Test accuracy of the scheme by generating Log-Log
%                           plots
Scheme 	= 'BW';
Task 	= 'Acc';

% Advection speed and final time
a  = 1;
Tf = 0.5;

% Initial condition
U0 =@(x,t) 1*(x-a*t < 0);

% Set the function H for the scheme
%   u^{n+1} = H(u^n)
% as well as the number of ghost cells Ng needed at each end of the mesh (for
% boundary conditions) and the scheme name for saving solutions.
% Here lam = a*k/h and U is the extended solution vector
if(strcmp(Scheme,'UW'))
    Ng    = 1;
    Hfunc =@(U,lam) (1 - lam)*U(2:end-1) + lam*U(1:end-2);
    name  = 'Upwind';
elseif(strcmp(Scheme,'LF'))
    Ng    = 1;
    Hfunc =@(U,lam) (1 - lam)/2*U(3:end) + (1 + lam)/2*U(1:end-2);
    name  = 'Lax-Friedrichs';
elseif(strcmp(Scheme,'LW'))
    Ng    = 1;
    Hfunc =@(U,lam) (1 - lam^2)*U(2:end-1) + (lam^2-lam)/2*U(3:end)...
                     + (lam^2+lam)/2*U(1:end-2);
    name  = 'Lax-Wendroff';
elseif(strcmp(Scheme,'BW'))
    Ng    = 2;
    Hfunc =@(U,lam) (1 - 3*lam/2 + lam^2/2)*U(3:end-2) ...
                    + (2*lam - lam^2)*U(2:end-3)...
                    + (-lam + lam^2)/2*U(1:end-4);
    name  = 'Beam-Warming';
else
    error('Unknown Scheme selected!!');
end


% Performing Tasks
% Run scheme
if strcmp(Task,'Solve')
    fprintf('Solving problem with %s scheme\n',name)
    
    % Discretization
    h  = 0.0025;
    k  = 0.5*h;
    x  = -1:h:1;
    t  = 0:k:Tf;
    N  = numel(x);
    Nt = numel(t);
    
    % Set initial condition
    U = U0(x,0);
    
    % Solve using the numerical scheme. We use open boundary conditions
    lam = a*k/h;
    plot_every = 10; % Plot after every 10 time instances
    
    for i = 2:Nt
        Uext   = [ones(1,Ng)*U(1),U,ones(1,Ng)*U(N)];
        U      = Hfunc(Uext,lam);
        
        if(mod(i,plot_every)==0 || i==Nt)
            Uexact = U0(x,t(i));
            figure(1)
            plot(x,U,'-r','LineWidth',2)
            hold all
            plot(x,Uexact,'-k','LineWidth',2)
            ylim([-0.5,1.5])
            grid on;
            legend('Numerical','Exact','Location','best')
            title([name,', time = ',num2str(t(i))]);
            hold off
            set(gca,'XTick',-0.5:0.5:1,'FontSize',20)
        end
        
    end
    
    % Save plot
    FIG = figure(1);
    fname = sprintf('%s_Profile.pdf',name);
    set(FIG,'Units','Inches');
    pos = get(FIG,'Position');
    set(FIG,'PaperPositionMode','Auto','PaperUnits','Inches','PaperSize',[pos(3), pos(4)])
    print(FIG,fname,'-dpdf','-r0')
    
% Accuracy test
elseif strcmp(Task,'Acc')
    fprintf('Finding accuracy of %s scheme\n',name)
    H = 0.05*2.^(0:-1:-8);
    E = zeros(numel(H),1);
    n = zeros(numel(H),1);
    % Find Error
    for i = 1:numel(H)
        
        % Discretization
        h  = H(i);
        fprintf('... Solving for h = %f\n',h)
        k  = 0.5*h;
        x  = -1:h:1;
        t  = 0:k:Tf;
        N  = numel(x);
        Nt = numel(t);
        
        % Set initial condition
        U = U0(x,0);
        
        % Exact solution at final time
        Uexact = U0(x,Tf);
        
        % Solve using the numerical scheme. We use open boundary conditions
        lam = a*k/h;
        
        for j = 2:Nt
            Uext   = [ones(1,Ng)*U(1),U,ones(1,Ng)*U(N)];
            U      = Hfunc(Uext,lam);
        end
        
        % Measure error in 1 norm
        E(i) = sum(abs(U-Uexact))*h;
        n(i) = N;
    end
    % Make loglog plot
    p = polyfit(log(n),log(E),1);
    FIG = figure(1);
    loglog(n,E,'-ok');grid on;
    xlabel('Resolution','FontSize',20);
    ylabel('Error','FontSize',20);
    title([name,'. Slope = ',num2str(p(1))],'FontSize',20);
    set(gca,'FontSize',20)
    set(FIG,'Units','Inches');
    pos = get(FIG,'Position');
    set(FIG,'PaperPositionMode','Auto','PaperUnits','Inches','PaperSize',[pos(3), pos(4)])
    fname = sprintf('%s_Accuracy.pdf',name);
    print(FIG,fname,'-dpdf','-r0')
end







\end{lstlisting}

\end{exerciseList}
\begin{thebibliography}{99}

\bibitem{Varah}
\newblock{A lower bound for the smallest singular value of a matrix}, by J. M. Varah.
\newblock{\em Linear Algebra and its Applications, vol. 11, issue 1, pg 3-5, 1975}.
 
\end{thebibliography}


\end{document}
