\documentclass[10pt,letterpaper]{article}
\usepackage[english]{babel}
\usepackage{graphicx}
\usepackage[margin=2cm]{geometry}

\usepackage{float}
\usepackage{subfloat}
\usepackage{caption}
\usepackage{subcaption}
\usepackage{listings}

\usepackage{mathtools}
\usepackage{amsfonts}
\usepackage{amsmath}
\usepackage{amssymb}
\usepackage{amsthm}
\newcommand{\p}{ } % TODO: What was the purpose of this macro???
\newcommand{\cM}{} % TODO: What was the purpose of this macro???
\newcommand{\cR}{} % TODO: What was the purpose of this macro???
\newcommand{\cS}{} % TODO: What was the purpose of this macro???
\newcommand{\dtdo}{} % TODO: What was the purpose of this macro???
\newcommand{\wt}{} % TODO: What was the purpose of this macro???
\newcommand{\ol}{} % TODO: What was the purpose of this macro???
\newcommand{\om}{} % TODO: What was the purpose of this macro???
\newcommand{\Om}{} % TODO: What was the purpose of this macro???
\newcommand{\hpar}{} % TODO: What was the purpose of this macro???
\newcommand{\Del}{} % TODO: What was the purpose of this macro???
\newcommand{\dif}[1][]{\mathrm{d} {#1}\,}
\newcommand{\rb}[1]{ \left(  {#1} \right) }
\newcommand{\frb}[1]{ \left(  {#1} \right) }
\newcommand{\norm}[1]{ \left\|  {#1} \right\| }
\newcommand{\jph}{{j+\frac{1}{2}}}
\newcommand{\jmh}{{j-\frac{1}{2}}}
\newcommand{\sqb}[1]{ \left(  {#1} \right) }
\newcommand{\mat}[1]{ \begin{array}{cc}  {#1} \end{array} }
\newcommand{\myvector}[1]{ \begin{array}{c}  {#1} \end{array} }
\newcommand{\iph}{{i + \frac{1}{2}}}
\newcommand{\imh}{{i - \frac{1}{2}}}

% \usepackage[amsmath,thmmarks,standard]{ntheorem}
\newtheoremstyle{break}
{\topsep}
{\topsep}%
{\normalfont}
{}%
{\bfseries}
{}%
{\newline}
{}%
\theoremstyle{break}
\newtheorem{exercise}{Exercise}
\newtheorem*{information}{Information}
\newtheorem{mysolution}{Solution}
% \newenvironment{solution}{\begin{comment}}{\end{comment}}
\newtheorem*{solutioninformation}{Solution Information}

\usepackage{comment}
% Switch between showing and hiding solutions by commenting out either of the following lines
\newenvironment{solution}{\begin{mysolution}}{\end{mysolution}}
% \excludecomment{solution}



%========================================================================
\begin{document}



\title{Discontinuous Galerkin}
\date{}
\author{}

\maketitle










Let $\Om=\Om_x\times\Om_t$, where $\Om_x=\rb{-1,1}$ is the spacial domain and $\Om_t=\rb{0,T}$ is the time domain.

\begin{exercise}

\begin{enumerate}
\item
Consider the scalar problem
\begin{equation}
	u_t+au_x=bu
	\quad
	x\in\rb{-1,1}
\end{equation}
with proper initial conditions and $a$ and $b$ being real constants.
\textbf{(a)}
Propose a discontinuous Galerkin method for solving this problem.
\textbf{(b)}
Prove that the semi-discrete scheme is stable. 

\item
Consider the scalar PDE
\begin{gather} \label{WEH}
	u_t+au_x=g(x,t)
\end{gather}%
in $\Om$, with proper initial conditions and periodic boundary conditions.
Here $a>0$ is a real constant.
Write the weak discontinuous Galerkin (DG) formulation for the problem.
Pick an appropriate numerical flux and prove that the semi-discrete scheme is stable.
(Hint: To prove stability, consider the problem for the error between the computed and the exact solution - known as the error equation).
\end{enumerate}




\item
Consider the system 
\begin{align}%a
	u_{t}+v_{x} &=0\\
	v_{t}+u_{x} &=0
\end{align}%a
in $\Om$, subject to periodic boundary conditions.
Write the weak DG formulation for the problem.
Show that the formulation with the upwind flux is stable.



\item
Consider the ODE system
\begin{gather} \label{odeSys}
	u'=L(u)\ .
\end{gather}%
Suppose there exists a positive constant $k_{FE}$ such that the forward Euler method
$$
	v^{n+1}=v^n+kL\frb{v^n}
$$
with $0< k\le k_{FE}$, applied to \eqref{odeSys} satisfies
\begin{gather} \label{stabEst}
	\|v^{n+1}\|\le\|v^n\|
\end{gather}%
in some norm $\|\cdot\|$.
Show that the numerical approximation $U$ of \eqref{odeSys} obtained by the following 3rd-order Runga-Kutta method
\begin{align*}
	U^{\left(1\right)}= & U^{n}+kL\left(U^{n}\right)\\
	U^{\left(2\right)}= & \frac{3}{4}U^{n}+\frac{1}{4}U^{\left(1\right)}+\frac{1}{4}kL\frb{U^{\left(1\right)}}\\
	U^{n+1}= & \frac{1}{3}U^{n}+\frac{2}{3}U^{\left(2\right)}+\frac{2}{3}kL\frb{U^{\left(2\right)}}
\end{align*}
satisfies \eqref{stabEst}, provided $0< k\le k_{FE}$.


%\item
%Find the \emph{nodal} DG code on the course website.
%\begin{enumerate}[label={\textbf{(\alph*)}}]
%
%\item
%Modify the solver for the scalar advection equation to solve 
%\begin{gather} \label{WEg}
%	u_t+au_x=h(x,t)
%	\quad
%	x\in\rb{0,2}
%\end{gather}%
%with $a>0$ and 
%\begin{gather}%
%	u(0,t)=g\frb{t}\ ,\quad u(x,0)=f(x)\ .
%\end{gather}%
%
%\item
%Verify the convergence rate of the scheme for various $N$ on \eqref{WEg} with $h(x,t)=0$.
%\end{enumerate}
%



\end{exercise}



\def\hbu{\boldsymbol{\hat u}}
\def\bpsi{\boldsymbol{\psi}}
%\def\hbf{\boldsymbol{\hat f}}


\def\bu{\boldsymbol{u}}
\def\bv{\boldsymbol{v}}

\def\bpsi{\boldsymbol{\psi}}
\def\hbForce{\boldsymbol{\hat g}}
%\def\bl{\boldsymbol{\ell}}




%========================================================================



\begin{solution}
	\begin{enumerate}
	\item
	%{\bf (a)}\quad To develop a DG method for the PDE \begin{gather} 	u_t+au_x=bu \end{gather} we suppose the approximation $u_h$ is given in the $k$th element $D^k=\sqb{x_l^k,x_r^k}$ by \begin{gather} 	u_h^k(x,t)=\sum_{n=1}^{N_p} \hat u_n^k(t)\ \psi_n^k(x)\ , \end{gather} where $\psi_n^k$ is some basis of the space of polynomials of degree no grater than $N=N_p-1$. We require that for each $k=1,\ldots,K$, the residual $\cR_h=\rb{\partial_t+a\partial_x-b}u_h$ satisfies
	%\begin{gather}
	%	0=\rb{\cR_h\ , \psi_j^k}_{D^k}=\int_{D^k} \cR_h\psi_j^k\dif x
	%	\qquad\qquad j=1,\ldots,N_p\ .
	%\end{gather}
	%We use the divergence theorem and substitute the boundary terms by the numerical flux $f^*$ to get \begin{gather} \label{dg} 	\sum_{n=1}^{N_p} \dtdo{\hat u_n^k}{t} \int_{D^k} \psi_n^k\psi_j^k\dif x 		-\sum_{n=1}^{N_p}a\hat u_n^k \int_{D^k} \psi_n^k \dtdo{\psi_j^k}{x}\dif x 		-\sum_{n=1}^{N_p}b\hat u_n^k \int_{D^k} \psi_n^k\psi_j^k\dif x 	=-\sqb{\psi_j^kf^*}_{x_l^k}^{x_r^k}\ . \end{gather} Equations \eqref{dg} with $j=1,\ldots,N_p$, can also be written as a linear system \begin{gather} \label{dgSys} 	\hat\cM^k\dtdo{}{t}\hbu_h^k -a\rb{\hat \cS^k}^T \hbu_h^k-b\hat\cM^k\hbu_h^k 	=-\sqb{\bpsi^kf^*}_{x_l^k}^{x_r^k} \end{gather} where $\hbu_h^k=\rb{\hat u_1^k,\ldots,\hat u_{N_p}^k}^T$, and $\bpsi^k=\rb{\psi_1^k,\ldots,\psi_{N_p}^k}^T$. \\ {\bf (b)}\quad In this exercise, we suppose that $a>0$ and use the upwind flux \begin{gather} 	f_{k,k+1}^*=au_r^k\ . 	\qquad\qquad 	u_r^k=u_h^k\rb{x_r^k,\cdot}\ . \end{gather} We also write  $u_h^k\rb{x_l^k,\cdot}=u_l^k$. Multiplying \eqref{dgSys} by $\hbu_h^k$, and summing over the elements provides \begin{gather} 	\dtdo{}{t}\norm{u_h}^2-b\norm{u_h}^2 	=\sum_{k=1}^K 	\bigg[ 		a\rb{u_r^k}^2 		-a\rb{u_l^k}^2 		-2u_r^k f_{k,k+1}^* 		+2u_l^k f_{k-1,k}^* 	\bigg]\ . \end{gather} Thus, we have \begin{align} 	\dtdo{}{t}\norm{u_h}^2-b\norm{u_h}^2 	&=-\sum_{k=1}^{K-1} 	\bigg[ 		-a\rb{u_r^k}^2 		+2u_r^k f_{k,k+1}^* 		+a\rb{u_l^{k+1}}^2 		-2u_l^{k+1} f_{k,k+1}^* 	\bigg]\notag\\ 	&\quad 		+a\rb{u_r^K}^2 		-2u_r^K f_{K,K+1}^* 		-a\rb{u_l^{1}}^2 		+2u_l^{1} f_{0,1}^*\ . \end{align} Since the terms \[ 	+a\rb{u_r^K}^2 		-2u_r^K f_{K,K+1}^* 		-a\rb{u_l^{1}}^2 		+2u_l^{1} f_{0,1}^* \] correspond to the boundary conditions, we will not address them here. Therefore, it is left to show that for each $k=1,\ldots,K-1$, the term \begin{gather} 	\delta^k=-a\rb{u_r^k}^2 			+2u_r^k f_{k,k+1}^* 			+a\rb{u_l^{k+1}}^2 			-2u_l^{k+1} f_{k,k+1}^* \end{gather} is nonnegative. Since clearly, \begin{gather} 	\delta^k=-a\rb{u_r^k}^2 			+2a\ u_r^k u_r^k 			+a\rb{u_l^{k+1}}^2 			-2a u_r^{k} u_l^{k+1} 		=a\rb{u_r^k-u_l^{k+1}}^2\ge0 \end{gather} the scheme is stable.




	\item
	Consider the scalar PDE
	\begin{equation}
		u_t+au_x=g(x,t)
	\end{equation}
	in $\Om$ with proper initial conditions and periodic boundary conditions.
	\begin{enumerate}

		\item
		Suppose the approximation $u_h$ is given in the $k$th element $D^k=\rb{x_l^k,x_r^k}$ by
		\begin{gather} \label{uh_def}
			u_h^k(x,t)=\sum_{n=1}^{N_p} \hat u_n^k(t)\ \psi_n^k(x)\ ,
		\end{gather}
		where $\psi_n^k$ is some basis of the space of polynomials of degree no grater than $N=N_p-1$.
		We require that for each $k=1,\ldots,K$,
		\begin{gather} % TODO: What is \cR?
			\rb{\cR_h\, ,\, \psi_j^k}_{D^k}=\hat g_j^k
			\qquad\qquad j=1,\ldots,N_p\ ,
		\end{gather}
		where $\cR_h=\rb{\p_t+a\p_x}u_h$, $\hat g_j^k=\rb{g,\psi_j^k}_{D^k}$, and
		\begin{gather}
			\rb{\eta,\psi}_{D^k}=\int_{D^k} \eta\psi\dif x\ .
		\end{gather}
		We use the divergence theorem and substitute the boundary terms by the numerical flux $f^*$, to get
		\begin{gather}
			\rb{\p_t u_h^k\, ,\, \psi_j^k}_{D^k} -a\rb{\, u_h^k\, ,\rb{\psi_j^{k}}'}_{D^k}
				=-\sqb{f^*\psi_j^k}_{x_l^k}^{x_r^k}+\hat g_j^k
			\qquad
			j=1,\ldots,N_p\ .
		\end{gather}
		Substituting \eqref{uh_def} into the last equation, provides
		\begin{gather}
			\sum_{n=1}^{N_p} \rb{\psi_n^k, \psi_j^k}_{D^k} \rb{\hat u_n^k}'
				-a \sum_{n=1}^{N_p} \rb{\psi_n^k\, , \rb{\psi_j^k}'}_{D^k}\hat u_n^k
			=-\sqb{f^*\psi_j^k}_{x_l^k}^{x_r^k}+\hat g_j^k\ ,
		\end{gather}
		for $j=1,\ldots,N_p$.
		This can be also written as a system,
		\begin{gather} \label{dg2}
			\hat\cM^k \rb{ \hbu_h^k}' - a \rb{\hat\cS^k}^T \hbu_h^k
			=
			-\sqb{ f^*\bpsi^k }_{ x_l^k }^{ x_r^k }+\hbForce^k\ ,
		\end{gather}
		where
		\begin{gather}
			\hat\cM^k_{jn}=\rb{\psi_n^k,\psi_j^k}_{D^k}
			\qquad
			\hat\cS^k_{nj}=\rb{\psi_n^k\, , \rb{\psi_j^k}'}_{D^k}\ .
		\end{gather}
		The periodic boundary conditions are enforced by requiring $f^*\big|_{x_l^1}=f^*\big|_{x_r^K}$.
		This is can be obtained by formally defining $u_h^0=u_h^K$, $u_h^{K+1}=u_h^1$ and then using the same numerical flux at the entire domain (including the boundaries $x=\pm1$).

		\item
		We take the dot product of \eqref{dg2} and $\hbu_h^k$, and sum over $k=1,\ldots,K$ to get
		\begin{gather}
			\dtdo{}{t}\norm{u_h}^2
				=\sum_{k=1}^K
					\bigg[
						a\rb{u_r^k}^2 -a\rb{u_l^k}^2 -2u_r^k f_{k+1/2}^* +2u_l^k f_{k-1/2}^*
					\bigg]
					+2\rb{g,u_h}\ ,
		\end{gather}
		where, for $k=1,\ldots,K$, $u_l^k=u_h^k\frb{x_l^k,\cdot}$, $u_r^k=u_h^k\frb{x_r^k,\cdot}$, $u_r^0=u_r^K$, $u_l^{K+1}=u_l^1$, and
		\begin{gather}
			f^*_{k+1/2} =f^*\frb{u_r^k,u_l^{k+1}}
				\quad
				k=0,\ldots,K\ .
		\end{gather}
		By rearranging the sum we get
		\begin{gather}
			\dtdo{}{t}\norm{u_h}^2
				=-\sum_{k=0}^K \rb{u_r^k-u_l^{k+1}}\Big( 2f^*_{k+1/2}-a\rb{u_r^k+u_l^{k+1}} \Big)
					+2\rb{g,u_h}\ .
		\end{gather}
		Substituting the upwind flux $f^*(u_l,u_r)=au_l$ provides
		\begin{gather}
			\dtdo{}{t}\norm{u_h}^2
				=-\sum_{k=0}^K a\rb{u_r^k-u_l^{k+1}}^2
					+2\rb{g,u_h}
			\le 2\rb{g,u_h}\ .
		\end{gather}
		Finally, by the Cauchy-Schwarz inequality we get
		\begin{gather}
			\frac{1}{2}\dtdo{}{t}\norm{u_h}^2\le\rb{g,u_h}
				\le\norm{g}\norm{u_h}\ ,
		\end{gather}
		which leads to the desired bound on the norm of $u_h$.
	\end{enumerate}



	\item
	\begin{enumerate}

		\item
		The system
		\begin{align}
			u_t+v_x&=0\\
			v_t+u_x&=0\ ,
		\end{align}
		has the form
		\begin{gather}
			w_t+Aw_x=0
		\end{gather}
		where $A$ is a symmetric matrix.
		In the $k$th element $D^k$ the approximation $w_h$ is given by
		\begin{gather}
			w_h^k(x,t)=\sum_{n=1}^{N_p} \psi_n^k(x)\hat w_n^k(t)\ .
		\end{gather}
		We require
		\begin{gather}
			\int_{D^k} \psi_j^k\rb{\p_t w_h^k+A\p_x w_h^k}\dif x =0
			\qquad\qquad k=1,\ldots,K\ .
		\end{gather}
		We integrate by parts, and replace $Aw_h^k$ in the boundary terms by a numerical flux $f^*$ to get
		\begin{gather} \label{DGel}
			\sum_{n=1}^{N_p} \rb{\psi_j^k,\psi_n^k}_{D^k}\rb{\hat w_n^k}'\
					-\sum_{n=1}^{N_p} \rb{\rb{\psi_j^k}',\, \psi_n^k}_{D^k} A\hat  w_n^k
				=-\sqb{\psi_j^kf^*}_{x_l^k}^{x_r^k}\ .
		\end{gather}
		That is
		\begin{align}
			&\hat\cM^k \rb{\hat\bu_n^k}'-\rb{\hat\cS^k}^T\hat\bv_h^k
				=\sqb{\bpsi g^*}_{x_l^k}^{x_r^k}\\
			&\hat\cM^k \rb{\hat\bv_n^k}'-\rb{\hat\cS^k}^T\hat\bu_h^k
				=\sqb{\bpsi h^*}_{x_l^k}^{x_r^k}\ ,
		\end{align}
		where $f^*=\rb{g^*,h^*}^T$.

		\item
		By taking the dot product of \eqref{DGel} and $\hat w_j^k$, and summing over $j=1,\ldots,N_p$, we get
		\begin{gather}
			\frac{1}{2}\dtdo{}{t}\norm{w_h^k}_{D^k}^2-\int_{D^k}\dtdo{w_h^k}{x}\cdot A w_h^k\ \dif x
				=-\sqb{w_h^k\cdot f^*}_{x_l^k}^{x_r^k}\ .
		\end{gather}
		Since $A$ is symmetric, this implies
		\begin{gather}
			\dtdo{}{t}\norm{w_h^k}_{D^k}^2=\sqb{w_h^k\cdot \rb{Aw_h^k- 2 f^*}}_{x_l^k}^{x_r^k}
				=w_r^k\cdot \rb{Aw_r^k- 2 f_{k,k+1}^*}-w_l^k\cdot \rb{Aw_l^k- 2 f_{k-1,k}^*}\ .
		\end{gather}
		Here $w_r^k=w_h^k\frb{x_r^k,\cdot}$, and $w_l^k=w_h^k\frb{x_l^k,\cdot}$.
		Summing over $k=1,\ldots,K$ provides
		\begin{align} \label{tDer}
			\dtdo{}{t}\norm{w_h}^2 &=\sum_{k=1}^K w_r^k\cdot \rb{Aw_r^k- 2 f_{k+1/2}^*}
					-\sum_{k=0}^{K-1} w_l^{k+1}\cdot \rb{Aw_l^{k+1}- 2 f_{k+1/2}^*}\\
				&=-\sum_{k=0}^{K}\bigg[w_l^{k+1}\cdot Aw_l^{k+1} -w_r^k\cdot Aw_r^k
					-2\rb{w_l^{k+1}-w_r^k}\cdot f_{k+1/2}^*\bigg]\ .
		\end{align}
		Now we substitute the upwind flux
		\begin{gather}
			f_{k+1/2}^*=\frac{1}{2}\ A\rb{w_l^{k+1}+w_r^k} -\frac{1}{2}\ |A|\rb{w_l^{k+1}-w_r^k}
		\end{gather}
		into \eqref{tDer} to get
		\begin{gather}
			\dtdo{}{t}\norm{w_h}^2
				=-\sum_{k=1}^{K}\rb{w_l^{k+1}-w_r^k}\cdot|A|\rb{w_l^{k+1}-w_r^k}\le 0\ .
		\end{gather}
	\end{enumerate}

	\item
	Consider the ODE system
	\begin{gather} 
		u'=L(u)\ .
	\end{gather}
	Suppose there exists a positive constant $k_{FE}$ such that the forward Euler method
	$$
		v^{n+1}=v^n+kL\frb{v^n}
	$$
	with $0< k\le k_{FE}$, applied to \eqref{odeSys} satisfies
	\begin{gather} 
		\|v^{n+1}\|\le\|v^n\|
	\end{gather}
	in some norm $\|\cdot\|$.
	Suppose $U$ is the numerical approximation of \eqref{odeSys} obtained by the following 3rd-order Runga-Kutta method
	\begin{align*}
		U^{\left(1\right)}= & U^{n}+kL\left(U^{n}\right)\\
		U^{\left(2\right)}= & \frac{3}{4}U^{n}+\frac{1}{4}U^{\left(1\right)}+\frac{1}{4}kL\frb{U^{\left(1\right)}}\\
		U^{n+1}= & \frac{1}{3}U^{n}+\frac{2}{3}U^{\left(2\right)}+\frac{2}{3}kL\frb{U^{\left(2\right)}}\ ,
	\end{align*}
	and $0< k\le k_{FE}$. Next we show that
	\begin{gather} \label{estUi}
		\|U\hpar{i}\|\le\|U^n\|
	\end{gather}
	holds for $i=1,2$.
	As $0<k\le k_{FE}$, it is clear that $U\hpar{1}$ satisfies \eqref{estUi}.
	This implies
	\begin{gather}
		\|U\hpar{2}\|\le \frac{3}{4}\|U^n\| +\frac{1}{4}\norm{U\hpar{1} +kL\frb{U\hpar{1}}}
			\le \frac{3}{4}\|U^n\| +\frac{1}{4}\|U\hpar{1}\|
	\end{gather}
	which by \eqref{estUi} with $i=1$, implies \eqref{estUi} also for $i=2$.
	Finally, we have
	\begin{gather}
		\|U^{n+1}\|\le \frac{1}{3}\|U^n\| +\frac{2}{3}\norm{U\hpar{2} +kL\frb{U\hpar{2}}}
			\le \frac{1}{3}\|U^n\| +\frac{2}{3}\|U\hpar{2}\|
	\end{gather}
	which by \eqref{estUi} with $i=2$, implies the proposition.
	\end{enumerate}
\end{solution}

\end{document}
