\documentclass{article}



\usepackage{../auxFiles/ExStyPac}

\usepackage[framed]{../auxFiles/mcode}





\input ../auxFiles/mac.tex
\graphicspath{{./figures/}}





%========================================================================
\begin{document}



\Head{7}{Linear Systems (Solution)}





\begin{exerciseList}



%\item
%To write the scheme in explicit
%form we simply insert the flux in the conservation law and reduce
%to get
%\begin{gather}
%	U_{j}^{n+1}=U_{j}^{n}
%		-\frac{k}{h}\left[A^{-}\left(U_{j+1}^{n}-U_{j}^{n}\right)
%			+A^{+}\left(U_{j}^{n}-U_{j-1}^{n}\right)\right]\ .
%\end{gather}

\item
We start by calculating the spectral decomposition of
\begin{gather}
	A=\sqb{\mat{
		u_{0} & K_{0}\\
		1/\rho_{0} & u_{0}
	}}\ .
\end{gather}
Since this is a $2\times2$ matrix, this can be done easily.
We have
\begin{gather}
	\lambda_{1}=u_{0}-c_{0},
	\dqquad
	S_{1}=\rb{-\rho_{0}c_{0}\ ,\ 1}^T\ ,
\end{gather}
and
\begin{gather}
	\lambda_{2}=u_{0}+c_{0},
	\dqquad
	S_{2}=\rb{\rho_{0}c_{0}\ , \ 1}^T\ ,
\end{gather}
where $c_{0}=\sqrt{K_{0}/\rho_{0}}$.
Thus,
\begin{equation}
	S=\sqb{\mat{
		-\rho_{0}c_{0} & \rho_{0}c_{0}\\
		1 & 1
	}}
	\dqquad
	S^{-1}=\frac{1}{2\rho_{0}c_{0}}
		\sqb{\mat{
			-1 & \rho_{0}c_{0}\\
			1 & \rho_{0}c_{0}
		}}\ .
\end{equation}
The matrices $\Lambda^{+}$ and $\Lambda^{-}$ can be expressed as
\begin{gather}
	\Lambda^{+}=\sqb{\mat{
		\lambda_1^+ & 0\\
		0 & \lambda_2^+
	}}
	\qquad
	\Lambda^{-}=\sqb{\mat{
		\lambda_1^- & 0\\
		0 & \lambda_2^-
	}}\ 
\end{gather}
where $\lambda_s^+ = \max\left(\lambda_s,0\right)$ and $\lambda_s^- = \min\left(\lambda_s,0\right)$ for $s=1,2$. Thus we obtain
\begin{gather}
	A^{+}=S\Lambda^{+}S^{-1}
		=\frac{1}{2}\sqb{\mat{
			(\lambda_1^+ + \lambda_2^+) & \rho_{0}c_{0} (-\lambda_1^+ + \lambda_2^+)\\
			(\rho_0c_0)^{-1} (-\lambda_1^+ + \lambda_2^+)& (\lambda_1^+ + \lambda_2^+)
		}}
\end{gather}
and
\begin{gather}
	A^{-}=S\Lambda^{-}S^{-1}
		=\frac{1}{2}\sqb{\mat{
			(\lambda_1^- + \lambda_2^-) & \rho_{0}c_{0} (-\lambda_1^- + \lambda_2^-)\\
			(\rho_0c_0)^{-1} (-\lambda_1^- + \lambda_2^-)& (\lambda_1^- + \lambda_2^-)
		}}\ .
\end{gather}


\item
The CFL condition is a necessary condition for stability.
Here, the CFL condition requires that for each eigenvalue $\lambda_{s}$ of $A$, the following must hold
\begin{gather}
	\frac{|\lambda_{s}|k}{h} \leq1\ .
\end{gather}
Making use of the eigenvalue structure for the given system, we arrive at the condition
\begin{equation}
	\left(|u_{0}|+c_{0}\right)\frac{k}{h}\leq1\ .
\end{equation}



\item
See the Matlab code attached at the end of this solution manual.


\item
See figures generated by the Matlab code.
Since the conservation law is a linear system, we know that discontinuities move only along characteristics and can not spontaneously form.
If the initial condition is smooth and satisfies the boundary conditions, then the solution is also smooth.
When using the second set of initial conditions, which is  discontinuous, the discontinuities
seem to get smeared out as the scheme advances.
This is however due to numerical diffusion in the scheme, and not because discontinuities disappear.


\item
See figures generated by the Matlab code attached at the end of this solution manual.
With this third set of initial conditions, it is not straightforward to locate the shocks as these are smeared out to a point where they are indistinguishable from other smooth regions of the solution.

Using the generalized upwind method for the linear system, we resolve and propagate information in a proper up-winded manner yielding less numerical diffusion than other schemes that return monotone solutions over discontinuities, such as the Lax-Friedrichs method.
However, the upwind method is still only a first order method and in general first order methods are not suitable for long time integration or resolving fine details.
Note that in the case of a linear system we can actually construct the exact solution.
See the Matlab code attached for an example on how to do this. 


\end{exerciseList}


%\newpage
\lstinputlisting{./Code/Exercise7.m}
\newpage
\lstinputlisting{./Code/GodunovFlux.m}
\lstinputlisting{./Code/apply_bc.m}
\lstinputlisting{./Code/find_exact.m}
\end{document}
