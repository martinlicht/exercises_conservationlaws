\documentclass{article}



\usepackage{../auxFiles/ExStyPac}



\input ../auxFiles/mac.tex
\graphicspath{{./figures/}}





%========================================================================
\begin{document}



\Head{12}{WENO Reconstruction}








\begin{exerciseList}


\item
The WENO reconstruction is based on a convex combination, with coefficients $\om_r$, of approximations $v_{i+1/2}\hpar{r}$ calculated on $k$ different stencils.
The coefficients $\om_r$ depend on another set of coefficients $d_r$ and some smoothness indicators $\beta_r$.

\begin{enumerate}



\item
The Smoothness indicators $\beta_r$ are defined by
\begin{gather} \label{smWts}
	\beta_r=\sum_{l=1}^{k-1}\Delta x^{2l-1}
		\int_{x_i-1/2}^{x_{i+1/2} }\rb{\frac{\dif^lp_r}{\dif x^l}\rb{x}}^2\dif x
	\qquad\qquad r=0,\ldots,k-1\ .
\end{gather}%
For $k=2$ (3rd-order reconstruction), this yields the following expressions:
\begin{gather} \label{eqSW21}
	\beta_0 =\rb{\ol U_{i+1}-\ol U_{i}}^2\ ,
	\qquad\qquad
	\beta_1 =\rb{\ol U_{i}-\ol U_{i-1}}^2\ .
\end{gather}%
Verify \eqref{eqSW21}.


\item
The coefficients $d_r$ are chosen so that
\begin{gather}%
	\sum_{r=0}^{k-1} d_rv_{i+1/2}\hpar{r}=v\frb{x_{i+1/2}}+O\frb{\Del x^{2k-1}}\ .
\end{gather}%
For $k=2,3$,
\begin{align}
\label{dk2}
	k=2: \quad & d_{0}=\frac{2}{3}\ ,\quad d_{1}=\frac{1}{3}\\
	k=3: \quad & d_{0}=\frac{3}{10}\ ,\quad d_{1}=\frac{3}{5}\ ,\quad d_{2}=\frac{1}{10}\ .
\end{align}
Explain how these values are obtained and verify \eqref{dk2}.

\end{enumerate}

\item
Write a program that implements WENO reconstruction for both $k=2,3$. The indicators $\beta_r$ for $k=3$ are in the lecture notes.

\item Write a finite volume code for
\[
u_t + au_x = 0
\]
where the interface values are obtained by using the WENO reconstruction. Plug these values in the Godunov flux, and integrate the semi-discrete scheme using SSP-RK3. Implement the following initial conditions on the domain $[-1,1]$ with periodic BC
\begin{gather} \label{inData1}
	v_0(x)=\sin(\pi x)\ ,  
	\quad
	T_f = 5
\end{gather}%
\begin{gather} \label{inData2}
	v_0(x)=\begin{cases}
		1 & |x|<0.5\\
		-1 & |x|>0.5
	\end{cases}\ ,
	\quad
	T_f = 0.5.
\end{gather}%
Use a CFL of 0.5 to evaluate the time-step. 

\item
Run the code for $k=2,3$ and $a=1$. Do you recover the expected order of convergence for \eqref{inData1}? Is the solution oscillatory for \eqref{inData2}?



\end{exerciseList}

\end{document}
