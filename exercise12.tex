\documentclass{article}



\usepackage{../auxFiles/ExStyPac}



\input ../auxFiles/mac.tex
\graphicspath{{./figures/}}





%========================================================================
\begin{document}



\Head{12}{WENO Reconstruction}








\begin{exerciseList}


\item
The WENO reconstruction is based on a convex combination, with coefficients $\om_r$, of approximations $v_{i+1/2}\hpar{r}$ calculated on $k$ different stencils.
The coefficients $\om_r$ depend on another set of coefficients $d_r$ and some smoothness indicators $\beta_r$.

\begin{enumerate}



\item
The Smoothness indicators $\beta_r$ are defined by
\begin{gather} \label{smWts}
	\beta_r=\sum_{l=1}^{k-1}\Delta x^{2l-1}
		\int_{x_i-1/2}^{x_{i+1/2} }\rb{\frac{\dif^lp_r}{\dif x^l}\rb{x}}^2\dif x
	\qquad\qquad r=0,\ldots,k-1\ .
\end{gather}%
For $k=2$ (3rd-order reconstruction), this yields the following expressions:
\begin{gather} \label{eqSW21}
	\beta_0 =\rb{\ol U_{i+1}-\ol U_{i}}^2\ ,
	\qquad\qquad
	\beta_1 =\rb{\ol U_{i}-\ol U_{i-1}}^2\ .
\end{gather}%
Verify \eqref{eqSW21}.


\item
The coefficients $d_r$ are chosen so that
\begin{gather}%
	\sum_{r=0}^{k-1} d_rv_{i+1/2}\hpar{r}=v\frb{x_{i+1/2}}+O\frb{\Del x^{2k-1}}\ .
\end{gather}%
For $k=2,3$,
\begin{align}
\label{dk2}
	k=2: \quad & d_{0}=\frac{2}{3}\ ,\quad d_{1}=\frac{1}{3}\\
	k=3: \quad & d_{0}=\frac{3}{10}\ ,\quad d_{1}=\frac{3}{5}\ ,\quad d_{2}=\frac{1}{10}\ .
\end{align}
Explain how these values are obtained and verify \eqref{dk2}.

\end{enumerate}

\item
Write a program that implements WENO reconstruction for both $k=2,3$.
% The indicators $\beta_r$ for $k=3$ are in the lecture notes. %TODO

\item Write a finite volume code for
\[
u_t + au_x = 0
\]
where the interface values are obtained by using the WENO reconstruction. Plug these values in the Godunov flux, and integrate the semi-discrete scheme using SSP-RK3. Implement the following initial conditions on the domain $[-1,1]$ with periodic BC
\begin{gather} \label{inData1}
	v_0(x)=\sin(\pi x)\ ,  
	\quad
	T_f = 5
\end{gather}%
\begin{gather} \label{inData2}
	v_0(x)=\begin{cases}
		1 & |x|<0.5\\
		-1 & |x|>0.5
	\end{cases}\ ,
	\quad
	T_f = 0.5.
\end{gather}%
Use a CFL of 0.5 to evaluate the time-step. 

\item
Run the code for $k=2,3$ and $a=1$. Do you recover the expected order of convergence for \eqref{inData1}? Is the solution oscillatory for \eqref{inData2}?



\end{exerciseList}

\end{document}



\documentclass{article}



\usepackage{../auxFiles/ExStyPac}

\usepackage[framed]{../auxFiles/mcode}





\input ../auxFiles/mac.tex
\graphicspath{{./figures/}}





%========================================================================
\begin{document}



\Head{12}{WENO Reconstruction (Solution)}




\begin{exerciseList}






\item
\begin{enumerate}

\item
We start the calculation by writing
\begin{gather}
	P_{r}(x)=\sum_{j=0}^{k}V\frb{x_{i-r+j-1/2}} \ell_{j}\frb{\frac{x-x_{i-r-1/2}}{h}}\ ,
\end{gather}
where $\ell_{j}$ are the Lagrange polynomials.
For $k=2$, we have
\begin{gather}
	P_{r}(x)= & \frac{V\frb{x_{i-r-0.5}}}{2h^{2}}\left(x-x_{i-r+1/2}\right)\left(x-x_{i-r+3/2}\right)\\
		& \quad -\frac{V\left(x_{i-r+1/2}\right)}{h^{2}}\left(x-x_{i-r-1/2}\right)\left(x-x_{i-r+3/2}\right)\\
		& \quad +\frac{V\left(x_{i-r+3/2}\right)}{2h^{2}}\left(x-x_{i-r-1/2}\right)\left(x-x_{i-r+1/2}\right)
\end{gather}
which implies 
\begin{gather}
	\frac{\dif p_{r}}{\dif x}(x) &=\frac{\dif^{2}P_{r}}{\dif x^{2}}(x)
			=\frac{1}{h^{2}}\Big[V\frb{x_{i-r+3/2}}-2V\frb{x_{i-r+1/2}}+V\frb{x_{i-r-1/2}} \Big]\\[0.5em]
		&=\frac{1}{h}\left(\overline{U}_{i-r+1}-\overline{U}_{i-r}\right)
\end{gather}
Therefore,
\begin{gather}
	\beta_{r}=h\intop_{x_{i-1/2}}^{x_{i+1/2}}\frac{1}{h^{2}}\left(\overline{U}_{i-r+1}-\overline{U}_{i-r}\right)^{2}\dif x
		=\left(\overline{U}_{i-r+1}-\overline{U}_{i-r}\right)^{2}
\end{gather}
which completes the proof.


\item
Suppose $v$ is a smooth function, and for each $i$, its average on the $i$th cell $I_i=(x_{i-1/2},x_{i+1/2})$ is $\ol v_i$.
Fix some $i$. The reconstruction of the value of $v$ at (the cell boundary) $x_{i+1/2}$ obtained from the $r$th stencil of size $k=2$ is given by
\begin{gather}
	v\hpar{r}_{i+1/2}=\sum_{j=0}^1 c_{rj}\ol v_{i-r+j}\ .
\end{gather}
Explicitly, we have
\begin{gather} \label{Reconk2}
	v\hpar{0}_{i+1/2}=\frac{1}{2}\ol v_{i}+\frac{1}{2}\ol v_{i+1}
	\quad
	v\hpar{1}_{i+1/2}=-\frac{1}{2}\ol v_{i-1}+\frac{3}{2}\ol v_{i}\ .
\end{gather}
The weights $d_r$ are chosen so that
\begin{gather}
	\sum_{r=0}^1 d_r v\hpar{r}_{i+1/2}=v\frb{x_{i+1/2}}+O\frb{h^3}\ .
\end{gather}
On the other hand, by \eqref{Reconk2},
\begin{gather}
	\sum_{r=0}^1 d_r v\hpar{r}_{i+1/2}
		=-\frac{d_1}{2}\ol v_{i-1}+\rb{\frac{d_0}{2}+\frac{3d_1}{2}}\ol v_{i}+\frac{d_0}{2}\ol v_{i+1}\ .
\end{gather}
This is a 3rd-order approximation to $v\frb{x_{i+1/2}}$ obtained from $\ol v_{i-1+j}$, with $j=1,2,3$.
This is simply the reconstruction at $x_{i+1/2}$ obtained from the stencil of size $k=3$, associated with $r=1$:
\begin{gather}
	\wt v\hpar{1}_{i+1/2}=-\frac{1}{6}\ol v_{i-1}+\frac{5}{6}\ol v_{i} +\frac{1}{3}\ol v_{i+1}
\end{gather}
By requiring
\begin{gather}
	\wt v\hpar{1}_{i+1/2}=\sum_{r=0}^1 d_r v\hpar{r}_{i+1/2}
\end{gather}
and comparing coefficients, we get
\begin{gather}
	d_0=\frac{2}{3}\ ,
	\quad
	d_1=\frac{1}{3}\ .
\end{gather}


\end{enumerate}

\item See codes {\tt WENO.m} and {\tt ReconstructWeights.m} attached at the end of the solution manual. In Exercise 11, we had written a slightly different code to evaluate the coefficients $c_{rj}$. A more compact algorithm is used for this exercise (also available in the Software folder uploaded on the moodle platform).

\item See codes attached at the end of the solution manual.

\item If you look at the plots generated by the code, we obtain third-order convergence for k=2,3 with a smooth initial condition. Even though we use a fifth-order discretization in space, we are limited by the fact the time integration is only third-order accurate, and $\Delta t \sim h$. Furthermore, the solutions are non-oscillatory with the discontinuous initial condition.

\end{exerciseList}


\lstinputlisting{./Code/Exercise12.m}

\lstinputlisting{./Code/solver.m}

\lstinputlisting{./Code/apply_bc.m}

\lstinputlisting{./Code/evalRHS.m}

\lstinputlisting{./Code/ReconstructWeights.m}

\lstinputlisting{./Code/WENO.m}

\lstinputlisting{./Code/find_exact.m}

\lstinputlisting{./Code/find_err.m}



\end{document}
