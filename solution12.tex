\documentclass{article}



\usepackage{../auxFiles/ExStyPac}

\usepackage[framed]{../auxFiles/mcode}





\input ../auxFiles/mac.tex
\graphicspath{{./figures/}}





%========================================================================
\begin{document}



\Head{12}{WENO Reconstruction (Solution)}




\begin{exerciseList}






\item
\begin{enumerate}

\item
We start the calculation by writing
\begin{gather}
	P_{r}(x)=\sum_{j=0}^{k}V\frb{x_{i-r+j-1/2}} \ell_{j}\frb{\frac{x-x_{i-r-1/2}}{h}}\ ,
\end{gather}
where $\ell_{j}$ are the Lagrange polynomials.
For $k=2$, we have
\begin{gather}a
	P_{r}(x)= & \frac{V\frb{x_{i-r-0.5}}}{2h^{2}}\left(x-x_{i-r+1/2}\right)\left(x-x_{i-r+3/2}\right)\\
		& \quad -\frac{V\left(x_{i-r+1/2}\right)}{h^{2}}\left(x-x_{i-r-1/2}\right)\left(x-x_{i-r+3/2}\right)\\
		& \quad +\frac{V\left(x_{i-r+3/2}\right)}{2h^{2}}\left(x-x_{i-r-1/2}\right)\left(x-x_{i-r+1/2}\right)
\end{gather}a
which implies 
\begin{gather}a
	\frac{\dif p_{r}}{\dif x}(x) &=\frac{\dif^{2}P_{r}}{\dif x^{2}}(x)
			=\frac{1}{h^{2}}\Big[V\frb{x_{i-r+3/2}}-2V\frb{x_{i-r+1/2}}+V\frb{x_{i-r-1/2}} \Big]\\[0.5em]
		&=\frac{1}{h}\left(\overline{U}_{i-r+1}-\overline{U}_{i-r}\right)
\end{gather}a
Therefore,
\begin{gather}
	\beta_{r}=h\intop_{x_{i-1/2}}^{x_{i+1/2}}\frac{1}{h^{2}}\left(\overline{U}_{i-r+1}-\overline{U}_{i-r}\right)^{2}\dif x
		=\left(\overline{U}_{i-r+1}-\overline{U}_{i-r}\right)^{2}
\end{gather}
which completes the proof.


\item
Suppose $v$ is a smooth function, and for each $i$, its average on the $i$th cell $I_i=(x_{i-1/2},x_{i+1/2})$ is $\ol v_i$.
Fix some $i$. The reconstruction of the value of $v$ at (the cell boundary) $x_{i+1/2}$ obtained from the $r$th stencil of size $k=2$ is given by
\begin{gather}
	v\hpar{r}_{i+1/2}=\sum_{j=0}^1 c_{rj}\ol v_{i-r+j}\ .
\end{gather}
Explicitly, we have
\begin{gather} \label{Reconk2}
	v\hpar{0}_{i+1/2}=\frac{1}{2}\ol v_{i}+\frac{1}{2}\ol v_{i+1}
	\dqquad
	v\hpar{1}_{i+1/2}=-\frac{1}{2}\ol v_{i-1}+\frac{3}{2}\ol v_{i}\ .
\end{gather}
The weights $d_r$ are chosen so that
\begin{gather}
	\sum_{r=0}^1 d_r v\hpar{r}_{i+1/2}=v\frb{x_{i+1/2}}+O\frb{h^3}\ .
\end{gather}
On the other hand, by \deq{Reconk2},
\begin{gather}
	\sum_{r=0}^1 d_r v\hpar{r}_{i+1/2}
		=-\frac{d_1}{2}\ol v_{i-1}+\rb{\frac{d_0}{2}+\frac{3d_1}{2}}\ol v_{i}+\frac{d_0}{2}\ol v_{i+1}\ .
\end{gather}
This is a 3rd-order approximation to $v\frb{x_{i+1/2}}$ obtained from $\ol v_{i-1+j}$, with $j=1,2,3$.
This is simply the reconstruction at $x_{i+1/2}$ obtained from the stencil of size $k=3$, associated with $r=1$:
\begin{gather}
	\wt v\hpar{1}_{i+1/2}=-\frac{1}{6}\ol v_{i-1}+\frac{5}{6}\ol v_{i} +\frac{1}{3}\ol v_{i+1}
\end{gather}
By requiring
\begin{gather}
	\wt v\hpar{1}_{i+1/2}=\sum_{r=0}^1 d_r v\hpar{r}_{i+1/2}
\end{gather}
and comparing coefficients, we get
\begin{gather}
	d_0=\frac{2}{3}\ ,
	\quad
	d_1=\frac{1}{3}\ .
\end{gather}


\end{enumerate}

\item See codes {\tt WENO.m} and {\tt ReconstructWeights.m} attached at the end of the solution manual. In Exercise 11, we had written a slightly different code to evaluate the coefficients $c_{rj}$. A more compact algorithm is used for this exercise (also available in the Software folder uploaded on the moodle platform).

\item See codes attached at the end of the solution manual.

\item If you look at the plots generated by the code, we obtain third-order convergence for k=2,3 with a smooth initial condition. Even though we use a fifth-order discretization in space, we are limited by the fact the time integration is only third-order accurate, and $\Delta t \sim h$. Furthermore, the solutions are non-oscillatory with the discontinuous initial condition.

\end{exerciseList}


\lstinputlisting{./Code/Exercise12.m}

\lstinputlisting{./Code/solver.m}

\lstinputlisting{./Code/apply_bc.m}

\lstinputlisting{./Code/evalRHS.m}

\lstinputlisting{./Code/ReconstructWeights.m}

\lstinputlisting{./Code/WENO.m}

\lstinputlisting{./Code/find_exact.m}

\lstinputlisting{./Code/find_err.m}



\end{document}
