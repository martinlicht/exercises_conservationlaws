\documentclass[10pt,letterpaper]{article}
\usepackage[english]{babel}
\usepackage{graphicx}
\usepackage[margin=2cm]{geometry}

\usepackage{float}
\usepackage{subfloat}
\usepackage{caption}
\usepackage{subcaption}
\usepackage{listings}

\usepackage{mathtools}
\usepackage{amsfonts}
\usepackage{amsmath}
\usepackage{amssymb}
\usepackage{amsthm}
\newcommand{\dif}[1][]{\mathrm{d} {#1}\,}
\newcommand{\rb}[1]{ \left(  {#1} \right) }
\newcommand{\frb}[1]{ \left(  {#1} \right) }
\newcommand{\norm}[1]{ \left\|  {#1} \right\| }
\newcommand{\jph}{{j+\frac{1}{2}}}
\newcommand{\jmh}{{j-\frac{1}{2}}}


% \usepackage[amsmath,thmmarks,standard]{ntheorem}
\newtheoremstyle{break}
{\topsep}
{\topsep}%
{\normalfont}
{}%
{\bfseries}
{}%
{\newline}
{}%
\theoremstyle{break}
\newtheorem{exercise}{Exercise}
\newtheorem*{information}{Information}
\newtheorem{mysolution}{Solution}
% \newenvironment{solution}{\begin{comment}}{\end{comment}}
\newtheorem*{solutioninformation}{Solution Information}

\usepackage{comment}
% Switch between showing and hiding solutions by commenting out either of the following lines
\newenvironment{solution}{\begin{mysolution}}{\end{mysolution}}
% \excludecomment{solution}






\begin{document}

\title{Additional Finite Difference Methods}
\date{}

\maketitle





\begin{exercise}[Non-conservative scheme]
	Consider the non-conservative scheme for Burgers' equation:
	\begin{equation}
		U^{n+1}_j = 	U^{n}_j +  	2 \frac{k}{h} U^{n}_j \left( 	U^{n}_j - U^{n}_{j-1} \right).
	\end{equation}
	Show that this scheme maps non-negative monotonely increasing sequences onto non-negative monotonely increasing sequences. 

	\textit{Remark: while this scheme is not conservative, is still of interest in a few contexts. The non-negativity condition is due to the upwinding implicit in the scheme.}
\end{exercise}

\begin{solution}

\end{solution}


\begin{exercise}[Monotonicity of FTCS]
	What is the flux of the FTCS scheme for the transport equation? Relate it to other fluxes that you know. Determine whether the FTCS scheme is monotone.
\end{exercise}

\begin{solution}

\end{solution}








\begin{exercise}[3. Order of schemes]
	Show that the FTFS scheme for the transport equation is of order $(1,1)$ and that the FTCS scheme is of order $(2,1)$.
\end{exercise}

\begin{solution}

\end{solution}





\begin{exercise}[Order of schemes]
	Under what conditions is a linear scheme $F(U,V) = w_1 U + w_2 V$ a monotone scheme?
	Describe the discrete Kruzkov entropy entropy-flux pairs.
\end{exercise}

\begin{solution}

\end{solution}







\begin{exercise}[Roe flux]
	Show that the Roe flux can be written as 
	\begin{equation}
		F(U,V) = \frac{ f(V) + f(U) }{2} + \frac{1}{2} \operatorname{sgn}( V - U ) |f(V) - f(U)|.
	\end{equation}
	Either show that the Roe flux is monotonicity-preserving or give a counter example. 
\end{exercise}

\begin{solution}

\end{solution}







\end{document}
