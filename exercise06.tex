%Typeset with LaTeX
%\documentclass{owrart}
\documentclass[10pt,reqno]{amsart}
\usepackage[a4paper,margin=1.0in]{geometry}
\UseRawInputEncoding
%% -------------------------------------------------------------------------------
%% Enter additionally required packages below this comment.
\usepackage{algorithm}
\usepackage{algpseudocode}

\makeatletter
% Reinsert missing \algbackskip
\def\algbackskip{\hskip-\ALG@thistlm}
\makeatother

\usepackage{amsmath}
\usepackage{mathtools}
%\usepackage[a4paper]{geometry}
%\usepackage{amsfonts}
\usepackage{amssymb}
\usepackage{amsthm}
\usepackage{mathrsfs}
\usepackage{bm}
\usepackage{algpseudocode}
\usepackage{upgreek}
\usepackage{subcaption}
% \usepackage{showkeys} %%%%%%%%%
% \usepackage{amscd}
% \usepackage{latexsym}
% \usepackage{nicefrac} 3
% \usepackage{cite}
 %\usepackage{epic}
% \usepackage{eepic}
% \usepackage{fancybox}
\newcommand{\bsnul}{{\boldsymbol 0}}
\newcommand{\cNN}{{\mathcal {NN}}}

\usepackage{graphicx}
\usepackage{epstopdf}

\mathtoolsset{showonlyrefs}

%\usepackage{epstopdf}
%\usepackage{graphicx} 

%\usepackage[pdf]{pstricks}

\usepackage[shortlabels]{enumitem}
%Standard operator
\newcommand{\abs}[1]{\lvert #1\rvert}
\newcommand{\supp}{Peratorname{supp}}
\newcommand{\grad}{{\mathbf{grad}}}
\newcommand{\curl}{\normalfont{\text{curl}}}
\newcommand{\Log}{\text{Log}}
\newcommand{\cc}{{\normalfont{\text{c}}}}
\newcommand{\loc}{{\normalfont{\text{loc}}}}

\DeclareMathOperator*{\argmax}{arg\,max}
\DeclareMathOperator*{\argmin}{arg\,min}

% Package for colors
\usepackage{xcolor}
\definecolor{ao}{rgb}{0.0, 0.5, 0.0}

%Edit colors
\newcommand{\CS}[1]{{\color{magenta}{#1}}}
\newcommand{\ra}[1]{{\color{blue}{#1}}}
\newcommand{\fh}[1]{{\color{cyan}{#1}}}
\newcommand{\todo}[1]{{\color{red}[#1]}}

\newcommand{\com}[1]{{\color{red} [#1]}}

%\usepackage{fancyhdr}
%\usepackage{datetime}
%
%\fancyhf{}
%\fancyfoot[L]{\fontsize{10}{12} \today \ \currenttime}
%\pagestyle{fancy} \usepackage{tikz}

\DeclareSymbolFont{extraup}{U}{zavm}{m}{n}
\DeclareMathSymbol{\varheart}{\mathalpha}{extraup}{86}
\DeclareMathSymbol{\vardiamond}{\mathalpha}{extraup}{87}

\usepackage{chngcntr}
\counterwithout{equation}{subsection}
\counterwithout{equation}{section} 

\usepackage{tikz}

\newcommand\encircle[1]{%
  \tikz[baseline=(X.base)] 
    \node (X) [draw, shape=circle, inner sep=0] {\strut #1};}

\usepackage{pifont}

\usepackage{fancyhdr}
\usepackage[us,12hr]{datetime} % `us' makes \today behave as usual in TeX/LaTeX
\fancypagestyle{plain}{
\fancyhf{}
\rhead{\footnotesize Version of \today}
\lhead{\thepage}
\renewcommand{\headrulewidth}{0pt}}
\pagestyle{plain}

\usepackage{color}
\newcommand{\half}{\frac{1}{2}}

\newcommand{\norm}[1]{\left \lVert #1 \right \rVert}
\newcommand{\snorm}[1]{\left \lvert #1 \right \rvert}

%fields, mathbb operators
\newcommand{\IR}{\mathbb{R}}
\newcommand{\IC}{\mathbb{C}}
\newcommand{\IZ}{\mathbb{Z}}
\newcommand{\IQ}{\mathbb{Q}}
\newcommand{\IN}{\mathbb{N}}
\newcommand{\IT}{\mathbb{T}}

% Boundary Integral Operators
\newcommand{\OV}{\mathsf{V}}
\newcommand{\OK}{\mathsf{K}}
\newcommand{\OW}{\mathsf{W}}
% 

%
%\DeclareMathOperator*{\argmax}{arg\,max}
%\DeclareMathOperator*{\argmin}{arg\,min}

\newcommand{\OA}{\mathsf{A}}
\newcommand{\Id}{\mathsf{Id}}
\newcommand{{\D}}{\normalfont{\text{D}}}
\newcommand{{\G}}{\normalfont{\text{G}}}
\newcommand{{\U}}{\normalfont{\text{U}}}
\newcommand{{\I}}{\normalfont{\text{I}}}
\newcommand{{\per}}{\normalfont{\text{per}}}
\newcommand{{\y}}{{\boldsymbol{{y}}}}
\newcommand{\Of}{\mathsf{f}}

\newcommand{{\bc}}{{\bf c}}
\newcommand{{\bx}}{{\bf x}}
\newcommand{{\by}}{{\bf y}}
\newcommand{{\bz}}{{\bf z}}
\newcommand{{\bd}}{{\bf d}}
\renewcommand{\d}{\!\!\operatorname{d}}

% Traces
\newcommand{\traced}{\gamma_{\text{D}}}
\newcommand{\tracen}{\gamma_{\text{N}}}
\newcommand{\dual}[2]{\left \langle #1,#2 \right \rangle}
\newcommand{\dotp}[2]{\left ( #1,#2 \right )}

% Matrix trace
\newcommand{\tr}[1]{\text{tr}\left ( #1 \right )}

\newtheorem{theorem}{Theorem}[section]
\newtheorem{corollary}[theorem]{Corollary}
\newtheorem{lemma}[theorem]{Lemma}
\newtheorem{prob}[theorem]{Problem}
\newtheorem{definition}[theorem]{Definition}
\newtheorem{assumption}[theorem]{Assumption}
\newtheorem{proposition}[theorem]{Proposition}
\newtheorem{example}[theorem]{Example}
\newtheorem{problem}[theorem]{Problem}

\theoremstyle{remark}
\newtheorem{remark}{Remark}
%\newtheorem{assumption}[remark]{Assumption}
%\newtheorem{example}[remark]{Example}
%\newtheorem{problem}[thm]{Problem}


%\numberwithin{equation}{section}

\begin{document}
%\begin{talk}
%
%---------
% TITLE
%---------
%

\title{Exercise Set 3\\ 
MATH-459--Numerical Methods for Conservation Laws \\  
Autumn Semester 2021
}

\author{Professor: Martin W. Licht \\ Assistant: Fernando Henriquez B. \\ Published on 6.10.21} 

\maketitle



\subsection*{1. Non-conservative scheme}
${}$\\

\noindent Consider the non-conservative scheme for Burgers' equation:
\begin{equation}
	U^{n+1}_j = 	U^{n}_j +  	2 \frac{k}{h} U^{n}_j \left( 	U^{n}_j - U^{n}_{j-1} \right).
\end{equation}
Show that this scheme maps non-negative monotonely increasing sequences onto non-negative monotonely increasing sequences. 

\textit{Remark: while this scheme is not conservative, is still of interest in a few contexts. The non-negativity condition is due to the upwinding implicit in the scheme.}
${}$\\



\subsection*{2. Monotonicity of FTCS}
${}$\\


\noindent What is the flux of the FTCS scheme for the transport equation? Relate it to other fluxes that you know. Determine whether the FTCS scheme is monotone.
${}$\\




\subsection*{3. Order of schemes}
${}$\\


\noindent Show that the FTFS scheme for the transport equation is of order $(1,1)$ and that the FTCS scheme is of order $(2,1)$.
${}$\\




\subsection*{4. Order of schemes}
${}$\\


\noindent Under what conditions is a linear scheme $F(U,V) = w_1 U + w_2 V$ a monotone scheme?
Describe the discrete Kruzkov entropy entropy-flux pairs.
${}$\\




\subsection*{5. Roe flux}
${}$\\

\noindent Show that the Roe flux can be written as 
\begin{equation}
	F(U,V) = \frac{ f(V) + f(U) }{2} + \frac{1}{2} \operatorname{sgn}( V - U ) |f(V) - f(U)|.
\end{equation}
Either show that the Roe flux is monotonicity-preserving or give a counter example. 
${}$\\




% \end{exerciseList}

\end{document}


\Head{6}{Godunov's Method}







\item
Discuss qualitatively the derivation of \emph{Godunov's method}; sketch each step in the solution process. Which part of the algorithm can make its implementation particularly difficult?

\item
Consider the scalar conservation law
\begin{gather}%
	u_t+f(u)_x=0\ ,
\end{gather}%
and initial condition
\begin{gather}%
	u(x,0)=\begin{cases}
		u_l & x<0\\
		u_r & 0<x
	\end{cases}\ ,
\end{gather}%
where the flux $f$ is convex ($f''>0$).
Godunov's method relies on finding the intermediate state $u^*=u^*(u_l,u_r)$ for which $u(0,t)=u^*$, for $t>0$.
\begin{enumerate}
\item
Show that this intermediate state is given by the following:
\begin{align*}
	&1.\ \ f'(u_l), f'(u_r)\ge 0 \ \ \Longrightarrow \ \ u^*=u_l \\
	&2.\ \ f'(u_l), f'(u_r)\le 0 \ \ \Longrightarrow \ \ u^*=u_r \\[0.5em]
	&3.\ \ f'(u_l) \ge 0\ge f'(u_r) \ \ \Longrightarrow \ \
		u^*=\begin{cases}
			u_l & s>0 \\
			u_r & s<0
		\end{cases}\ ,
		\qquad s=\frac{f(u_r)-f(u_r)}{u_r-u_l}\\[0.5em]
	&4.\ \ f'(u_l) < 0 < f'(u_r)  \ \ \Longrightarrow \ \ u^*=u_m\ ,
		\qquad\text{where $u_m$ is the solution to $f'\frb{u_m}=0$.}
\end{align*}

\item
Use (a) to show that Godunov's flux is given by
\begin{gather} \label{GodFlx}
	F(u_l,u_r)=\begin{cases}
			\displaystyle
			\min_{u_l\le u\le u_r} f(u) & u_l\le u_r \\[1em]
			\displaystyle
			\max_{u_r\le u\le u_l} f(u) & u_l> u_r 
		\end{cases}\ .
\end{gather}%
\item
Show that Godunov's flux \eqref{GodFlx} is monotone.
\end{enumerate}



\item
The purpose of this exercise is to illustrate the \emph{Lax-Wendroff Theorem}.
This theorem states that if there exists a sequence $\{(h_l,k_l)\}_l^\infty$ with $k_l = \lambda h_l$ ($\lambda$ is kept constant), such that the corresponding numerical solutions $\left\{ v_{l}\right\} _{l=1}^{\infty}$ obtained by a conservative method converges to some function $u$, then the limit $u$ is a weak solution of the conservation law. Notice that to deduce the conclusion, we assume that $v_{l}$ converges as $l\rightarrow\infty$.
That is, convergence is not a conclusion of the Lax-Wendroff theorem.
Also recall that in general weak solutions are not unique, so the theorem does not guarantee the limit is the correct entropy solution. 

Consider a conservative method 
\begin{gather}%
	v_{j}^{n+1}=v_{j}^{n} -\frac{k}{h}\sqb{F\frb{v_{j}^{n},v_{j+1}^{n}}-F\frb{v_{j-1}^{n},v_{j}^{n}}}
\end{gather}%
where the numerical flux F is given by 
\begin{gather}%
	F\frb{v,w}=\begin{cases}
		f(v) & \frac{f(v)-f\frb{w}}{v-w}\geq0\\[0.5em]
		f\frb{w} & \frac{f(v)-f\frb{w}}{v-w}<0
	\end{cases}\ .
\end{gather}%
\begin{enumerate}
\item
Construct the entropy solution to the following initial value problem
\begin{gather} \label{IVP}
	u_{t}+\rb{\frac{1}{2}u^{2}}_{x}=0
	\quad
		u(x,0)=\begin{cases}
		-1 & x<1\\
		1 & x>1
	\end{cases}\ .
\end{gather}%
%satisfying the initial condition
%\begin{gather} \label{incond2}
%	u(x,0)=\begin{cases}
%		-1 & x<1\\
%		1 & x>1
%	\end{cases}\ .
%\end{gather}%
\item
Fix $k/h=0.5$, and implement the above method to \eqref{IVP}, in $x\in(0,2)$, $0<t\le0.25$, with the initial data 
discretized using cell averages. On the boundaries, set $u(0,t)=-1$, and $u\frb{2,t}=1$.

\item Run the computations by choosing i) $h_l = \frac{2}{l}$, ii)$h_l = \frac{2}{2 l}$, iii) $h_l = \frac{2}{2 l + 1}$, for $l\in\mathbb{N}$.

\item
What can you deduce from your results regarding the each of the three sequences of numerical solutions obtained.
Explain your results and conclude how they fit with the Lax-Wendroff theorem.
\end{enumerate}










\documentclass{article}



\usepackage{../auxFiles/ExStyPac}


\usepackage[framed]{../auxFiles/mcode}



\usepackage{bm}

\input ../auxFiles/mac.tex






%========================================================================
\begin{document}



\Head{6}{Godunov's Method (Solution)}





\begin{exerciseList}


\item

Godunov's method can be outlined in two steps.
Suppose we have an approximation $v^n$ of the solution $u$ at $t_n$.
(i) Define $\widetilde{u}^{n}(x,t)$ for all $x$ and $t_n<t<t_{n+1}=t_n+k$ as the exact solution to the conservation law, satisfying the initial condition
\begin{gather} \label{GodIn}
	\wt u^n\frb{x,t_n}=v^n_j
	\quad
	x\in(x_{j-1/2},x_{j+1/2})
	\quad\forall j\ .
\end{gather}
(ii) Average the resulting function $\widetilde{u}^{n}\frb{x,t_{n+1}}$ over each cell $(x_{j-1/2},x_{j+1/2})$ to obtain the approximation
\begin{gather} \label{GogAvg}
	v_{j}^{n+1}=\frac{1}{h}\int_{x_{j-1/2}}^{x_{j+1/2}}\tilde{u}^{n}\frb{x,t_{n+1}}\dif x
\end{gather}
at $t_{n+1}$.
Now this procedure can be repeated to advance to the next time-step.

In Step (i), we need to solve an exact Riemann problem at each cell-interface, over a small time interval $(t_n,t_{n+1})$. Since $\wt u^n$ is a solution to the conservation law, \eqref{GogAvg} yields
\begin{gather} \label{Godavg2}
	v_{j}^{n+1} &=\frac{1}{h}\int_{x_{j-1/2}}^{x_{j+1/2}}\tilde{u}^{n}\frb{x,t_{n}}\dif x\\
		&\quad -\frac{1}{h}\rb{
			\int_{t_n}^{t_n+k} f\frb{\tilde{u}^{n}\frb{x_{j-1/2},t}}\dif t
			-\int_{t_n}^{t_n+k} f\frb{\tilde{u}^{n}\frb{x_{j-1/2},t}}\dif t}\ .
\end{gather}
First notice that
\begin{gather}
	v^n_j=\frac{1}{h}\int_{x_{j-1/2}}^{x_{j+1/2}}\tilde{u}^{n}\frb{x,t_{n}}\dif x\ ,
\end{gather}
since $\wt u^n$ satisfies \eqref{GodIn}.
Now, let us look at the other two integrals in \eqref{Godavg2}.
Because $\wt u^n\frb{\cdot,t_n}$ is piecewise constant, with a discontinuity at each $x_{j-1/2}$, and given that the time step $k$ is sufficiently small, $\wt u^n\frb{x_{j-1/2},\cdot}$ is constant. Let
\begin{gather}
	\tilde{u}^{n}\frb{x_{j-1/2},t}=u^*\frb{v^n_{j-1},v^n_{j}}
	\quad
	t\in(t_n,t_{n+1})\ .
\end{gather}
So, if we take
\begin{gather}
	F(u_l,u_r)=f\frb{u^*(u_l,u_r)}=\frac{1}{k}\int_{t_n}^{t_n+k} f\frb{u^*(u_l,u_r)}\dif t\ ,
\end{gather}
then \eqref{GogAvg} becomes
\begin{gather} \label{GodCons}
	v^{n+1}_j=v^n_j-\frac{k}{h}\Big[ F\frb{v^n_{j+1},v^n_{j}} -F\frb{v^n_{j},v^n_{j-1}}\Big]\ .
\end{gather}
This is how the method is applied in practice, provided we can find $u^*$. Unfortunately, evaluating the intermediate can be very expensive, and at times impossible. This motivates the need to construct \textit{approximate Riemann solvers}.



\item
Consider the scalar conservation law
\begin{gather}
	u_t+f(u)_x=0\ ,
\end{gather}
and initial condition
\begin{gather}
	u(x,0)=\begin{cases}
		u_l & x<0\\
		u_r & 0<x
	\end{cases}\ ,
\end{gather}
where the flux $f$ is convex ($f''>0$).
Godunov's method relies on finding the intermediate state $u^*=u^*(u_l,u_r)$ for which $u(0,t)=u^*$, for $t>0$.
\begin{enumerate}
\item
We show that $u^*$ is given by
\begin{align*}
	&1.\ \ f'(u_l), f'(u_r)\ge 0 \ \ \Longrightarrow \ \ u^*=u_l \\
	&2.\ \ f'(u_l), f'(u_r)\le 0 \ \ \Longrightarrow \ \ u^*=u_r \\[0.5em]
	&3.\ \ f'(u_l) \ge 0\ge f'(u_r) \ \ \Longrightarrow \ \
		u^*=\begin{cases}
			u_l & s>0 \\
			u_r & s<0
		\end{cases}\ ,
		\qquad s=\frac{f(u_r)-f(u_r)}{u_r-u_l}\\[0.5em]
	&4.\ \ f'(u_l) < 0 < f'(u_r)  \ \ \Longrightarrow \ \ u^*=u_m\ ,
		\qquad\text{where $u_m$ is the solution to $f'\frb{u_m}=0$.}
\end{align*}
Note that since $f$ is strictly convex, the Jacobian $f^\prime$ is a strictly increasing function.

Suppose $f'(u_l), f'(u_r)\ge 0$.
If $u_l>u_r$, the entropy solution is a shock moving at speed given by the RH condition
\begin{gather} \label{RHJumpCond}
	s=\frac{f(u_l)-f(u_r)}{u_l-u_r}\,
\end{gather}
and $f'(u_l) > s > f'(u_r)$ according to the entropy condition.
This implies that the shock speed is positive, and thus we have $u^*=u_l$. 
If $u_l\le u_r$, the entropy solution is a rarefaction wave.
Since $f'(u_l)>0$, the left front of the wave moves to the right, and thus $u^*=u_l$.

Next suppose $f'(u_l), f'(u_r)\le 0$.
By similar arguments we get $u^*=u_r$.

If $f'(u_l) \ge 0\ge f'(u_r)$, the entropy solution is a shock, and the intermediate state $u^*$ is determined by the sign of the shock speed \eqref{RHJumpCond}.

Finally suppose $f'(u_l) < 0 < f'(u_r)$.
Then, the entropy solution is a rarefaction wave.
This time $x=0$ falls inside the rarefaction fan.
As we have seen in a previous exercise (see exercise set 3), the rarefaction solution is given by $u(x,t)=w\frb{x/t}$, where $w$ is the solution to $f'\frb{w\frb{\xi}}=\xi$.
Since we are interested in the value of $u$ at $x=0$, we have $u^*=w\frb{0}$.

\item
Before we begin, let as remark that since $f$ is strictly convex, the its maximum on a given closed interval $[u_1,u_2]$ is achieved its maximum at an end point of the interval. Furthermore, there exists a unique $\theta$ where $f^\prime$ vanishes. This point corresponds to the global minima of $f$. If $\theta \in (u_1,u_2)$, then f achieves its minimum at $\theta$, else at an end point of the interval.
To show that Godunove's flux is given by
\begin{gather} \label{GodFlx}
	F(u_l,u_r)=\begin{cases}
			\displaystyle
			\min_{u_l\le u\le u_r} f(u) & u_l\le u_r \\[1em]
			\displaystyle
			\max_{u_r\le u\le u_l} f(u) & u_l> u_r 
		\end{cases}\ .
\end{gather}
we look at the different possible cases.

If $u_l>u_r$, the entropy solution is a shock.
In this case, $u^*$ is determined by the sign of $s$.
If $f(u_l)>f(u_r)$, then $u^*=u_l$, and if $f(u_l)<f(u_r)$, then $u^*=u_r$.
Either way, $f\frb{u^*}$ is the maximum of $f$ in $[u_r,u_l]$.

If $u_l\le u_r$, the entropy solution is a rarefaction wave.
This is possible only when 1,2 or 4 are valid.
If 1 is valid, $f$ is increasing in $[u_l,u_r]$, and $f(u_l)$ is the minimum of $f$ in this interval.
Similarly, if 2 is valid, $f$ is decreasing in $[u_l,u_r]$, and $f(u_r)$ is the minimum of $f$ in this interval.
If 4 is valid, $f$ achieves its minimum at the internal point $u_m$, where $f'\frb{u_m}=0$.


\item
To show that Goduno's flux, given by \eqref{GodFlx}, is monotone, we show that it is non-decreasing in its first argument and non-increasing in its second argument.
If $u_l<u_r$ and $\eps>0$ is small enough, then
\begin{gather}
	F\frb{u_l+\eps,u_r}=\min_{u_l+\eps\le u\le u_r} f(u) \ge \min_{u_l\le u\le u_r} f(u)
		=F(u_l,u_r)\ ,
\end{gather}
and if $u_l\ge u_r$, then
\begin{gather}
	F\frb{u_l+\eps,u_r}=\max_{u_r\le u\le u_l+\eps} f(u) \ge \max_{u_r\le u\le u_l} f(u)
		=F(u_l,u_r)\ .
\end{gather}
Similarly, one can show that $F$ is non-increasing in its second argument.
\end{enumerate}



\item\label{exLWTNonConvSeq}
\begin{enumerate}
\item
Since Burgers equation is convex and
\begin{gather} \label{InCond}
	u(x,0)=\begin{cases}
		-1 & x<1\\
		1 & x>1
	\end{cases}\ ,
\end{gather}
that is $u_l<u_r$, a shock is not admissible.
Therefore, the entropy solution must be a rarefaction wave.
The exact solution to the problem with initial data \eqref{InCond} is given by
\begin{gather}
	u(x,t)=\begin{cases}
		-1 & x\leq-t+1\\
		\frac{x-1}{t} & -t+1<x\leq t+1\\
		1 & \; x>t+1
	\end{cases}\ .
\end{gather}

\item[(b),(c)]
\addtocounter{enumii}{2}
See Matlab code attached at the end of this manual. Experiment with different values of $l$.

\item
We notice that in case of $k_{l}=\frac{1}{2l}$, for any choice
of $l\in\mathbb{N}$, we get an approximation to the entropy solution,
and that the sequence of solutions converges as $l\rightarrow\infty$.
In the case $k_{l}=\frac{1}{2l+1}$, for any choice of $l\in\mathbb{N}$ we get the entropy violating solution of a stationary discontinuity.
Therefore, the sequence of solutions obtained by taking $k_{l}=\frac{1}{2l+1}$ converges to the same entropy violating solution as $l\rightarrow\infty$.
Note that in either case, the converged solution is still a weak solution, which is in accordance with the Lax-Wendroff theorem. However, by taking $k_{l}=\frac{1}{l}$, the numerical solution does not converge to any solution with the current scheme.
\end{enumerate}


\end{exerciseList}


%\begin{figure}[H] 
%\center
%
%\subfloat[\label{fig:Exc4_2a_U}]{\input{Exc4_2a_U.tikz}
%
%}\hspace{10mm}\subfloat[\label{fig:Exc4_2a_C}]{\input{Exc4_2a_C.tikz}
%
%}
%
%\caption{\label{fig:Exc4_2b} Exercise 1.3b. Initial data $u_0(x)$ Eq. 16 applied to the inviscid burgers equation. The two shocks move away from each other with equal speed and never merge. \textbf{(a)} Initial data. \textbf{(b)} Characteristics. }
%
%\end{figure}

\newpage
\begin{lstlisting}

% Solution05: Problem 3
% This script was written for EPFL MATH459, Numerical Methods for
% Conservation Laws. The invisvid Burgers equation is solved using the
% Roe flux

clc
clear all
close all

% Initial left and right states for the Riemann Problem
U0 = @(x) -1*(x<1) + 1*(x>1);

FinalTime = 0.25;

% Roe flux
RoeFlux =@(UL,UR) 0.5*UL.^2.*( (UL+UR) >=0) + 0.5*UR.^2.*( (UL+UR) <0);


% Discretization
l = 50:10:100;
%l = 51:10:101;
%l = 50:5:100;

M = length(l);

for i=1:M
    h = 2/l(i);
    k = 0.5*h;
    xf = 0:h:2;
    xc = 0.5*h:h:2-0.5*h;
    N  = length(xc);
    
    % Finding cell-average values of initial condition
    % Since we are looking at first-order schemes, it is enough to
    % approximate the integrals using the mid-point quadrature rule. This
    % essentially means we take the cell-center values as the cell-averages
    U = U0(xc);
    
    time = 0;
    while(time<FinalTime)
        
        if(time+k>FinalTime)
            k = FinalTime-time;
        end
        
        U_ext = [U(1),U,U(end)];
        
        Flux  = RoeFlux(U_ext(1:end-1),U_ext(2:end));
        
        U = U - (k/h)*(Flux(2:end) - Flux(1:end-1));
        
        time = time + k;
    end
    
    figure(1)
    plot(xc,U,'LineWidth',2)
    title(['Solution on a mesh with h = 2/',num2str(l(i))])
    hold all
    pause(0.1)
    
end




\end{lstlisting}
\end{document}
