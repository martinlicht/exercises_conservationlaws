%Typeset with LaTeX
%\documentclass{owrart}
\documentclass[10pt,reqno]{amsart}
\usepackage[a4paper,margin=1.0in]{geometry}
\UseRawInputEncoding
%% -------------------------------------------------------------------------------
%% Enter additionally required packages below this comment.
\usepackage{algorithm}
\usepackage{algpseudocode}

\makeatletter
% Reinsert missing \algbackskip
\def\algbackskip{\hskip-\ALG@thistlm}
\makeatother

\usepackage{amsmath}
\usepackage{mathtools}
%\usepackage[a4paper]{geometry}
%\usepackage{amsfonts}
\usepackage{amssymb}
\usepackage{amsthm}
\usepackage{mathrsfs}
\usepackage{bm}
\usepackage{algpseudocode}
\usepackage{upgreek}
\usepackage{subcaption}
% \usepackage{showkeys} %%%%%%%%%
% \usepackage{amscd}
% \usepackage{latexsym}
% \usepackage{nicefrac} 3
% \usepackage{cite}
 %\usepackage{epic}
% \usepackage{eepic}
% \usepackage{fancybox}
\newcommand{\bsnul}{{\boldsymbol 0}}
\newcommand{\cNN}{{\mathcal {NN}}}

\usepackage{graphicx}
\usepackage{epstopdf}

\mathtoolsset{showonlyrefs}

%\usepackage{epstopdf}
%\usepackage{graphicx} 

%\usepackage[pdf]{pstricks}

\usepackage[shortlabels]{enumitem}
%Standard operator
\newcommand{\abs}[1]{\lvert #1\rvert}
\newcommand{\supp}{Peratorname{supp}}
\newcommand{\grad}{{\mathbf{grad}}}
\newcommand{\curl}{\normalfont{\text{curl}}}
\newcommand{\Log}{\text{Log}}
\newcommand{\cc}{{\normalfont{\text{c}}}}
\newcommand{\loc}{{\normalfont{\text{loc}}}}

\DeclareMathOperator*{\argmax}{arg\,max}
\DeclareMathOperator*{\argmin}{arg\,min}

% Package for colors
\usepackage{xcolor}
\definecolor{ao}{rgb}{0.0, 0.5, 0.0}

%Edit colors
\newcommand{\CS}[1]{{\color{magenta}{#1}}}
\newcommand{\ra}[1]{{\color{blue}{#1}}}
\newcommand{\fh}[1]{{\color{cyan}{#1}}}
\newcommand{\todo}[1]{{\color{red}[#1]}}

\newcommand{\com}[1]{{\color{red} [#1]}}

%\usepackage{fancyhdr}
%\usepackage{datetime}
%
%\fancyhf{}
%\fancyfoot[L]{\fontsize{10}{12} \today \ \currenttime}
%\pagestyle{fancy} \usepackage{tikz}

\DeclareSymbolFont{extraup}{U}{zavm}{m}{n}
\DeclareMathSymbol{\varheart}{\mathalpha}{extraup}{86}
\DeclareMathSymbol{\vardiamond}{\mathalpha}{extraup}{87}

\usepackage{chngcntr}
\counterwithout{equation}{subsection}
\counterwithout{equation}{section} 

\usepackage{tikz}

\newcommand\encircle[1]{%
  \tikz[baseline=(X.base)] 
    \node (X) [draw, shape=circle, inner sep=0] {\strut #1};}

\usepackage{pifont}

\usepackage{fancyhdr}
\usepackage[us,12hr]{datetime} % `us' makes \today behave as usual in TeX/LaTeX
\fancypagestyle{plain}{
\fancyhf{}
\rhead{\footnotesize Version of \today}
\lhead{\thepage}
\renewcommand{\headrulewidth}{0pt}}
\pagestyle{plain}

\usepackage{color}
\newcommand{\half}{\frac{1}{2}}

\newcommand{\norm}[1]{\left \lVert #1 \right \rVert}
\newcommand{\snorm}[1]{\left \lvert #1 \right \rvert}

%fields, mathbb operators
\newcommand{\IR}{\mathbb{R}}
\newcommand{\IC}{\mathbb{C}}
\newcommand{\IZ}{\mathbb{Z}}
\newcommand{\IQ}{\mathbb{Q}}
\newcommand{\IN}{\mathbb{N}}
\newcommand{\IT}{\mathbb{T}}

% Boundary Integral Operators
\newcommand{\OV}{\mathsf{V}}
\newcommand{\OK}{\mathsf{K}}
\newcommand{\OW}{\mathsf{W}}
% 

%
%\DeclareMathOperator*{\argmax}{arg\,max}
%\DeclareMathOperator*{\argmin}{arg\,min}

\newcommand{\OA}{\mathsf{A}}
\newcommand{\Id}{\mathsf{Id}}
\newcommand{{\D}}{\normalfont{\text{D}}}
\newcommand{{\G}}{\normalfont{\text{G}}}
\newcommand{{\U}}{\normalfont{\text{U}}}
\newcommand{{\I}}{\normalfont{\text{I}}}
\newcommand{{\per}}{\normalfont{\text{per}}}
\newcommand{{\y}}{{\boldsymbol{{y}}}}
\newcommand{\Of}{\mathsf{f}}

\newcommand{{\bc}}{{\bf c}}
\newcommand{{\bx}}{{\bf x}}
\newcommand{{\by}}{{\bf y}}
\newcommand{{\bz}}{{\bf z}}
\newcommand{{\bd}}{{\bf d}}
\renewcommand{\d}{\!\!\operatorname{d}}

% Traces
\newcommand{\traced}{\gamma_{\text{D}}}
\newcommand{\tracen}{\gamma_{\text{N}}}
\newcommand{\dual}[2]{\left \langle #1,#2 \right \rangle}
\newcommand{\dotp}[2]{\left ( #1,#2 \right )}

% Matrix trace
\newcommand{\tr}[1]{\text{tr}\left ( #1 \right )}

\newtheorem{theorem}{Theorem}[section]
\newtheorem{corollary}[theorem]{Corollary}
\newtheorem{lemma}[theorem]{Lemma}
\newtheorem{prob}[theorem]{Problem}
\newtheorem{definition}[theorem]{Definition}
\newtheorem{assumption}[theorem]{Assumption}
\newtheorem{proposition}[theorem]{Proposition}
\newtheorem{example}[theorem]{Example}
\newtheorem{problem}[theorem]{Problem}

\theoremstyle{remark}
\newtheorem{remark}{Remark}
%\newtheorem{assumption}[remark]{Assumption}
%\newtheorem{example}[remark]{Example}
%\newtheorem{problem}[thm]{Problem}


%\numberwithin{equation}{section}

\begin{document}
%\begin{talk}
%
%---------
% TITLE
%---------
%

\title{Exercise Set 3\\ 
MATH-459--Numerical Methods for Conservation Laws \\  
Autumn Semester 2021
}

\author{Professor: Martin W. Licht \\ Assistant: Fernando Henriquez B. \\ Published on 6.10.21} 

\maketitle



\subsection*{1. Non-conservative scheme}
${}$\\

\noindent Consider the non-conservative scheme for Burgers' equation:
\begin{equation}
	U^{n+1}_j = 	U^{n}_j +  	2 \frac{k}{h} U^{n}_j \left( 	U^{n}_j - U^{n}_{j-1} \right).
\end{equation}
Show that this scheme maps non-negative monotonely increasing sequences onto non-negative monotonely increasing sequences. 

\textit{Remark: while this scheme is not conservative, is still of interest in a few contexts. The non-negativity condition is due to the upwinding implicit in the scheme.}
${}$\\



\subsection*{2. Monotonicity of FTCS}
${}$\\


\noindent What is the flux of the FTCS scheme for the transport equation? Relate it to other fluxes that you know. Determine whether the FTCS scheme is monotone.
${}$\\




\subsection*{3. Order of schemes}
${}$\\


\noindent Show that the FTFS scheme for the transport equation is of order $(1,1)$ and that the FTCS scheme is of order $(2,1)$.
${}$\\




\subsection*{4. Order of schemes}
${}$\\


\noindent Under what conditions is a linear scheme $F(U,V) = w_1 U + w_2 V$ a monotone scheme?
Describe the discrete Kruzkov entropy entropy-flux pairs.
${}$\\




\subsection*{5. Roe flux}
${}$\\

\noindent Show that the Roe flux can be written as 
\begin{equation}
	F(U,V) = \frac{ f(V) + f(U) }{2} + \frac{1}{2} \operatorname{sgn}( V - U ) |f(V) - f(U)|.
\end{equation}
Either show that the Roe flux is monotonicity-preserving or give a counter example. 
${}$\\




% \end{exerciseList}

\end{document}


\Head{6}{Godunov's Method}







\item
Discuss qualitatively the derivation of \emph{Godunov's method}; sketch each step in the solution process. Which part of the algorithm can make its implementation particularly difficult?

\item
Consider the scalar conservation law
\begin{gather}%
	u_t+f(u)_x=0\ ,
\end{gather}%
and initial condition
\begin{gather}%
	u(x,0)=\begin{cases}
		u_l & x<0\\
		u_r & 0<x
	\end{cases}\ ,
\end{gather}%
where the flux $f$ is convex ($f''>0$).
Godunov's method relies on finding the intermediate state $u^*=u^*(u_l,u_r)$ for which $u(0,t)=u^*$, for $t>0$.
\begin{enumerate}
\item
Show that this intermediate state is given by the following:
\begin{align*}
	&1.\ \ f'(u_l), f'(u_r)\ge 0 \ \ \Longrightarrow \ \ u^*=u_l \\
	&2.\ \ f'(u_l), f'(u_r)\le 0 \ \ \Longrightarrow \ \ u^*=u_r \\[0.5em]
	&3.\ \ f'(u_l) \ge 0\ge f'(u_r) \ \ \Longrightarrow \ \
		u^*=\begin{cases}
			u_l & s>0 \\
			u_r & s<0
		\end{cases}\ ,
		\qquad s=\frac{f(u_r)-f(u_r)}{u_r-u_l}\\[0.5em]
	&4.\ \ f'(u_l) < 0 < f'(u_r)  \ \ \Longrightarrow \ \ u^*=u_m\ ,
		\qquad\text{where $u_m$ is the solution to $f'\frb{u_m}=0$.}
\end{align*}

\item
Use (a) to show that Godunov's flux is given by
\begin{gather} \label{GodFlx}
	F(u_l,u_r)=\begin{cases}
			\displaystyle
			\min_{u_l\le u\le u_r} f(u) & u_l\le u_r \\[1em]
			\displaystyle
			\max_{u_r\le u\le u_l} f(u) & u_l> u_r 
		\end{cases}\ .
\end{gather}%
\item
Show that Godunov's flux \eqref{GodFlx} is monotone.
\end{enumerate}



\item
The purpose of this exercise is to illustrate the \emph{Lax-Wendroff Theorem}.
This theorem states that if there exists a sequence $\{(h_l,k_l)\}_l^\infty$ with $k_l = \lambda h_l$ ($\lambda$ is kept constant), such that the corresponding numerical solutions $\left\{ v_{l}\right\} _{l=1}^{\infty}$ obtained by a conservative method converges to some function $u$, then the limit $u$ is a weak solution of the conservation law. Notice that to deduce the conclusion, we assume that $v_{l}$ converges as $l\rightarrow\infty$.
That is, convergence is not a conclusion of the Lax-Wendroff theorem.
Also recall that in general weak solutions are not unique, so the theorem does not guarantee the limit is the correct entropy solution. 

Consider a conservative method 
\begin{gather}%
	v_{j}^{n+1}=v_{j}^{n} -\frac{k}{h}\sqb{F\frb{v_{j}^{n},v_{j+1}^{n}}-F\frb{v_{j-1}^{n},v_{j}^{n}}}
\end{gather}%
where the numerical flux F is given by 
\begin{gather}%
	F\frb{v,w}=\begin{cases}
		f(v) & \frac{f(v)-f\frb{w}}{v-w}\geq0\\[0.5em]
		f\frb{w} & \frac{f(v)-f\frb{w}}{v-w}<0
	\end{cases}\ .
\end{gather}%
\begin{enumerate}
\item
Construct the entropy solution to the following initial value problem
\begin{gather} \label{IVP}
	u_{t}+\rb{\frac{1}{2}u^{2}}_{x}=0
	\quad
		u(x,0)=\begin{cases}
		-1 & x<1\\
		1 & x>1
	\end{cases}\ .
\end{gather}%
%satisfying the initial condition
%\begin{gather} \label{incond2}
%	u(x,0)=\begin{cases}
%		-1 & x<1\\
%		1 & x>1
%	\end{cases}\ .
%\end{gather}%
\item
Fix $k/h=0.5$, and implement the above method to \eqref{IVP}, in $x\in(0,2)$, $0<t\le0.25$, with the initial data 
discretized using cell averages. On the boundaries, set $u(0,t)=-1$, and $u\frb{2,t}=1$.

\item Run the computations by choosing i) $h_l = \frac{2}{l}$, ii)$h_l = \frac{2}{2 l}$, iii) $h_l = \frac{2}{2 l + 1}$, for $l\in\mathbb{N}$.

\item
What can you deduce from your results regarding the each of the three sequences of numerical solutions obtained.
Explain your results and conclude how they fit with the Lax-Wendroff theorem.
\end{enumerate}


