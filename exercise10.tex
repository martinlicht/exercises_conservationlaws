\documentclass[10pt,letterpaper]{article}
\usepackage[english]{babel}
\usepackage{graphicx}
\usepackage[margin=2cm]{geometry}

\usepackage{float}
\usepackage{subfloat}
\usepackage{caption}
\usepackage{subcaption}
\usepackage{listings}

\usepackage{mathtools}
\usepackage{amsfonts}
\usepackage{amsmath}
\usepackage{amssymb}
\usepackage{amsthm}
\newcommand{\dif}[1][]{\mathrm{d} {#1}\,}
\newcommand{\rb}[1]{ \left(  {#1} \right) }
\newcommand{\frb}[1]{ \left(  {#1} \right) }
\newcommand{\norm}[1]{ \left\|  {#1} \right\| }
\newcommand{\jph}{{j+\frac{1}{2}}}
\newcommand{\jmh}{{j-\frac{1}{2}}}
\newcommand{\sqb}[1]{ \left(  {#1} \right) }
\newcommand{\mat}[1]{ \begin{array}{cc}  {#1} \end{array} }
\newcommand{\myvector}[1]{ \begin{array}{c}  {#1} \end{array} }
\newcommand{\iph}{{i + \frac{1}{2}}}
\newcommand{\imh}{{i - \frac{1}{2}}}

% \usepackage[amsmath,thmmarks,standard]{ntheorem}
\newtheoremstyle{break}
{\topsep}
{\topsep}%
{\normalfont}
{}%
{\bfseries}
{}%
{\newline}
{}%
\theoremstyle{break}
\newtheorem{exercise}{Exercise}
\newtheorem*{information}{Information}
\newtheorem{mysolution}{Solution}
% \newenvironment{solution}{\begin{comment}}{\end{comment}}
\newtheorem*{solutioninformation}{Solution Information}

\usepackage{comment}
% Switch between showing and hiding solutions by commenting out either of the following lines
\newenvironment{solution}{\begin{mysolution}}{\end{mysolution}}
% \excludecomment{solution}






\begin{document}

\title{High-order schemes and stencil selection}
\date{}

\maketitle


\begin{exercise}
	Consider the linear advection equation
	\begin{gather} \label{linadv}
		u_t + a u_x = 0\ .
	\end{gather}%
	We can construct a high-order scheme for \eqref{linadv} by suitably reconstructing the interface values (which is used to evaluate the numerical flux),
	followed by a Runge-Kutta integration in time.
	In each cell $i$, a quadratic polynomial can be obtained using the cell-average values,
	by choosing one of the following stencils
	\[
		S_r = \{x_{i-r}\ , \ x_{i+1-r}\ , \ x_{i+2-r} \}, \quad r = 0,1,2
	\]
	where $r$ represents the number of cells to the left of cell $i$ in the stencil $S_r$. For a fixed $r$, the left and right interface values can be expressed as
	\begin{gather} \label{rec}
	u_\iph^- = \sum \limits_{j=0}^2 c_{rj} \overline{u}_{i-r+j}\ , \qquad  u_\imh^+ = \sum \limits_{j=0}^2 \tilde{c}_{rj} \overline{u}_{i-r+j}
	\end{gather}%
	each of which are third-order accurate approximations. 


	\begin{enumerate}
		\item 
		Find the coefficients ${c}_{rj}$ and $\tilde{c}_{rj}$ for $r,j = 0,1,2$. (You could use two methods to obtain these: i) differentiating the interpolating polynomial for the primitive of $u$, or ii) directly using Taylor series expansions and trying to satisfy order constraints.)

		\item Write a finite volume code for \eqref{linadv}, where the interface values can be obtained by using either of the three stencils. Plug these values in the Godunov flux, and integrate the semi-discrete scheme using SSP-RK3. Implement the following initial conditions on the domain $[-1,1]$
		\begin{gather} \label{inData1}
			u_0(x)=\sin(\pi x)\ ,  
			\quad
			T_f = 5, \quad \text{with periodic BC}
		\end{gather}%
		\begin{gather} \label{inData2}
			u_0(x)=\begin{cases}
				1 & x<0\\
				-1 & x>0
			\end{cases}\ ,
			\quad
			T_f = 0.5, \quad \text{with open BC}.
		\end{gather}%
		Use a CFL of 0.2 to evaluate the time-step. 

		\item
		Run the code for $r=0,1,2$ and $a=1$. What do you observe with each type of stencil? Do you recover 3rd-order convergence for \eqref{inData1}? How do the results change if you choose $a=-1$ instead?
	\end{enumerate}
\end{exercise}


\begin{solution}
	\begin{enumerate}

		\item The interface values are given by
		\begin{gather} \label{rec}
		u_\iph^- = \sum \limits_{j=0}^2 c_{r,j} \overline{u}_{i-r+j}\ , \qquad  u_\imh^+ = \sum \limits_{j=0}^2 \tilde{c}_{r,j} \overline{u}_{i-r+j}
		\end{gather}
		where $\tilde{c}_{r,j} = c_{r-1,j}$. The coefficients are given in Table \ref{tab:crj}. Note that we also define the values for r=-1, since it is needed to define $\tilde{c}_{r,j}$ when r=0.
		
		\begin{table}[htbp]
		\centering
		\begin{tabular}{|c|c|c|c|}
		\hline
		\multicolumn{1}{|l|}{\textbf{\begin{tabular}[c]{@{}c@{}}   \\ \hspace{0.5cm} j\\      r\end{tabular}}} & \multicolumn{1}{c|}{\textbf{0}} & \multicolumn{1}{c|}{\textbf{1}} & \multicolumn{1}{c|}{\textbf{2}} \\ \hline
		\textbf{-1}                                                                       & $\frac{11}{6}$                  & $-\frac{7}{6}$                  & $\frac{1}{3}$                   \\ \hline
		\textbf{0}                                                                        & $\frac{1}{3}$                   & $\frac{5}{6}$                   & $-\frac{1}{6}$                  \\ \hline
		\textbf{1}                                                                        & $-\frac{1}{6}$                  & $\frac{5}{6}$                   & $\frac{1}{3}$                   \\ \hline
		\textbf{2}                                                                        & $\frac{1}{3}$                   & $-\frac{7}{6}$                  & $\frac{11}{6}$                  \\ \hline
		\end{tabular}
		\caption{Table of coefficients.}
		\label{tab:crj}
		\end{table}
		
		\item
		See the Matlab code attached at the end of this solution manual.
		
		
		\item
		See figures generated by the Matlab code.
		
		The solutions evaluated with r=0,2 are not stable, and blow up quickly with mesh refinement. However, the solutions are stable with r=1, which corresponds to choosing the central stencil for each cell. Third-order accuracy is observed for the smooth initial condition with r=1, which drops to below first-order for the discontinuous data. Furthermore, small Gibbs oscillations appear near the discontinuity.
		
		The results from this exercise indicate that simply using a high-order reconstruction may not ensure stability. One can obtain stable solutions by adaptively choosing the stencil for reconstruction in each cell. This is the strategy used in the so called \textit{essentially non-oscillatory} (ENO) reconstruction method.
		
		
	\end{enumerate}
		
		
% 	\lstinputlisting{./10Code/Exercise10.m}		TODO: Find the code 
% 	\lstinputlisting{./10Code/solver.m}			TODO: Find the code 
% 	\lstinputlisting{./10Code/apply_bc.m}		TODO: Find the code 
% 	\lstinputlisting{./10Code/evalRHS.m}		TODO: Find the code 
% 	\lstinputlisting{./10Code/rec3.m}			TODO: Find the code 
% 	\lstinputlisting{./10Code/find_exact.m}		TODO: Find the code 
% 	\lstinputlisting{./10Code/find_err.m}		TODO: Find the code 
\end{solution}








\end{document}




