%Typeset with LaTeX
%\documentclass{owrart}
\documentclass[10pt,reqno]{amsart}
\usepackage[a4paper,margin=1.0in]{geometry}
\UseRawInputEncoding
%% -------------------------------------------------------------------------------
%% Enter additionally required packages below this comment.
\usepackage{algorithm}
\usepackage{algpseudocode}

\makeatletter
% Reinsert missing \algbackskip
\def\algbackskip{\hskip-\ALG@thistlm}
\makeatother

\usepackage{amsmath}
\usepackage{mathtools}
%\usepackage[a4paper]{geometry}
%\usepackage{amsfonts}
\usepackage{amssymb}
\usepackage{amsthm}
\usepackage{mathrsfs}
\usepackage{bm}
\usepackage{algpseudocode}
\usepackage{upgreek}
\usepackage{subcaption}
% \usepackage{showkeys} %%%%%%%%%
% \usepackage{amscd}
% \usepackage{latexsym}
% \usepackage{nicefrac} 3
% \usepackage{cite}
 %\usepackage{epic}
% \usepackage{eepic}
% \usepackage{fancybox}
\newcommand{\bsnul}{{\boldsymbol 0}}
\newcommand{\cNN}{{\mathcal {NN}}}

\usepackage{graphicx}
\usepackage{epstopdf}

\mathtoolsset{showonlyrefs}

%\usepackage{epstopdf}
%\usepackage{graphicx} 

%\usepackage[pdf]{pstricks}

\usepackage[shortlabels]{enumitem}
%Standard operator
\newcommand{\abs}[1]{\lvert #1\rvert}
\newcommand{\supp}{Peratorname{supp}}
\newcommand{\grad}{{\mathbf{grad}}}
\newcommand{\curl}{\normalfont{\text{curl}}}
\newcommand{\Log}{\text{Log}}
\newcommand{\cc}{{\normalfont{\text{c}}}}
\newcommand{\loc}{{\normalfont{\text{loc}}}}

\DeclareMathOperator*{\argmax}{arg\,max}
\DeclareMathOperator*{\argmin}{arg\,min}

% Package for colors
\usepackage{xcolor}
\definecolor{ao}{rgb}{0.0, 0.5, 0.0}

%Edit colors
\newcommand{\CS}[1]{{\color{magenta}{#1}}}
\newcommand{\ra}[1]{{\color{blue}{#1}}}
\newcommand{\fh}[1]{{\color{cyan}{#1}}}
\newcommand{\todo}[1]{{\color{red}[#1]}}

\newcommand{\com}[1]{{\color{red} [#1]}}

%\usepackage{fancyhdr}
%\usepackage{datetime}
%
%\fancyhf{}
%\fancyfoot[L]{\fontsize{10}{12} \today \ \currenttime}
%\pagestyle{fancy} \usepackage{tikz}

\DeclareSymbolFont{extraup}{U}{zavm}{m}{n}
\DeclareMathSymbol{\varheart}{\mathalpha}{extraup}{86}
\DeclareMathSymbol{\vardiamond}{\mathalpha}{extraup}{87}

\usepackage{chngcntr}
\counterwithout{equation}{subsection}
\counterwithout{equation}{section} 

\usepackage{tikz}

\newcommand\encircle[1]{%
  \tikz[baseline=(X.base)] 
    \node (X) [draw, shape=circle, inner sep=0] {\strut #1};}

\usepackage{pifont}

\usepackage{fancyhdr}
\usepackage[us,12hr]{datetime} % `us' makes \today behave as usual in TeX/LaTeX
\fancypagestyle{plain}{
\fancyhf{}
\rhead{\footnotesize Version of \today}
\lhead{\thepage}
\renewcommand{\headrulewidth}{0pt}}
\pagestyle{plain}

\usepackage{color}
\newcommand{\half}{\frac{1}{2}}

\newcommand{\norm}[1]{\left \lVert #1 \right \rVert}
\newcommand{\snorm}[1]{\left \lvert #1 \right \rvert}

%fields, mathbb operators
\newcommand{\IR}{\mathbb{R}}
\newcommand{\IC}{\mathbb{C}}
\newcommand{\IZ}{\mathbb{Z}}
\newcommand{\IQ}{\mathbb{Q}}
\newcommand{\IN}{\mathbb{N}}
\newcommand{\IT}{\mathbb{T}}

% Boundary Integral Operators
\newcommand{\OV}{\mathsf{V}}
\newcommand{\OK}{\mathsf{K}}
\newcommand{\OW}{\mathsf{W}}
% 

%
%\DeclareMathOperator*{\argmax}{arg\,max}
%\DeclareMathOperator*{\argmin}{arg\,min}

\newcommand{\OA}{\mathsf{A}}
\newcommand{\Id}{\mathsf{Id}}
\newcommand{{\D}}{\normalfont{\text{D}}}
\newcommand{{\G}}{\normalfont{\text{G}}}
\newcommand{{\U}}{\normalfont{\text{U}}}
\newcommand{{\I}}{\normalfont{\text{I}}}
\newcommand{{\per}}{\normalfont{\text{per}}}
\newcommand{{\y}}{{\boldsymbol{{y}}}}
\newcommand{\Of}{\mathsf{f}}

\newcommand{{\bc}}{{\bf c}}
\newcommand{{\bx}}{{\bf x}}
\newcommand{{\by}}{{\bf y}}
\newcommand{{\bz}}{{\bf z}}
\newcommand{{\bd}}{{\bf d}}
\renewcommand{\d}{\!\!\operatorname{d}}

% Traces
\newcommand{\traced}{\gamma_{\text{D}}}
\newcommand{\tracen}{\gamma_{\text{N}}}
\newcommand{\dual}[2]{\left \langle #1,#2 \right \rangle}
\newcommand{\dotp}[2]{\left ( #1,#2 \right )}

% Matrix trace
\newcommand{\tr}[1]{\text{tr}\left ( #1 \right )}

\newtheorem{theorem}{Theorem}[section]
\newtheorem{corollary}[theorem]{Corollary}
\newtheorem{lemma}[theorem]{Lemma}
\newtheorem{prob}[theorem]{Problem}
\newtheorem{definition}[theorem]{Definition}
\newtheorem{assumption}[theorem]{Assumption}
\newtheorem{proposition}[theorem]{Proposition}
\newtheorem{example}[theorem]{Example}
\newtheorem{problem}[theorem]{Problem}

\theoremstyle{remark}
\newtheorem{remark}{Remark}
%\newtheorem{assumption}[remark]{Assumption}
%\newtheorem{example}[remark]{Example}
%\newtheorem{problem}[thm]{Problem}


%\numberwithin{equation}{section}

\begin{document}
%\begin{talk}
%
%---------
% TITLE
%---------
%

\title{Exercise Set 4\\ 
MATH-459--Numerical Methods for Conservation Laws \\  
Autumn Semester 2021
}

\author{Professor: Martin W. Licht \\ Assistant: Fernando Henriquez B. \\ Published on 14.10.21} 

\maketitle

Throughout this Exercise Set, consider the linear PDE
\begin{equation}\label{eq:advection_eq}
	u_t+au_x=0
\end{equation}

\subsection*{Problem 1. Finite Differences Method}
Define the following concepts:
\begin{itemize}
\item[(a)] Consistency;
\item[(b)] Stability; and,
\item[(c)] Convergence.
\end{itemize}
Discuss the importance of these concepts, how they relate to each other, and how in general they can be established.

\subsection*{Problem 2. Leapfrog Method}
The forward time, center space finite difference
approximation to the advection equation \eqref{eq:advection_eq}
leads to a method which is \emph{unstable}.
So let us try something different.
By approximating the time derivative by a centered difference,
instead of the forward Euler discretization, we get the Leapfrog method,
\begin{equation}\label{eq:LFrog}
	v_{j}^{n+1}=v_{j}^{n-1}-\frac{ak}{h}(v_{j+1}^{n}-v_{j-1}^{n})\ .
\end{equation}
\begin{itemize}
\item[(i)]
Draw the stencil of \eqref{eq:LFrog}.

\item[(ii)]
Show that \eqref{eq:LFrog} is second order accurate in both space and time.

\item[(iii)]
What is an obvious disadvantage of the Leapfrog method compared to the Lax-Friedrichs or Lax-Wendroff methods?
\end{itemize}


\subsection*{Problem 3. Stability of the Lax-Friedrichs Method}
Consider the Lax-Friedrichs method
$$
	v_{j}^{n+1}=\frac{1}{2}(v_{j+1}^{n}+v_{j-1}^{n})-\frac{ak}{2h}(v_{j+1}^{n}-v_{j-1}^{n})
$$
Show that this method is stable in the $l^\infty$ norm, provided that $k$ and $h$ satisfy the \textit{CFL condition}
$$
	\frac{|a|k}{h}\leq1.
$$
%
\subsection*{Problem 4. Unconditionally Stable Method}
We have seen in the previous exercise that the Lax-Friedrichs
method is stable provided $k$ and $h$ satisfy a CFL condition.
A method which is stable for any $k$ and $h$ is said to be
\textit{unconditionally stable}.
For $a>0$, prove that the following backward-time backward-space method
$$
	v_{j}^{n+1}=v_j^n-\frac{ak}{h}(v_{j}^{n+1}-v_{j-1}^{n+1})
$$
is unconditionally stable in the $l^\infty$ norm.

%\item
%\begin{enumerate}
%\item
%Explain qualitatively what the CFL condition is.
%
%\item
%Is the CFL condition sufficient for stability?
%If not, can you think of a counter example?
%\end{enumerate}

\subsection*{Problem 5. Matlab Implementation}
This exercise involves some programming.
Consider the scalar advection equation \eqref{eq:advection_eq}
with $a=1$.
Let $u$ be the solution of \eqref{eq:advection_eq}
in $(-1,1)$ that satisfies initial condition
$$
	u(x,0)=u_{0}(x)\ ,
$$
where
\begin{equation}\label{eq:init_cond}
	u_{0}(x)
	=
	\begin{cases}
			1 & x<0\\
			0 & x>0
	\end{cases}\ ,
\end{equation}
and boundary conditions
\begin{equation}\label{eq:bc_cond}
	u(-1,t)=1
	\quad\quad
	u(1,t)=0
\end{equation}
We already know that the exact solution for $0<t<1$ is given by $u_{0}(x-at)$,
but how well do numerical methods approximate such a problem?
In the following consider the schemes
\begin{align*}
	\textrm{Upwind:}\quad
		v_{j}^{n+1} & =v_{j}^{n}-\frac{ak}{h}(v_{j}^{n}-v_{j-1}^{n})\\
	\textrm{Lax-Friedrichs:}\quad
		v_{j}^{n+1} & =\frac{1}{2}(v_{j+1}^{n}+v_{j-1}^{n})
			-\frac{ak}{2h} (v_{j+1}^{n}-v_{j-1}^{n})\\
	\textrm{Lax-Wendroff:}\quad
		v_{j}^{n+1} & =v_{j}^{n}-\frac{ak}{2h}(v_{j+1}^{n}-v_{j-1}^{n})
			+\frac{(ak)^2}{2h^2}(v_{j+1}^{n}-2v_{j}^{n}+v_{j-1}^{n})\\
	\textrm{Beam-Warming:}\quad
		v_{j}^{n+1} & =v_{j}^{n}-\frac{ak}{2h}(3v_{j}^{n} -4v_{j-1}^{n}+v_{j-2}^{n})
			+\frac{(ak)^{2}}{2h^{2}} (v_{j}^{n}-2v_{j-1}^{n}+v_{j-2}^{n})\ .
\end{align*}

For each of the schemes above:
\begin{enumerate}
\item
Implement the scheme in Matlab to solve \eqref{eq:advection_eq}, 
in $(-1,1)$, with \eqref{eq:init_cond}, and \eqref{eq:bc_cond} in the time interval $t\in[0,0.5]$. 
For your computations use $h=0.0025$ and $k/h=0.5$.

\item
Visualize the numerical solution and the exact solution.

\item
Comment qualitatively on the solutions' behavior.

\item
Write a script to generate a log-log plot of the error as a function of the resolution 
(either $h$, or the number of points in the spatial grid), 
while keeping the ratio $k/h=0.5$ fixed (why is this important?).

\item
Use the plot to deduce the accuracy of the method.

\item
Compare your results with the accuracy you would expect on a smooth solution.
\end{enumerate}
%\bibliography{ref}{}
%\bibliographystyle{plain}
\end{document}
