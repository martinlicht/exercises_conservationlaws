\documentclass{article}



\usepackage{../auxFiles/ExStyPac}



\input ../auxFiles/mac.tex
\graphicspath{{./figures/}}





%========================================================================
\begin{document}



\Head{5}{Conservative Methods}





\begin{exerciseList}


\item
Answer the following questions:
\begin{enumerate}
\item
When is a numerical method said to be conservative?

\item
What is the benefit of using a conservative numerical method? (Hint: Take a look at the Lax-Wendroff Theorem)

\item
For a given conservation law and a conservative scheme, are we guaranteed that the weak solution obtained satisfies the entropy condition?
\end{enumerate}

\item
Consider the conservative scheme with the \textit{Engquist-Osher} flux
\begin{gather}
F^{EO}(u,v) = \frac{1}{2} \rb{ f(u) + f(v) - \int \limits_u^v |f^\prime(\xi)|  d\xi}.
\end{gather}
\begin{enumerate}
\item Show that the numerical flux leads to a monotone scheme under suitable CFL conditions.

\item Assuming $f$ is convex with a single minima at $\omega$, show that the Engquist-Osher flux reduces to
\begin{gather}
F^{EO}(u,v) = f(\max(u,\omega)) + f(\min(v,\omega)) - f(\omega).
\end{gather}
\end{enumerate}


\item
In the previous exercises, we have seen that difficulties can arise when trying to approximate solution for linear problem. Additional issues can arise when dealing with non-linear problems. Consider the Burgers equation in the quasilinear form
\begin{gather} \label{quasiLinBurg}
	u_{t}+uu_{x}=0\ .
\end{gather}
A ``natural'' finite difference method can be obtained with a minor modification of the upwind method applied to the advection equation, assuming $v_{j}^{n}\geq0$ for all $j,n$:
\begin{gather} \label{quasiLinUpWind}
	v_{j}^{n+1}=v_{j}^{n}-\frac{k}{h}v_{j}^{n}\left(v_{j}^{n} -v_{j-1}^{n}\right).
\end{gather}
This method converges on smooth solutions.
\begin{enumerate}
\item
Compute the numerical solution obtained by \eqref{quasiLinUpWind}, driven by the initial condition
\begin{gather} \label{inCond1}
	u(x,0)=\begin{cases}
			1\; & x<0\\
			0\; & x\geq0
		\end{cases}\ .
\end{gather}
Implement the method in Matlab and solve \eqref{quasiLinBurg} up to $T=0.5$ in the interval $(-1,1)$ with initial condition \eqref{inCond1}. In your computations use $k=0.5h$, $h=0.01$.


\item
Is the solution obtained a weak solution? Is it the entropy
solution?

\item
Now use the following initial condition in your code 
\begin{gather} \label{inCond2}
	u\left(x,0\right)=\begin{cases}
			1.2\; & x<0\\
			0.4\; & x\geq0
		\end{cases}\ .
\end{gather}

\item
Compare the solution to the known entropy solution for Burgers, which can be constructed by considering the characteristics and shocks.
\end{enumerate}


\item
\begin{enumerate}
\item
Apply the generalization of the Lax-Friedrichs method
\begin{gather}
	v_{j}^{n+1}=\frac{1}{2}\rb{v_{j+1}^{n}+v_{j-1}^{n}}
		-\frac{k}{2h}\rb{f\frb{v_{j+1}^{n}}-f\frb{v_{j-1}^{n}}}
\end{gather}
to Burgers equation in conservation form,
\begin{gather}
	u_t+\rb{\frac{1}{2}\, u^2}_x=0\ ,
\end{gather}
with the initial conditions \eqref{inCond1} and \eqref{inCond2}.

\item
Does the numerical solutions converge to a weak solution?

\item
Is this the entropy solution?

\item
Show that the generalization of the Lax-Friedrichs method to nonlinear conservation laws can be written in conservative form.
\end{enumerate}

\item
\begin{enumerate}
\item
Solve the Burgers equation with the Engquist-Osher flux and the initial conditions \eqref{inCond1} and \eqref{inCond2}.

\item
Does the solution converge to the entropy solution?

\item
How does the solution compare to that obtained with the Lax-Friedrichs scheme?

\end{enumerate}



\end{exerciseList}

\end{document}
