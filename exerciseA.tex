\documentclass[10pt,letterpaper]{article}
\usepackage[english]{babel}
\usepackage{graphicx}
\usepackage[margin=2cm]{geometry}

\usepackage{float}
\usepackage{subfloat}
\usepackage{caption}
\usepackage{subcaption}
\usepackage{listings}

\usepackage{mathtools}
\usepackage{amsfonts}
\usepackage{amsmath}
\usepackage{amssymb}
\usepackage{amsthm}
\newcommand{\dif}[1][]{\mathrm{d} {#1}\,}
\newcommand{\rb}[1]{ \left(  {#1} \right) }
\newcommand{\frb}[1]{ \left(  {#1} \right) }
\newcommand{\norm}[1]{ \left\|  {#1} \right\| }
\newcommand{\jph}{{j+\frac{1}{2}}}
\newcommand{\jmh}{{j-\frac{1}{2}}}
\newcommand{\sqb}[1]{ \left(  {#1} \right) }
\newcommand{\mat}[1]{ \begin{array}{cc}  {#1} \end{array} }
\newcommand{\myvector}[1]{ \begin{array}{c}  {#1} \end{array} }
\newcommand{\iph}{{i + \frac{1}{2}}}
\newcommand{\imh}{{i - \frac{1}{2}}}
\newcommand{\cR}{R}
\newcommand{\p}{p}
\newcommand{\minmod}{\operatorname{minmod}}
\newcommand{\sign}{\operatorname{sign}}


% \usepackage[amsmath,thmmarks,standard]{ntheorem}
\newtheoremstyle{break}
{\topsep}
{\topsep}%
{\normalfont}
{}%
{\bfseries}
{}%
{\newline}
{}%
\theoremstyle{break}
\newtheorem{exercise}{Exercise}
\newtheorem*{information}{Information}
\newtheorem{solution}{Solution}
\newtheorem*{solutioninformation}{Solution Information}

\begin{document}

\title{Higher-Order Schemes}
\date{}

\maketitle











\begin{exercise}[Minmod]
    We have introduced the minmod function
    \begin{align}
    \minmod(a,b)
        = 
        \begin{cases}
            \sign(a) \min(|a|,|b|) & \text{ $a$ and $b$ have the same sign.}
            \\
            0                      & \text{ $a$ and $b$ have different signs.}
    \end{cases}
    \end{align}
    Consider three values $U, V, W \in \mathbb R$ and set 
    \begin{align}
        P = \minmod\left( \frac{U-V}{h}, \frac{V-W}{h} \right)
        .
    \end{align}
    Show that 
    \begin{align}
     U + \frac h 2 P = \left[ \min(V,W), \max(V,W) \right]
     ,
     \quad 
     U - \frac h 2 P = \left[ \min(U,V), \max(U,V) \right]
     .
    \end{align}
\end{exercise}

\begin{solution}
    \begin{enumerate}
        \item 
        In the case one of $\frac{U-V}{h}$ or $\frac{V-W}{h}$ is zero, $P = 0$ follows. 
        \item 
        Consider the case that $\frac{U-V}{h}$ and $\frac{V-W}{h}$ have different signs. 
        Then $V$ is either the smallest or the largest of the two,
        and hence $P=0$ must follow.
        \item 
        Consider the case that $\frac{U-V}{h}$ and $\frac{V-W}{h}$ have positive sign.
        It follows that $U < V < W$. We observe 
        \begin{align}
            V 
            +
            \frac h 2 P
            &
            =
            V
            + 
            \min\left( \frac h 2 \frac{U-V}{h}, \frac h 2 \frac{V-W}{h} \right) 
            % \\&
            =
            V
            +
            \min\left( \frac{U-V}{2}, \frac{V-W}{2} \right) 
            ,
            \\
            V 
            -
            \frac h 2 P
            &
            =
            V - \min\left( \frac h 2 \frac{U-V}{h}, \frac h 2 \frac{V-W}{h} \right) 
            % \\&
            =
            V - \min\left( \frac{U-V}{2}, \frac{V-W}{2} \right) 
            .
        \end{align}
        
    \end{enumerate}
\end{solution}







\begin{exercise}
    Consider the scheme
    \begin{align}
     U^{n+1}_{j}
     &
     =
     U^{n  }_{j}
     -
     \dfrac{k}{2h}
     \left(
        f^{n}_{j+1} + f^{n}_{j-1}
     \right)
     +
     \dfrac{k^2}{2 h^2}
     \left(
        f'\left( \dfrac{ U^{n}_{j+1} + U^{n}_{j} }{2} \right)
        \left(
            f^{n}_{j+1} - f^{n}_{j}
        \right)
        -
        f'\left( \dfrac{ U^{n}_{j} + U^{n}_{j-1} }{2} \right)
        \left(
            f^{n}_{j} - f^{n}_{j-1}
        \right)
     \right)     
%      \\&
%      =
%      U^{n  }_{j}
%      -
%      \dfrac{k}{2h}
%      \left(
%         f^{n}_{j+1} + f^{n}_{j-1}
%      \right)
%      -
%      \dfrac{k^2}{2 h^2}
%      \left(
%         f'\left( \dfrac{ U^{n}_{j} + U^{n}_{j-1} }{2} \right)
%         \left(
%             f^{n}_{j} - f^{n}_{j-1}
%         \right)
%         -
%         f'\left( \dfrac{ U^{n}_{j+1} + U^{n}_{j} }{2} \right)
%         \left(
%             f^{n}_{j+1} - f^{n}_{j}
%         \right)
%      \right)     
%      \\&
%      =
%      U^{n  }_{j}
%      -
%      \dfrac{k}{2h}
%      \left(
%         f^{n}_{j+1} + f^{n}_{j-1}
%      \right)
%      -
%      \dfrac{k^2}{2 h^2}
%      \left(
%         f'\left( \dfrac{ U^{n}_{j} + U^{n}_{j-1} }{2} \right)
%         \left(
%             f^{n}_{j} - f^{n}_{j-1}
%         \right)
%         +
%         f'\left( \dfrac{ U^{n}_{j+1} + U^{n}_{j} }{2} \right)
%         \left(
%             f^{n}_{j} - f^{n}_{j+1}
%         \right)
%      \right)     
     ,
    \end{align}
    where we write $f^n_{j} := f\left( U^{n}_{j} \right)$. 
    \begin{enumerate}
        \item Show that this is a conservative scheme and determine its flux.
        \item Develop this scheme from a Taylor approximation and keep track of the error terms. Use the following recipe: 
        \begin{enumerate}
            \item Express $u(t^{n+1},x_{j})$ by a Taylor series around $u(t^{n},x_{j})$
            \item Replace time derivatives by derivatives of the flux
            \item Replace $x$-derivatives by central difference quotients with half step sizes (outer derivatives first) and use 
			\begin{align*}
				f'(u(t^{n},x_{j+\frac 1 2}))
				\approx 
				f'\left( \dfrac{ U^{n}_{j} + U^{n}_{j-1} }{2} \right)
			\end{align*}
            and approximate $f'$ using a midpoint formula. 
        \end{enumerate}
    \end{enumerate}

\end{exercise}

\begin{solution}
	\begin{enumerate}
		We pick the flux 
		\begin{align}
		 F(U,V)
		 = 
		 \left(
			f(V) + f(U)
		 \right)
		 +
		 \dfrac{k}{2 h}
		 f'\left( \dfrac{ V + U }{2} \right)
		 \left(
			 f(V) - f(U)
		 \right)
		 .
		\end{align}
		We calculate 
		{\tiny
		\begin{align} 
		 U^{n+1}_{j}
		 &
		 =
		 U^{n  }_{j}
		 -
		 \dfrac{k}{h}
		 \left( 
			F(U^{n}_{j},U^{n}_{j+1})
			-
			F(U^{n}_{j-1},U^{n}_{j})
		 \right)
		 \\&
		 =
		 U^{n  }_{j}
		 -
		 \dfrac{k}{h}
		 \left( 
			\left(
				f(U^{n}_{j+1}) + f(U^{n}_{j})
			\right)
			+
			\dfrac{k}{2 h}
			f'\left( \dfrac{ U^{n}_{j+1} + U^{n}_{j} }{2} \right)
			\left(
				f(U^{n}_{j+1}) - f(U^{n}_{j})
			\right)
			-
			\left(
				f(U^{n}_{j}) + f(U^{n}_{j-1})
			\right)
			-
			\dfrac{k}{2 h}
			f'\left( \dfrac{ U^{n}_{j} + U^{n}_{j-1} }{2} \right)
			\left(
				f(U^{n}_{j}) - f(U^{n}_{j-1})
			\right)
		 \right)
		 \end{align}
		}
		We simplify that to 
		\begin{align} 
			U^{n+1}_{j}
			&
			=
			U^{n  }_{j}
			-
			\dfrac{k}{h}
			\left(
				f(U^{n}_{j+1}) + f(U^{n}_{j-1})
			\right)
			\\&\qquad\qquad
			+
			\dfrac{k^2}{2 h^2}
			\left( 
				f'\left( \dfrac{ U^{n}_{j+1} + U^{n}_{j} }{2} \right)
				\left(
					f(U^{n}_{j+1}) - f(U^{n}_{j})
				\right)
				-
				f'\left( \dfrac{ U^{n}_{j} + U^{n}_{j-1} }{2} \right)
				\left(
					f(U^{n}_{j}) - f(U^{n}_{j-1})
				\right)   
			\right)
			.
		\end{align}
		For the error estimate, we begin with 
		\begin{align}
			u(t^{n+1},x_{j})
			= 
			u(t^{n  },x_{j})
			+
			k
			\partial_t u(t^{n},x_{j})
			+
			\dfrac{k^2}{2}
			\partial^{2}_{tt} u(t^{n},x_{j})
			+
			O(k^3)
			.
		\end{align}
		We replace 
		\begin{align}
			\partial_t u(t^{n},x_{j})
			=
			\partial_x f(u(t^{n},x_{j}))
		\end{align}
		and 
		\begin{align}
			\partial^{2}_{tt} u(t^{n},x_{j})
			&=
			\partial_t \partial_x f(u(t^{n},x_{j}))
			\\&=
			\partial_x \partial_t f(u(t^{n},x_{j}))
			\\&=
			\partial_x \left( f'(u(t^{n},x_{j})) \cdot \partial_t u(t^{n},x_{j}) \right)
			\\&=
			\partial_x \left( f'(u(t^{n},x_{j})) \cdot \partial_x f(u(t^{n},x_{j})) \right)
			.
		\end{align}
		Thus 
		\begin{align}
			u(t^{n+1},x_{j})
			= 
			u(t^{n  },x_{j})
			+
			k
			\partial_x f(u(t^{n},x_{j}))
			+
			\dfrac{k^2}{2}
			\partial_x \left( f'(u(t^{n},x_{j})) \cdot \partial_x f(u(t^{n},x_{j})) \right)
			+
			O(k^3)
			.
		\end{align}
		We now replace the $x$-derivatives by centralized difference quotients and use the midpoint rule for derivatives $f'$. 
		In that manner, 
		\begin{align}
			\partial_x f(u(t^{n},x_{j}))
			=
			\dfrac{ f(u(t^{n},x_{j+1})) - f(u(t^{n},x_{j-1})) }{2h}
			+
			O(h^2)
		\end{align}
		and 
		\begin{align}
			&
			\partial_x \left( f'(u(t^{n},x_{j})) \cdot \partial_x f(u(t^{n},x_{j})) \right)
			\\&\quad 
			=
			\dfrac{
				\left( f'(u(t^{n},x_{j+\frac 12})) \cdot \partial_x f(u(t^{n},x_{j+\frac 12})) \right)
				-
				\left( f'(u(t^{n},x_{j-\frac 12})) \cdot \partial_x f(u(t^{n},x_{j-\frac 12})) \right)
			}{h}
			+
			O(h^2)
			\\&\quad 
			=
			\dfrac{
				f'(u(t^{n},x_{j+\frac 12})) 
			}{h^2}
			\cdot \left( f(u(t^{n},x_{j+1})) - f(u(t^{n},x_{j  })) \right)
			-
			\dfrac{
				f'(u(t^{n},x_{j-\frac 12})) 
			}{h^2}
			\cdot \left( f(u(t^{n},x_{j  })) - f(u(t^{n},x_{j-1})) \right)
			+
			O(h)
			+
			O(h^2)
			.
		\end{align}
		We use the approximation 
		\begin{align}
			f
		\end{align}
	\end{enumerate}
\end{solution}








\end{document}
