\documentclass{article}



\usepackage{../auxFiles/ExStyPac}

\usepackage[framed]{../auxFiles/mcode}

\newcommand{\iph}{{i + \frac{1}{2}}}
\newcommand{\imh}{{i - \frac{1}{2}}}
\newcommand{\ipo}{{i + 1}}




\input ../auxFiles/mac.tex





%========================================================================
\begin{document}



\Head{8}{Incremental form and Harten's lemma (Solution)}





\begin{exerciseList}


\item
Consider the scheme
\begin{gather} \label{3pt_scheme}
u_i^{n+1} = u^n_i - \frac{k}{h} \left[ F_\iph - F_\imh\right].
\end{gather}
The RHS of \deq{3pt_scheme} can be re-written as

\begin{align*}
RHS &= u^n_i - \frac{k}{h} \left[ F_\iph - F_\imh\right]\\
        &= u^n_i - \frac{k}{h} \left[ F_\iph  - f(u_i) + f(u_i) - F_\imh\right]\\
        &= u^n_i - \frac{k}{h} \left[ \frac{F_\iph  - f(u_i)}{\Delta ^+u_i^n} \right] \Delta ^+u_i^n - \frac{k}{h}\left[\frac{f(u_i) - F_\imh}{\Delta ^-u_i^n}\right]\Delta ^-u_i^n\\
        &= u^n_i +  D_\iph \Delta ^+u_i^n - C_\imh \Delta ^-u_i^n
\end{align*}
since $\Delta ^-u_i^n = \Delta ^+u_{i-1}^n$. Thus, the scheme \deq{3pt_scheme} can be written in incremental form.

\item 

\begin{enumerate}
\item
Note that all four fluxes can be written in the form
\[
F(u,v) = \frac{1}{2} \left( f(u) + f(v) - Q(u,v) (v-u)\right)
\]
where
\begin{align*}
Q^{LF}(u,v) &= \frac{h}{k}\ ,\\
Q^{LLF}(u,v) &= \alpha = \max_u |f^\prime(u)|\ ,\\
Q^{LW}(u,v) &= \frac{k}{h} f^\prime\left( \frac{u+v}{2}\right)\left(\frac{f(v) - f(u)}{v-u}\right)\ ,\\
Q^{Roe}(u,v) &= \left|\frac{f(v) - f(u)}{v-u}\right|\ .
\end{align*}
Thus, the incremental coefficients are given by
\begin{align*}
C_\iph &= \frac{k}{2h} \left [Q(u_i,u_{i+1})+\left(\frac{f(u_{i+1}) - f(u_i)}{u_{i+1} - u_i}\right) \right]\ ,\\
D_\iph &= \frac{k}{2h} \left [Q(u_i,u_{i+1})-\left(\frac{f(u_{i+1}) - f(u_i)}{u_{i+1} - u_i}\right) \right] \ .
\end{align*}
Note that the three conditions of Harten's lemma translates to 
\begin{align}
\label{1n2n3}
\frac{h}{k} \geq Q(u_i,u_{i+1}) \geq \left|\frac{f(u_{i+1}) - f(u_i)}{u_{i+1} - u_i}\right| .
\end{align}
Thus, a three-point conservative flux satisfying \deq{1n2n3} will lead to a TVD scheme. Note that the condition \deq{1n2n3} also implies that $Q(u_i,u_{i+1})$ is strictly positive.
%Furthermore, we assume that the usual CFL condition holds
%\[
%\frac{k}{h} \max_u |f^\prime(u)| \leq 1.
%\]

\item
For all schemes, we assume that the usual CFL condition holds
\begin{gather} \label{cfl}
\frac{k}{h} \max_u |f^\prime(u)| \leq 1.
\end{gather}
Also note that using mean value theorem, we can find a $\xi_\iph$ such that
\begin{gather} \label{mvt}
\frac{f(u_{i+1}) - f(u_i)}{u_{i+1} - u_i} = f^\prime(\xi_\iph).
\end{gather}
\begin{itemize}

\item For the Lax-Friedrichs flux, the left inequality of \deq{1n2n3} clearly holds. Using \deq{cfl} and \deq{mvt}, the right inequality of \deq{1n2n3} is also satisfied. 

\item For the local Lax-Friedrichs flux, using \deq{cfl} and \deq{mvt}, both inequalities of \deq{1n2n3} is also satisfied. 

\item For the Lax-Wendroff scheme, we can use \deq{mvt} to get
\[
Q^{LW}(u_i, u_{i+1} ) = \frac{k}{h} f^\prime\left( \frac{u_i+u_\ipo}{2}\right)f^\prime(\xi_\iph) \ ,
\]
which may fail to be positive. Thus, the \deq{1n2n3} need not be satisfied.

\textit{
Remark: At times, the following alternate expression for the Lax-Wendroff flux is also used
\[
\widetilde{F}^{LW} = \frac{1}{2} \left( f(u) + f(v) - \frac{k}{h} \left(\frac{f(v) - f(u)}{v-u}\right)(f(v) - f(u))\right) \ ,
\]
which leads to
\[
\widetilde{Q}^{LW}(u,v) = \frac{k}{h} \left(\frac{f(v) - f(u)}{v-u}\right)^2 \ ,
\]
which is clearly positive. Using \deq{mvt}, we have
\[
\widetilde{Q}^{LW}(u_i, u_{i+1}) = \frac{k}{h} \left(f^\prime(\xi_\iph)\right)^2.
\]
In order to satisfy the right condition of \deq{1n2n3}, we need
\[
\frac{k}{h}| f^\prime(\xi_\iph)| \geq 1
\]
which violates the CFL condition \deq{cfl}.
}

\item For the Roe flux, we can once again show that \deq{1n2n3} is satisfied by using \deq{mvt} and \deq{cfl}.
\end{itemize}

\item Note that the Roe flux satisfies the condition of Harten's lemma, and thus leads to a TVD scheme. But we know that the Roe scheme may give entropy violating solutions. This implies that the TVD property alone does not ensure convergence to entropy solutions.

\end{enumerate}

\item Let us consider the hybrid flux 
\[
F^\theta(u,v) = \theta F^{LW} + (1-\theta) F^{LF}, \quad 0 \leq \theta \leq 1,
\]
which can also be written as
\[
F^\theta(u,v) = \frac{1}{2} \left( f(u) + f(v) - Q^\theta(u,v) (v-u)\right) \ , 
\]
where
\[
Q^\theta(u,v) = \theta Q^{LW}(u,v) + (1-\theta) Q^{LF}(u,v)\ .
\]
Assuming the flux to be linear, i.e., $f(u) = cu$, we get the simplified expression
\begin{gather} \label{hybQ}
Q^\theta(u,v) = \theta \frac{k}{h} c^2 + (1-\theta) \frac{h}{k} = \theta\left[ \left(\frac{k}{h} c\right)^2 - 1\right]\frac{h}{k} + \frac{h}{k}\ ,
\end{gather}
while the condition \deq{1n2n3} reduces to 
\begin{gather} \label{1n2n3_lin}
\frac{h}{k} \geq Q \geq |c|\ .
\end{gather}
If the left inequality of \deq{1n2n3_lin} need to hold, we must have
\[
\theta\left[ \left(\frac{k}{h} c\right)^2 - 1\right]\frac{h}{k} + \frac{h}{k} \leq \frac{h}{k} \quad \iff \quad \left(\frac{k}{h} c\right)^2 \leq 1 \quad \iff \quad \frac{k}{h} |c| \leq 1\ ,
\]
which is always true due to the CFL condition \deq{cfl}. The right inequality of \deq{1n2n3_lin} requires
\begin{align*}
\theta\left[ \left(\frac{k}{h} c\right)^2 - 1\right]\frac{h}{k} + \frac{h}{k} \geq |c| &\quad \iff \quad \theta\left[ \left(\frac{k}{h} c\right)^2 - 1\right]\geq \frac{k}{h} |c| -1 \\
&\quad \iff \quad \theta \leq \frac{1}{\frac{k}{h} |c| + 1} = \theta^* \quad \text{(using CFL condition)}.
\end{align*}
Thus, by choosing $\theta \in [0,\theta^*]$, we can recover a TVD scheme.

\end{exerciseList}



\newpage



\end{document}
