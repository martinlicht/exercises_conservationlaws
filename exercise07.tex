\documentclass{article}



\usepackage{../auxFiles/ExStyPac}



\input ../auxFiles/mac.tex
\graphicspath{{./figures/}}





%========================================================================
\begin{document}



\Head{7}{Linear Systems}




This exercise pertains to \emph{Godunov's method} for linear systems with constant coefficients,
\begin{gather} \label{linSys}
	q_{t}+Aq_{x}=0\ .
\end{gather}%
On linear systems, Godunov's method reduces to a generalization of the upwind method where the numerical flux is given by the following equivalent expressions 
\begin{gather}%al{GudFlx}
	F\frb{Q_l,Q_r} &=AQ_l+A^{-}\rb{Q_r-Q_l}\\
		& =AQ_r-A^{+}\rb{Q_r-Q_l}
		=\frac{1}{2}A\rb{Q_r+Q_l}-\frac{1}{2}|A|\rb{Q_r-Q_l}\ .
\end{gather}%a
Where $|A|=A^+-A^-$ and $A^{\pm}=S\Lambda^{\pm}S^{-1}$. Here $\Lam^+$ and $\Lam^-$ are diagonal matrices with non-negative and non-positive entries, respectively, such that $S^{-1}AS=\Lam$ is the spectral decomposition of $A$,  with $\Lam=\Lam^+ +\Lam^-$.
Especially, notice that
$$
	A=A^{+}+A^{-}\ .
$$
Consider the one dimensional acoustics equation
\begin{gather} \label{AcEq}
	\sqb{\mat{ p \\ v }}_{t} +\sqb{\mat{ u_{0} & K_{0} \\ 1/\rho_{0} & u_{0}}} \sqb{\mat{ p \\ v }}_{x}=0\ .
\end{gather}%
This system is derived from the nonlinear Euler equation by linearizing around some fixed state, as sound waves are small perturbation in a background media.
Here, $K_{0}$ is the compressibility modulus, and $u_{0}$ and $\rho_{0}$ are the the velocity and pressure, respectively.
The speed of sound in the medium is given by
$$
	c_{0}=\sqrt{K_{0}/\rho_{0}}\ .
$$




\begin{exerciseList}


%\item
%Write Gudonov's scheme for \eqref{linSys} with flux \eqref{GudFlx}.


\item
For \eqref{AcEq}, calculate $A^{+}$ and $A^{-}$. 


\item
What is the CFL condition of Godunov's method for \eqref{AcEq}?


\item
Implement Godunov's method for \eqref{AcEq} the following two sets of initial data
\begin{gather} \label{inData1}
	p(x,0)=\sin\frb{2\pi x}\ ,
	\quad
	v(x,0)=0, \quad \text{with periodic BC}
\end{gather}%
\begin{gather} \label{inData2}
	p(x,0)=\begin{cases}
		0 & x<0\\
		1 & x>0
	\end{cases}\ ,
	\quad
	v(x,0)=0, , \quad \text{with open BC}.
\end{gather}%
Use $u_{0}=1/2$, $K_{0}=1$, $p_{0}=1$ and solve on the interval
$x\in\left[-1,1\right]$ with $h=0.01$ to $T=0.4$ and an appropriate
time-step satisfying the CFL condition.

\item
In the solution driven by \eqref{inData2}, are the discontinuities visible in the numerical solution at $T=0.4$? Plot and compare with the exact solution at the final time.

\item
Now, run your code with the initial data
\begin{gather} \label{inData3}
	p(x,0)=\begin{cases}
		1 & x<0\\
		\sin\frb{2\pi x} & x>0
	\end{cases}\ ,
	\quad
	v(x,0)=0\ .
\end{gather}%
Does the exact solution preserve the discontinuities present in the initial condition? Are you able to observe the discontinuities in the numerical solution at $T=0.4$?




%\item
%The Godunov's method requires the solution of
%Riemann problems at every cell boundary in each timestep. In practice,
%solving all these Riemann problems is expensive, particularly for
%nonlinear equation. In the Godunov method, outlined in solution set
%4 to exercise set 4, we discard most information in the solved Riemann
%problem by taking the average of the solution over the grid cell.
%In this process large numerical errors are introduced. This suggest
%that we may be able to obtain equally good numerical results with
%an \textbf{approximate Riemann} solution obtained by less expensive
%means. There are many ways to go about constructing an approximate
%Riemann solver, one of the most popular Riemann solvers currently
%in use is the Roe's approximate Riemann solver. This method was presented
%during the lecture. The idea is to solve a constant coeffiicient linear
%system of conservation laws instead of the original nonlinear system,
%i.e., solve a modified conservation law with flux $\hat{f}\left(u\right)=\hat{A}\left(u_{l},u_{r}\right)u$
%for each cell boundary. \textbf{(a)} State the conditions that should
%be imposed on the coefficient matrix $\hat{A}\left(u_{l},u_{r}\right)$.
%\textbf{(b)} What does each statement imply? \textbf{(c)} Determine
%$\hat{A}\left(u_{l},u_{r}\right)$ for a general scalar conservation
%law. \textbf{(d)} Determine a numerical flux function when using Roe's
%approximate Riemann solver.



\end{exerciseList}

\end{document}






\documentclass{article}



\usepackage{../auxFiles/ExStyPac}

\usepackage[framed]{../auxFiles/mcode}





\input ../auxFiles/mac.tex
\graphicspath{{./figures/}}





%========================================================================
\begin{document}



\Head{7}{Linear Systems (Solution)}





\begin{exerciseList}



%\item
%To write the scheme in explicit
%form we simply insert the flux in the conservation law and reduce
%to get
%\begin{gather}
%	U_{j}^{n+1}=U_{j}^{n}
%		-\frac{k}{h}\left[A^{-}\left(U_{j+1}^{n}-U_{j}^{n}\right)
%			+A^{+}\left(U_{j}^{n}-U_{j-1}^{n}\right)\right]\ .
%\end{gather}

\item
We start by calculating the spectral decomposition of
\begin{gather}
	A=\sqb{\mat{
		u_{0} & K_{0}\\
		1/\rho_{0} & u_{0}
	}}\ .
\end{gather}
Since this is a $2\times2$ matrix, this can be done easily.
We have
\begin{gather}
	\lambda_{1}=u_{0}-c_{0},
	\quad
	S_{1}=\rb{-\rho_{0}c_{0}\ ,\ 1}^T\ ,
\end{gather}
and
\begin{gather}
	\lambda_{2}=u_{0}+c_{0},
	\quad
	S_{2}=\rb{\rho_{0}c_{0}\ , \ 1}^T\ ,
\end{gather}
where $c_{0}=\sqrt{K_{0}/\rho_{0}}$.
Thus,
\begin{equation}
	S=\sqb{\mat{
		-\rho_{0}c_{0} & \rho_{0}c_{0}\\
		1 & 1
	}}
	\quad
	S^{-1}=\frac{1}{2\rho_{0}c_{0}}
		\sqb{\mat{
			-1 & \rho_{0}c_{0}\\
			1 & \rho_{0}c_{0}
		}}\ .
\end{equation}
The matrices $\Lambda^{+}$ and $\Lambda^{-}$ can be expressed as
\begin{gather}
	\Lambda^{+}=\sqb{\mat{
		\lambda_1^+ & 0\\
		0 & \lambda_2^+
	}}
	\qquad
	\Lambda^{-}=\sqb{\mat{
		\lambda_1^- & 0\\
		0 & \lambda_2^-
	}}\ 
\end{gather}
where $\lambda_s^+ = \max\left(\lambda_s,0\right)$ and $\lambda_s^- = \min\left(\lambda_s,0\right)$ for $s=1,2$. Thus we obtain
\begin{gather}
	A^{+}=S\Lambda^{+}S^{-1}
		=\frac{1}{2}\sqb{\mat{
			(\lambda_1^+ + \lambda_2^+) & \rho_{0}c_{0} (-\lambda_1^+ + \lambda_2^+)\\
			(\rho_0c_0)^{-1} (-\lambda_1^+ + \lambda_2^+)& (\lambda_1^+ + \lambda_2^+)
		}}
\end{gather}
and
\begin{gather}
	A^{-}=S\Lambda^{-}S^{-1}
		=\frac{1}{2}\sqb{\mat{
			(\lambda_1^- + \lambda_2^-) & \rho_{0}c_{0} (-\lambda_1^- + \lambda_2^-)\\
			(\rho_0c_0)^{-1} (-\lambda_1^- + \lambda_2^-)& (\lambda_1^- + \lambda_2^-)
		}}\ .
\end{gather}


\item
The CFL condition is a necessary condition for stability.
Here, the CFL condition requires that for each eigenvalue $\lambda_{s}$ of $A$, the following must hold
\begin{gather}
	\frac{|\lambda_{s}|k}{h} \leq1\ .
\end{gather}
Making use of the eigenvalue structure for the given system, we arrive at the condition
\begin{equation}
	\left(|u_{0}|+c_{0}\right)\frac{k}{h}\leq1\ .
\end{equation}



\item
See the Matlab code attached at the end of this solution manual.


\item
See figures generated by the Matlab code.
Since the conservation law is a linear system, we know that discontinuities move only along characteristics and can not spontaneously form.
If the initial condition is smooth and satisfies the boundary conditions, then the solution is also smooth.
When using the second set of initial conditions, which is  discontinuous, the discontinuities
seem to get smeared out as the scheme advances.
This is however due to numerical diffusion in the scheme, and not because discontinuities disappear.


\item
See figures generated by the Matlab code attached at the end of this solution manual.
With this third set of initial conditions, it is not straightforward to locate the shocks as these are smeared out to a point where they are indistinguishable from other smooth regions of the solution.

Using the generalized upwind method for the linear system, we resolve and propagate information in a proper up-winded manner yielding less numerical diffusion than other schemes that return monotone solutions over discontinuities, such as the Lax-Friedrichs method.
However, the upwind method is still only a first order method and in general first order methods are not suitable for long time integration or resolving fine details.
Note that in the case of a linear system we can actually construct the exact solution.
See the Matlab code attached for an example on how to do this. 


\end{exerciseList}


%\newpage
\lstinputlisting{./Code/Exercise7.m}
\newpage
\lstinputlisting{./Code/GodunovFlux.m}
\lstinputlisting{./Code/apply_bc.m}
\lstinputlisting{./Code/find_exact.m}
\end{document}
