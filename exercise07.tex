\documentclass[10pt,letterpaper]{article}
\usepackage[english]{babel}
\usepackage{graphicx}
\usepackage[margin=2cm]{geometry}

\usepackage{float}
\usepackage{subfloat}
\usepackage{caption}
\usepackage{subcaption}
\usepackage{listings}

\usepackage{mathtools}
\usepackage{amsfonts}
\usepackage{amsmath}
\usepackage{amssymb}
\usepackage{amsthm}
\newcommand{\dif}[1][]{\mathrm{d} {#1}\,}
\newcommand{\rb}[1]{ \left(  {#1} \right) }
\newcommand{\frb}[1]{ \left(  {#1} \right) }
\newcommand{\norm}[1]{ \left\|  {#1} \right\| }
\newcommand{\jph}{{j+\frac{1}{2}}}
\newcommand{\jmh}{{j-\frac{1}{2}}}
\newcommand{\sqb}[1]{ \left(  {#1} \right) }


% \usepackage[amsmath,thmmarks,standard]{ntheorem}
\newtheoremstyle{break}
{\topsep}
{\topsep}%
{\normalfont}
{}%
{\bfseries}
{}%
{\newline}
{}%
\theoremstyle{break}
\newtheorem{exercise}{Exercise}
\newtheorem*{information}{Information}
\newtheorem{mysolution}{Solution}
% \newenvironment{solution}{\begin{comment}}{\end{comment}}
\newtheorem*{solutioninformation}{Solution Information}

\usepackage{comment}
% Switch between showing and hiding solutions by commenting out either of the following lines
\newenvironment{solution}{\begin{mysolution}}{\end{mysolution}}
% \excludecomment{solution}






\begin{document}

\title{Linear Systems}
\date{}

\maketitle

















\begin{exercise}

	%\item
	%Write Gudonov's scheme for \eqref{linSys} with flux \eqref{GudFlx}.

	This exercise pertains to \emph{Godunov's method} for linear systems with constant coefficients,
	\begin{gather} \label{linSys}
		q_{t}+Aq_{x}=0\ .
	\end{gather}%
	On linear systems, Godunov's method reduces to a generalization of the upwind method where the numerical flux is given by the following equivalent expressions 
	\begin{align}%al{GudFlx}
		F\frb{Q_l,Q_r}
		&=AQ_l+A^{-}\rb{Q_r-Q_l}\\
		&=AQ_r-A^{+}\rb{Q_r-Q_l}
		=\frac{1}{2}A\rb{Q_r+Q_l}-\frac{1}{2}|A|\rb{Q_r-Q_l}\ .
	\end{align}%a
	Where $|A|=A^+-A^-$ and $A^{\pm}=S\Lambda^{\pm}S^{-1}$. Here $\Lambda^+$ and $\Lambda^-$ are diagonal matrices with non-negative and non-positive entries, respectively, such that $S^{-1}AS=\Lambda$ is the spectral decomposition of $A$,  with $\Lambda=\Lambda^+ +\Lambda^-$.
	Especially, notice that
	\begin{align}
		A=A^{+}+A^{-}\ .
	\end{align}
	Consider the one dimensional acoustics equation
	\begin{align} \label{AcEq}
		\sqb{ 
			\begin{array}{c} p \\ v \end{array}
		}_{t} 
		+
		\sqb{
			\begin{array}{cc} u_{0} & K_{0} \\ 1/\rho_{0} & u_{0} \end{array}
		}
		\sqb{
			\begin{array}{c} p \\ v \end{array}
		}_{x}=0\ .
	\end{align}%
	This system is derived from the nonlinear Euler equation by linearizing around some fixed state, as sound waves are small perturbation in a background media.
	Here, $K_{0}$ is the compressibility modulus, and $u_{0}$ and $\rho_{0}$ are the the velocity and pressure, respectively.
	The speed of sound in the medium is given by
	\begin{align}
		c_{0}=\sqrt{K_{0}/\rho_{0}}\ .
	\end{align}

	\begin{enumerate}

		\item
		For \eqref{AcEq}, calculate $A^{+}$ and $A^{-}$. 


		\item
		What is the CFL condition of Godunov's method for \eqref{AcEq}?


		\item
		Implement Godunov's method for \eqref{AcEq} the following two sets of initial data
		\begin{gather} \label{inData1}
			p(x,0)=\sin\frb{2\pi x}\ ,
			\quad
			v(x,0)=0, \quad \text{with periodic BC}
		\end{gather}%
		\begin{gather} \label{inData2}
			p(x,0)=\begin{cases}
				0 & x<0\\
				1 & x>0
			\end{cases}\ ,
			\quad
			v(x,0)=0, , \quad \text{with open BC}.
		\end{gather}%
		Use $u_{0}=1/2$, $K_{0}=1$, $p_{0}=1$ and solve on the interval
		$x\in\left[-1,1\right]$ with $h=0.01$ to $T=0.4$ and an appropriate
		time-step satisfying the CFL condition.

		\item
		In the solution driven by \eqref{inData2}, are the discontinuities visible in the numerical solution at $T=0.4$? Plot and compare with the exact solution at the final time.

		\item
		Now, run your code with the initial data
		\begin{gather} \label{inData3}
			p(x,0)=\begin{cases}
				1 & x<0\\
				\sin\frb{2\pi x} & x>0
			\end{cases}\ ,
			\quad
			v(x,0)=0\ .
		\end{gather}%
		Does the exact solution preserve the discontinuities present in the initial condition? Are you able to observe the discontinuities in the numerical solution at $T=0.4$?




		%\item
		%The Godunov's method requires the solution of
		%Riemann problems at every cell boundary in each timestep. In practice,
		%solving all these Riemann problems is expensive, particularly for
		%nonlinear equation. In the Godunov method, outlined in solution set
		%4 to exercise set 4, we discard most information in the solved Riemann
		%problem by taking the average of the solution over the grid cell.
		%In this process large numerical errors are introduced. This suggest
		%that we may be able to obtain equally good numerical results with
		%an \textbf{approximate Riemann} solution obtained by less expensive
		%means. There are many ways to go about constructing an approximate
		%Riemann solver, one of the most popular Riemann solvers currently
		%in use is the Roe's approximate Riemann solver. This method was presented
		%during the lecture. The idea is to solve a constant coeffiicient linear
		%system of conservation laws instead of the original nonlinear system,
		%i.e., solve a modified conservation law with flux $\hat{f}\left(u\right)=\hat{A}\left(u_{l},u_{r}\right)u$
		%for each cell boundary. \textbf{(a)} State the conditions that should
		%be imposed on the coefficient matrix $\hat{A}\left(u_{l},u_{r}\right)$.
		%\textbf{(b)} What does each statement imply? \textbf{(c)} Determine
		%$\hat{A}\left(u_{l},u_{r}\right)$ for a general scalar conservation
		%law. \textbf{(d)} Determine a numerical flux function when using Roe's
		%approximate Riemann solver.




	\end{enumerate}
\end{exercise}

\begin{solution}
	\begin{enumerate}
		\item
		We start by calculating the spectral decomposition of
		\begin{gather}
			A=\sqb{
				\begin{array}{cc}
					u_{0} & K_{0}\\
					1/\rho_{0} & u_{0}
				\end{array}
			}\ .
		\end{gather}
		Since this is a $2\times2$ matrix, this can be done easily.
		We have
		\begin{gather}
			\Lambda_{1}=u_{0}-c_{0},
			\quad
			S_{1}=\rb{-\rho_{0}c_{0}\ ,\ 1}^T\ ,
		\end{gather}
		and
		\begin{gather}
			\Lambda_{2}=u_{0}+c_{0},
			\quad
			S_{2}=\rb{\rho_{0}c_{0}\ , \ 1}^T\ ,
		\end{gather}
		where $c_{0}=\sqrt{K_{0}/\rho_{0}}$.
		Thus,
		\begin{equation}
			S=\sqb{\begin{array}{cc}
				-\rho_{0}c_{0} & \rho_{0}c_{0}\\
				1 & 1
			\end{array}}
			\quad
			S^{-1}=\frac{1}{2\rho_{0}c_{0}}
				\sqb{\begin{array}{cc}
					-1 & \rho_{0}c_{0}\\
					1 & \rho_{0}c_{0}
				\end{array}}\ .
		\end{equation}
		The matrices $\Lambda^{+}$ and $\Lambda^{-}$ can be expressed as
		\begin{gather}
			\Lambda^{+}=\sqb{\begin{array}{cc}
				\Lambda_1^+ & 0\\
				0 & \Lambda_2^+
			\end{array}}
			\qquad
			\Lambda^{-}=\sqb{\begin{array}{cc}
				\Lambda_1^- & 0\\
				0 & \Lambda_2^-
			\end{array}}\ 
		\end{gather}
		where $\Lambda_s^+ = \max\left(\Lambda_s,0\right)$ and $\Lambda_s^- = \min\left(\Lambda_s,0\right)$ for $s=1,2$. Thus we obtain
		\begin{gather}
			A^{+}=S\Lambda^{+}S^{-1}
				=\frac{1}{2}\sqb{\begin{array}{cc}
					(\Lambda_1^+ + \Lambda_2^+) & \rho_{0}c_{0} (-\Lambda_1^+ + \Lambda_2^+)\\
					(\rho_0c_0)^{-1} (-\Lambda_1^+ + \Lambda_2^+)& (\Lambda_1^+ + \Lambda_2^+)
				\end{array}}
		\end{gather}
		and
		\begin{gather}
			A^{-}=S\Lambda^{-}S^{-1}
				=\frac{1}{2}\sqb{\begin{array}{cc}
					(\Lambda_1^- + \Lambda_2^-) & \rho_{0}c_{0} (-\Lambda_1^- + \Lambda_2^-)\\
					(\rho_0c_0)^{-1} (-\Lambda_1^- + \Lambda_2^-)& (\Lambda_1^- + \Lambda_2^-)
				\end{array}}\ .
		\end{gather}


		\item
		The CFL condition is a necessary condition for stability.
		Here, the CFL condition requires that for each eigenvalue $\Lambda_{s}$ of $A$, the following must hold
		\begin{gather}
			\frac{|\Lambda_{s}|k}{h} \leq1\ .
		\end{gather}
		Making use of the eigenvalue structure for the given system, we arrive at the condition
		\begin{equation}
			\left(|u_{0}|+c_{0}\right)\frac{k}{h}\leq1\ .
		\end{equation}



		\item
		See the Matlab code attached at the end of this solution manual.


		\item
		See figures generated by the Matlab code.
		Since the conservation law is a linear system, we know that discontinuities move only along characteristics and can not spontaneously form.
		If the initial condition is smooth and satisfies the boundary conditions, then the solution is also smooth.
		When using the second set of initial conditions, which is  discontinuous, the discontinuities
		seem to get smeared out as the scheme advances.
		This is however due to numerical diffusion in the scheme, and not because discontinuities disappear.


		\item
		See figures generated by the Matlab code attached at the end of this solution manual.
		With this third set of initial conditions, it is not straightforward to locate the shocks as these are smeared out to a point where they are indistinguishable from other smooth regions of the solution.

		Using the generalized upwind method for the linear system, we resolve and propagate information in a proper up-winded manner yielding less numerical diffusion than other schemes that return monotone solutions over discontinuities, such as the Lax-Friedrichs method.
		However, the upwind method is still only a first order method and in general first order methods are not suitable for long time integration or resolving fine details.
		Note that in the case of a linear system we can actually construct the exact solution.
		See the Matlab code attached for an example on how to do this. 


	% \lstinputlisting{./07Code/Exercise7.m}
	\begin{lstlisting}
		% Solution07
		% This script was written for EPFL MATH459, Numerical Methods for
		% Conservation Laws.
		% The one dimensional linearized acoustic equations are solved with
		% periodic/open boundary conditions, and initial data
		% as given in the exercise.

		clc
		clear all
		close all

		% Initial data set
		data = 3;
		switch data
			case 1
				pIC =@(x) sin(2*pi*x);
				vIC =@(x) 0*x;
				bc  = 'Periodic';
			case 2
				pIC =@(x) 1*(x>0);
				vIC =@(x) 0*x;
				bc  = 'Open';
			case 3
				pIC =@(x) 1*(x<0) + sin(2*pi*x).*(x>=0);
				vIC =@(x) 0*x;
				bc  = 'Open';
		end

		% Define Discretization and time parameters
		h  = 0.01;
		xf = -1:h:1;
		xc = (-1+0.5*h):h:(1-0.5*h);
		N  = length(xc);
		Tfinal = 0.4;
		CFL    = 0.5;

		% Physical constants [u0,k0,p0]
		u0 = 1/2;
		K0 = 1;
		p0 = 1;
		c0 = sqrt(K0/p0);

		% Find various matrices
		A      = [u0,K0;1/p0,u0];
		S      = [-p0*c0, p0*c0; 1,1];
		Sinv   = [-1, p0*c0; 1, p0*c0]/(2*p0*c0);
		Lambda = [u0-c0,0;0,u0+c0];
		absA   = S*abs(Lambda)*Sinv;


		% Averaging initial conditions
		% Cell-center values sufficient for first-order schemes
		U = [pIC(xc);vIC(xc)];

		time = 0; iter = 0;
		plot_every = 10;

		% Solve
		while time < Tfinal
			
			k = CFL*h/(abs(u0) + c0);
			if(time + k > Tfinal)
				k = Tfinal - time;
			end
			
			% Applying boundary conditions to obtain extended vector
			U_ext = apply_bc(U,bc);
			
			Flux = GodunovFlux(A,absA,U_ext(:,1:end-1),U_ext(:,2:end));
			U = U - k/h*(Flux(:,2:end) - Flux(:,1:end-1));
			
			time = time + k;
			iter = iter + 1;
			
			if(mod(iter,plot_every)==0 || time ==Tfinal)
				% Finding exact solution at the current time
				Uexact = find_exact(pIC,vIC,S,Sinv,Lambda,xc,time);
				
				% Visualize the solution
				figure(1)
				subplot(2,1,1)
				plot(xc,U(1,:),'-r','LineWidth',2);
				hold all
				plot(xc,Uexact(1,:),'--k','LineWidth',2);
				ylim([-2 2]);xlim([-1 1]);
				legend('Numerical Pressure','Exact Pressure','Location','Best')
				grid on;
				title(['Time = ',num2str(time)])
				hold off
				
				subplot(2,1,2)
				plot(xc,U(2,:),'-r','LineWidth',2);
				hold all
				plot(xc,Uexact(2,:),'--k','LineWidth',2);
				ylim([-2 2]);xlim([-1 1]);
				legend('Numerical Velocity','Exact Velocity','Location','Best')
				grid on;
				hold off
				
			end
			
		end
	\end{lstlisting}

	
	% \lstinputlisting{./07Code/GodunovFlux.m}
	\begin{lstlisting}
		function Flux = GodunovFlux(A,absA,UL,UR)

		Flux = 0.5*A*(UL+UR) - 0.5*absA*(UR-UL);

		end
	\end{lstlisting}

	% \lstinputlisting{./07Code/apply_bc.m}
	\begin{lstlisting}
		% Function returns an extended vector, based on
		% the type of boundary condition requested

		function U_ext = apply_bc(U,bc)

		switch bc
			case 'Periodic'
				U_ext = [U(:,end) , U, U(:,1)];
			case 'Open'
				U_ext = [U(:,1)   , U, U(:,end)];
		end
	\end{lstlisting}
	
	% \lstinputlisting{./07Code/find_exact.m}
	\begin{lstlisting}
		% Function evaluates the exact solution at a given time

		function Uexact = find_exact(pIC,vIC,S,Sinv,Lambda,xc,time)

		VSelect = @(V,IDX) V(IDX,:);
		V0 =@(x) Sinv*[pIC(x);vIC(x)];
		VT_1 =@(x) VSelect(V0(x),1);
		VT_2 =@(x) VSelect(V0(x),2);

		Uexact = S*[VT_1(xc-Lambda(1,1)*time);
		VT_2(xc-Lambda(2,2)*time)];
	\end{lstlisting}
	

	\end{enumerate}
\end{solution}
	

\end{document}
