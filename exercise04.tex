\documentclass[10pt,letterpaper]{article}
\usepackage[english]{babel}
\usepackage{graphicx}
\usepackage[margin=2cm]{geometry}

\usepackage{subcaption}
\usepackage{mathtools}
\usepackage{amsfonts}
\usepackage{amsmath}
\usepackage{amssymb}
\usepackage{amsthm}
\newcommand{\dif}[1][]{\mathrm{d} {#1}\,}
\newcommand{\rb}[1]{ \left(  {#1} \right) }
\newcommand{\frb}[1]{ \left(  {#1} \right) }

% \usepackage[amsmath,thmmarks,standard]{ntheorem}
\newtheoremstyle{break}
{\topsep}
{\topsep}%
{\normalfont}
{}%
{\bfseries}
{}%
{\newline}
{}%
\theoremstyle{break}
\newtheorem{exercise}{Exercise}
\newtheorem*{information}{Information}
\newtheorem{solution}{Solution}
\newtheorem*{solutioninformation}{Solution Information}

\begin{document}

\title{}
\date{}











\begin{exercise}[Finite Differences Method]
	Define the following concepts:
	\begin{itemize}
	\item[(a)] Consistency;
	\item[(b)] Stability; and,
	\item[(c)] Convergence.
	\end{itemize}
	Discuss the importance of these concepts, how they relate to each other, and how in general they can be established.
\end{exercise}

\begin{exercise}[Leapfrog Method]
	The forward time, center space finite difference
	approximation to the advection equation $u_t+au_x=0$
	leads to a method which is \emph{unstable}.
	So let us try something different.
	By approximating the time derivative by a centered difference,
	instead of the forward Euler discretization, we get the Leapfrog method,
	\begin{equation}\label{eq:LFrog}
		v_{j}^{n+1}=v_{j}^{n-1}-\frac{ak}{h}(v_{j+1}^{n}-v_{j-1}^{n})\ .
	\end{equation}
	\begin{itemize}
		\item[(i)]	Draw the stencil of \eqref{eq:LFrog}.

		\item[(ii)]	Show that \eqref{eq:LFrog} is second order accurate in both space and time.

		\item[(iii)] What is an obvious disadvantage of the Leapfrog method compared to the Lax-Friedrichs or Lax-Wendroff methods?
	\end{itemize}
\end{exercise}

\begin{exercise}[Stability of the Lax-Friedrichs Method]
	As a possible numerical method for the linear transport equation $u_t+au_x=0$, consider the Lax-Friedrichs method
	\begin{equation}
		v_{j}^{n+1}=\frac{1}{2}(v_{j+1}^{n}+v_{j-1}^{n})-\frac{ak}{2h}(v_{j+1}^{n}-v_{j-1}^{n})
	\end{equation}
	Show that this method is stable in the $l^\infty$ norm, provided that $k$ and $h$ satisfy the \textit{CFL condition}
	\begin{equation}
		\frac{|a|k}{h}\leq1.
	\end{equation}
\end{exercise}

\begin{exercise}[Unconditionally Stable Method]
	We have seen in the previous exercise that the Lax-Friedrichs
	method is stable provided $k$ and $h$ satisfy a CFL condition.
	A method which is stable for any $k$ and $h$ is said to be
	\textit{unconditionally stable}.
	For $a>0$, prove that the following backward-time backward-space method
	$$
		v_{j}^{n+1}=v_j^n-\frac{ak}{h}(v_{j}^{n+1}-v_{j-1}^{n+1})
	$$
	is unconditionally stable in the $l^\infty$ norm.
	%\item
	%\begin{enumerate}
	%\item
	%Explain qualitatively what the CFL condition is.
	%
	%\item
	%Is the CFL condition sufficient for stability?
	%If not, can you think of a counter example?
	%\end{enumerate}
\end{exercise}


\begin{exercise}[Matlab Implementation]
	This exercise involves some programming.
	Consider the scalar advection equation \eqref{eq:advection_eq}
	with $a=1$.
	Let $u$ be the solution of $u_t+au_x=0$
	in $(-1,1)$ that satisfies initial condition
	\begin{equation}
		u(x,0)=u_{0}(x),
	\end{equation}
	where
	\begin{equation}\label{eq:init_cond}
		u_{0}(x)
		=
		\begin{cases}
				1 & x<0\\
				0 & x>0
		\end{cases}\ ,
	\end{equation}
	and boundary conditions
	\begin{equation}\label{eq:bc_cond}
		u(-1,t)=1
		\quad\quad
		u(1,t)=0
	\end{equation}
	We already know that the exact solution for $0<t<1$ is given by $u_{0}(x-at)$,
	but how well do numerical methods approximate such a problem?
	In the following consider the schemes
	\begin{align*}
		\textrm{Upwind:}\quad
			v_{j}^{n+1} & =v_{j}^{n}-\frac{ak}{h}(v_{j}^{n}-v_{j-1}^{n})\\
		\textrm{Lax-Friedrichs:}\quad
			v_{j}^{n+1} & =\frac{1}{2}(v_{j+1}^{n}+v_{j-1}^{n})
				-\frac{ak}{2h} (v_{j+1}^{n}-v_{j-1}^{n})\\
		\textrm{Lax-Wendroff:}\quad
			v_{j}^{n+1} & =v_{j}^{n}-\frac{ak}{2h}(v_{j+1}^{n}-v_{j-1}^{n})
				+\frac{(ak)^2}{2h^2}(v_{j+1}^{n}-2v_{j}^{n}+v_{j-1}^{n})\\
		\textrm{Beam-Warming:}\quad
			v_{j}^{n+1} & =v_{j}^{n}-\frac{ak}{2h}(3v_{j}^{n} -4v_{j-1}^{n}+v_{j-2}^{n})
				+\frac{(ak)^{2}}{2h^{2}} (v_{j}^{n}-2v_{j-1}^{n}+v_{j-2}^{n})\ .
	\end{align*}

	For each of the schemes above:
	\begin{enumerate}
	\item
	Implement the scheme in Matlab to solve $u_t+au_x=0$, 
	in $(-1,1)$, with \eqref{eq:init_cond}, and \eqref{eq:bc_cond} in the time interval $t\in[0,0.5]$. 
	For your computations use $h=0.0025$ and $k/h=0.5$.

	\item
	Visualize the numerical solution and the exact solution.

	\item
	Comment qualitatively on the solutions' behavior.

	\item
	Write a script to generate a log-log plot of the error as a function of the resolution 
	(either $h$, or the number of points in the spatial grid), 
	while keeping the ratio $k/h=0.5$ fixed (why is this important?).

	\item
	Use the plot to deduce the accuracy of the method.

	\item
	Compare your results with the accuracy you would expect on a smooth solution.
	\end{enumerate}
\end{exercise}


\end{document}
