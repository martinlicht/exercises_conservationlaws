\documentclass{article}



\usepackage{../auxFiles/ExStyPac}

\usepackage[framed]{../auxFiles/mcode}



\newcommand{\iph}{{i + \frac{1}{2}}}
\newcommand{\imh}{{i - \frac{1}{2}}}

\input ../auxFiles/mac.tex
\graphicspath{{./figures/}}





%========================================================================
\begin{document}



\Head{10}{High-Order Schemes and stencil selection (Solution)}





\begin{exerciseList}

\item The interface values are given by
\begin{gather} \label{rec}
u_\iph^- = \sum \limits_{j=0}^2 c_{r,j} \overline{u}_{i-r+j}\ , \qquad  u_\imh^+ = \sum \limits_{j=0}^2 \tilde{c}_{r,j} \overline{u}_{i-r+j}
\end{gather}
where $\tilde{c}_{r,j} = c_{r-1,j}$. The coefficients are given in Table \ref{tab:crj}. Note that we also define the values for r=-1, since it is needed to define $\tilde{c}_{r,j}$ when r=0.

\begin{table}[htbp]
\centering
\begin{tabular}{|c|c|c|c|}
\hline
\multicolumn{1}{|l|}{\textbf{\begin{tabular}[c]{@{}c@{}}   \\ \hspace{0.5cm} j\\      r\end{tabular}}} & \multicolumn{1}{c|}{\textbf{0}} & \multicolumn{1}{c|}{\textbf{1}} & \multicolumn{1}{c|}{\textbf{2}} \\ \hline
\textbf{-1}                                                                       & $\frac{11}{6}$                  & $-\frac{7}{6}$                  & $\frac{1}{3}$                   \\ \hline
\textbf{0}                                                                        & $\frac{1}{3}$                   & $\frac{5}{6}$                   & $-\frac{1}{6}$                  \\ \hline
\textbf{1}                                                                        & $-\frac{1}{6}$                  & $\frac{5}{6}$                   & $\frac{1}{3}$                   \\ \hline
\textbf{2}                                                                        & $\frac{1}{3}$                   & $-\frac{7}{6}$                  & $\frac{11}{6}$                  \\ \hline
\end{tabular}
\caption{Table of coefficients.}
\label{tab:crj}
\end{table}

\item
See the Matlab code attached at the end of this solution manual.


\item
See figures generated by the Matlab code.

The solutions evaluated with r=0,2 are not stable, and blow up quickly with mesh refinement. However, the solutions are stable with r=1, which corresponds to choosing the central stencil for each cell. Third-order accuracy is observed for the smooth initial condition with r=1, which drops to below first-order for the discontinuous data. Furthermore, small Gibbs oscillations appear near the discontinuity.

The results from this exercise indicate that simply using a high-order reconstruction may not ensure stability. One can obtain stable solutions by adaptively choosing the stencil for reconstruction in each cell. This is the strategy used in the so called \textit{essentially non-oscillatory} (ENO) reconstruction method.


\end{exerciseList}


\newpage
\lstinputlisting{./Code/Exercise10.m}
\lstinputlisting{./Code/solver.m}
\lstinputlisting{./Code/apply_bc.m}
\lstinputlisting{./Code/evalRHS.m}
\lstinputlisting{./Code/rec3.m}
\lstinputlisting{./Code/find_exact.m}
\lstinputlisting{./Code/find_err.m}

\end{document}
