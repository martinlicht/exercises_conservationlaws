\documentclass{article}



\usepackage{../auxFiles/ExStyPac}

\usepackage[framed]{../auxFiles/mcode}





\input ../auxFiles/mac.tex
\graphicspath{{./figures/}}





%========================================================================
\begin{document}



\Head{11}{ENO Reconstruction (Solution)}





\begin{exerciseList}


\item
See the Matlab code attached at the end of this solution manual. The stencil selection depends on the index shift $r$, which is evaluated by the function {\tt find\_shift(U,k)}.

\item
The coefficients $c_{rj}$ are computed once at the beginning of the solver by the function {\tt eval\_crj(k)}, assuming a uniform grid. 

\item Once the shift $r$ is evaluate for each cell, the solution at the left and right interfaces of the cell are computed in the function {\tt eno\_recon}.

\item Note that the code is capable of implementing both periodic and open boundary conditions, for arbitrary order ENO. The order can be set via the variable $k$.

\item
Third-order convergence is obtained for the smooth initial condition. Moreover, the oscillations observed with the discontinuous initial data for fixed central reconstruction in Exercise 9, are no longer visible.


\end{exerciseList}


%\newpage
\lstinputlisting{./Code/Exercise10.m}
\lstinputlisting{./Code/solver.m}
\lstinputlisting{./Code/evalRHS.m}
\lstinputlisting{./Code/apply_bc.m}
\lstinputlisting{./Code/eval_crj.m}
\lstinputlisting{./Code/eno_recon.m}
\lstinputlisting{./Code/find_shift.m}
\lstinputlisting{./Code/find_exact.m}
\lstinputlisting{./Code/find_err.m}
\end{document}
