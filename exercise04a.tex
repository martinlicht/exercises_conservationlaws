\documentclass[10pt,letterpaper]{article}
\usepackage[english]{babel}
\usepackage{graphicx}
\usepackage[margin=2cm]{geometry}

\usepackage{subcaption}
\usepackage{mathtools}
\usepackage{amsfonts}
\usepackage{amsmath}
\usepackage{amssymb}
\usepackage{amsthm}
\newcommand{\dif}[1][]{\mathrm{d} {#1}\,}
\newcommand{\rb}[1]{ \left(  {#1} \right) }
\newcommand{\frb}[1]{ \left(  {#1} \right) }
% \usepackage[amsmath,thmmarks,standard]{ntheorem}
\newtheoremstyle{break}
{\topsep}
{\topsep}%
{\normalfont}
{}%
{\bfseries}
{}%
{\newline}
{}%
\theoremstyle{break}
\newtheorem{exercise}{Exercise}
\newtheorem*{information}{Information}
\newtheorem{mysolution}{Solution}
% \newenvironment{solution}{\begin{comment}}{\end{comment}}
\newtheorem*{solutioninformation}{Solution Information}

\usepackage{comment}
% Switch between showing and hiding solutions by commenting out either of the following lines
\newenvironment{solution}{\begin{mysolution}}{\end{mysolution}}
% \excludecomment{solution}






\begin{document}

\title{Rarefaction Waves and Shocks}
\date{}

\maketitle





























\begin{exercise}[Total Variation of Rarefaction wave]
    Consider the rarefaction wave 
    \begin{align}
     u(x,t)
     = 
     \left\{
     \begin{array}{ll}
        u_l & x < f'(u_l) t,
        \\
        \frac 1 2 \frac x t & f'(u_l)t < x < f'(u_r)t,
        \\
        u_r & f'(u_r)t < x.
     \end{array}
     \right.
    \end{align}
    This solves Burgers' equation ($f(u)=u^2$) with initial data 
    \begin{align}
     u_{0}(x)
     = 
     \left\{
     \begin{array}{cc}
        u_l & x < 0,
        \\
        u_r & x > 0,
     \end{array}
     \right.
    \end{align}
    where $u_l < u_r$. 
    \begin{enumerate}
     \item Compute the total variation of $u(\cdot,t)$ for all times $t \geq 0$.
     \item Show that
     \begin{align}
      TV(u_{0}) = \sup\left\{ \int u_0(x) \phi'(x) dx \;\middle|\; \phi \in C^{1}(\mathbb R) \text{ with compact support}, \; \max |\phi| \leq 1 \right\}
     \end{align}
    \end{enumerate}
\end{exercise}


\begin{solution}
    \begin{enumerate}
     \item 
     For $t = 0$, the total variation is just the jump of $u_0$, that is, $TV(u(\cdot,0)) = |u_l - u_r|$.
     Therefore we now assume that $t > 0$.
     
     The function $u$ is continuous, and it is constant outside of the interval $[ f'(u_l)t, f'(u_r)t ]$.
     Using the formulation of total variation for piecewise smooth functions, we see that 
     \begin{align}
      TV(u(\cdot,t))
      =
      \int_{-\infty}^{\infty} | \partial_x u(\cdot,t) | dx
      =
      \int_{f'(u_l)t}^{f'(u_r)t} | \partial_x u(\cdot,t) | dx
     \end{align}
     We can substitute the definition of $u$ and the flux. Then we use the fundamental theorem of calculus:
     \begin{align}
      TV(u(\cdot,t))
      =
      \int_{2 u_l t}^{2 u_r t} \frac 1 {2t} dx
      =
      \frac 1 2 ( 2 u_l t - 2 u_r t ) \frac 1 {2t}
      =
      u_r - u_l.
     \end{align}
     Note that the total variation stays constant. 
    
     
     \item Recall that $TV(u_{0}) = u_r - u_l$. Now pick any test function $\phi \in C^{1}(\mathbb R)$ with compact support, satisfying $\max |\phi| \leq 1$.
     We compute 
     \begin{align}
      \int u_0(x) \phi'(x) dx
      &
      =
      \int_{-\infty}^{0} u_0(x) \phi'(x) dx
      +
      \int_{0}^{-\infty} u_0(x) \phi'(x) dx
      \\&
      =
      -
      u_0(0-) \phi(x)
      +
      u_0(0+) \phi(x)
      =
      -
      u_l \phi(x)
      +
      u_r \phi(x)
      .
     \end{align}
     On the one hand, since $|\phi|$ is bounded from above by $1$, the integral is bounded from above by $u_r - u_l$.
     This implies that the total variation is upper bound for the supremum.
     On the other hand, the supremum attains the value $u_r - u_l$ when we pick any test function satisfying the above constraints 
     such that $\phi(0) = 1$. 
    \end{enumerate}
\end{solution}





\begin{exercise}[Rarefaction Wave]
    Find the rarefaction wave for the following conservation law and initial data:
    \begin{align}
     \partial_t u + \partial_x \left( \exp(u) \right) = 0,
     \quad 
     u_{0}(x)
     = 
     \left\{
     \begin{array}{cc}
        1 & x < 0,
        \\
        2 & x > 0.
     \end{array}
     \right.
    \end{align}
\end{exercise}

\begin{solution}
    From the formulation 
    \begin{align}
     \partial_t u + \exp(u) \partial_x u = 0
    \end{align}
    the direction of the characteristics are apparent. 
    In particular, they move to the right regardless of the value of $u$. 
    Drawing a picture immediately shows that we need to fill the area 
    \begin{align}
     f'(1) < x/t < f'(2)
    \end{align}
    with values to define the solution. 
    We pick characteristics emerging from the origin (otherwise, they would emerge from a shock).
    
    Along each characteristics along which the value $u^{\ast}$ is transported, we have 
    \begin{align}
     x = f'(u^{\ast}) t
     \quad\Longleftrightarrow\quad
     x = \exp(u^{\ast}) t
     \quad\Longleftrightarrow\quad
     \ln(x/t) = u^{\ast}
     .
    \end{align}
    Consequently, the rarefaction wave is given by 
    \begin{align}
     u(x,t)
     = 
     \left\{
     \begin{array}{ll}
        u_l & x < \exp(u_l) t,
        \\
        \ln(x/t) & \exp(u_l)t < x < \exp(u_r)t,
        \\
        u_r & \exp(u_r)t < x.
     \end{array}
     \right.
    \end{align}
\end{solution}






\begin{exercise}[Another Shock wave]
    Solve Burgers' equation ($f(u)=u^2$) with initial data 
    \begin{align}
     u_{0}(x)
     = 
     \left\{
     \begin{array}{cc}
        1 & x < 0,
        \\
        (1-x)/2 & 0 < x < 1,
        \\
        0 & x > 1.
     \end{array}
     \right.
    \end{align}
    Find the solution with the method of characteristics and tools you already know.
    At what time does the solution change its structure? 
\end{exercise}

\begin{solution}
    The characteristics over the interval $[0,1]$ travel with speed $2u^{\ast}$, where $u^{\ast}$ is the value along the characteristic.
    We see that they merge at time $t=1/2$ and position $x=1$. At that point, the solution will be discontinuity,
    \begin{align}
     u(x,1)
     = 
     \left\{
     \begin{array}{cc}
        1 & x < 1,
        \\
        0 & x > 1,
     \end{array}
     \right.
    \end{align}
    and from there on, the solutions will be a traveling discontinuity. The speed of the discontinuity is $1$, as can be checked by the Rankine-Hugoniot condition.
\end{solution}



\begin{exercise}[Flux discontinuous in $x$]

    \textit{Conservation laws where the flux is discontinuous in $x$ have attracted some attention over the past years. The following example models, e.g., traffic flow with varying speed limits.}
    \\
    
    \noindent 
    Consider the conservation law 
    \begin{align}
     \partial_t u + \partial_x\left( a(x) u \right) = 0
    \end{align}
    with speed variable 
    \begin{align}
     a(x)
     = 
     \left\{
     \begin{array}{cc}
        1 & x < 0,
        \\
        2 & x > 0.
     \end{array}
     \right.
    \end{align}
    \begin{enumerate}
        \item
        Sketch a solution with the method of characteristics.
        For which $(x,t)$ is it a strong solution?
        \item
        Show that it is a weak solution under the simplifying assumption that the initial data $u_0 \in C^{1}(\mathbb R)$ are differentiable. 
        \textit{You can use an argument similar to the RH condition. You do not need an explicit formula for $u$.}
    \end{enumerate}
\end{exercise}

\begin{solution}
    \begin{enumerate}
        \item
        To the left of the line $x=0$, the characteristics emerge with speed $1$, whereas to the right of $x=0$, they emerge with speed $2$.
        All the characteristics on the left eventually reach the line $x=0$, from whereon they proceed with speed $2$.
        The solution is continuous but not necessarily differentiable at $x=0$.
        If the initial data $u_{0}$ are differentiable at $x_0 \in \mathbb R$, then it is differentiable at the characteristic emerging at $x_0$, except for the crossover at $x=0$.
        Wherever the solution is differentiable, it is also a strong solution. 
        \item
        Suppose that the function $u_0 \in C^{1}(\mathbb R)$ is differentiable. 
        The solution constructed above is differentiable to the left of $x=0$ and to the right of $x=0$, but not across that line.
        Consequently, it is a strong solution there.
        
        Suppose that $\phi$ is a compactly supported smooth function. 
        \begin{align}
         &
         \int_{\mathbb R} \int_{0}^{\infty} u \partial_t \phi + a u \partial_x \phi \;dxdt
         \\&\qquad
         =
         \int_{x < 0} \int_{0}^{\infty} u \partial_t \phi + a u \partial_x \phi \;dxdt
         + 
         \int_{x > 0} \int_{0}^{\infty} u \partial_t \phi + a u \partial_x \phi \;dxdt
         \\
         &\qquad
         =
         -
         \int_{x < 0} \int_{0}^{\infty} \partial_t u \phi + a \partial_x u \phi \;dxdt
         - 
         \int_{x > 0} \int_{0}^{\infty} \partial_t u \phi + a \partial_x u \phi \;dxdt
         \\&\qquad \qquad 
         +
         \int_{0}^{\infty} u(-0,t) \phi \;dt
         - 
         \int_{0}^{\infty} 2u(+0,t) \phi \;dt
         \\&\qquad \qquad 
         +
         \int_{x < 0} u(x,0) \phi \;dx
         + 
         \int_{x > 0} u(x,0) \phi \;dx
         \\&\qquad 
         =
         \int_{0}^{\infty} u(-0,t) \phi \;dt
         - 
         \int_{0}^{\infty} 2u(+0,t) \phi \;dt
         +
         \int_{\mathbb R} u_{0}(x) \phi \;dx
         .
        \end{align}
        The two integrals over the time axis cancel out.
        The proof is complete.
    \end{enumerate}
\end{solution}


\end{document}
