%Typeset with LaTeX
%\documentclass{owrart}
\documentclass[10pt,reqno]{amsart}
\usepackage[a4paper,margin=1.0in]{geometry}
\UseRawInputEncoding
%% -------------------------------------------------------------------------------
%% Enter additionally required packages below this comment.
\usepackage{algorithm}
\usepackage{algpseudocode}

\makeatletter
% Reinsert missing \algbackskip
\def\algbackskip{\hskip-\ALG@thistlm}
\makeatother

\usepackage{amsmath}
\usepackage{mathtools}
%\usepackage[a4paper]{geometry}
%\usepackage{amsfonts}
\usepackage{amssymb}
\usepackage{amsthm}
\usepackage{mathrsfs}
\usepackage{bm}
\usepackage{algpseudocode}
\usepackage{upgreek}
\usepackage{subcaption}
% \usepackage{showkeys} %%%%%%%%%
% \usepackage{amscd}
% \usepackage{latexsym}
% \usepackage{nicefrac} 3
% \usepackage{cite}
 %\usepackage{epic}
% \usepackage{eepic}
% \usepackage{fancybox}
\newcommand{\bsnul}{{\boldsymbol 0}}
\newcommand{\cNN}{{\mathcal {NN}}}

\usepackage{graphicx}
\usepackage{epstopdf}

\mathtoolsset{showonlyrefs}

%\usepackage{epstopdf}
%\usepackage{graphicx} 

%\usepackage[pdf]{pstricks}

\usepackage[shortlabels]{enumitem}
%Standard operator
\newcommand{\abs}[1]{\lvert #1\rvert}
\newcommand{\supp}{Peratorname{supp}}
\newcommand{\grad}{{\mathbf{grad}}}
\newcommand{\curl}{\normalfont{\text{curl}}}
\newcommand{\Log}{\text{Log}}
\newcommand{\cc}{{\normalfont{\text{c}}}}
\newcommand{\loc}{{\normalfont{\text{loc}}}}

\DeclareMathOperator*{\argmax}{arg\,max}
\DeclareMathOperator*{\argmin}{arg\,min}

% Package for colors
\usepackage{xcolor}
\definecolor{ao}{rgb}{0.0, 0.5, 0.0}

%Edit colors
\newcommand{\CS}[1]{{\color{magenta}{#1}}}
\newcommand{\ra}[1]{{\color{blue}{#1}}}
\newcommand{\fh}[1]{{\color{cyan}{#1}}}
\newcommand{\todo}[1]{{\color{red}[#1]}}

\newcommand{\com}[1]{{\color{red} [#1]}}

%\usepackage{fancyhdr}
%\usepackage{datetime}
%
%\fancyhf{}
%\fancyfoot[L]{\fontsize{10}{12} \today \ \currenttime}
%\pagestyle{fancy} \usepackage{tikz}

\DeclareSymbolFont{extraup}{U}{zavm}{m}{n}
\DeclareMathSymbol{\varheart}{\mathalpha}{extraup}{86}
\DeclareMathSymbol{\vardiamond}{\mathalpha}{extraup}{87}

\usepackage{chngcntr}
\counterwithout{equation}{subsection}
\counterwithout{equation}{section} 

\usepackage{tikz}

\newcommand\encircle[1]{%
  \tikz[baseline=(X.base)] 
    \node (X) [draw, shape=circle, inner sep=0] {\strut #1};}

\usepackage{pifont}

\usepackage{fancyhdr}
\usepackage[us,12hr]{datetime} % `us' makes \today behave as usual in TeX/LaTeX
\fancypagestyle{plain}{
\fancyhf{}
\rhead{\footnotesize Version of \today}
\lhead{\thepage}
\renewcommand{\headrulewidth}{0pt}}
\pagestyle{plain}

\usepackage{color}
\newcommand{\half}{\frac{1}{2}}

\newcommand{\norm}[1]{\left \lVert #1 \right \rVert}
\newcommand{\snorm}[1]{\left \lvert #1 \right \rvert}

%fields, mathbb operators
\newcommand{\IR}{\mathbb{R}}
\newcommand{\IC}{\mathbb{C}}
\newcommand{\IZ}{\mathbb{Z}}
\newcommand{\IQ}{\mathbb{Q}}
\newcommand{\IN}{\mathbb{N}}
\newcommand{\IT}{\mathbb{T}}

% Boundary Integral Operators
\newcommand{\OV}{\mathsf{V}}
\newcommand{\OK}{\mathsf{K}}
\newcommand{\OW}{\mathsf{W}}
% 

%
%\DeclareMathOperator*{\argmax}{arg\,max}
%\DeclareMathOperator*{\argmin}{arg\,min}

\newcommand{\OA}{\mathsf{A}}
\newcommand{\Id}{\mathsf{Id}}
\newcommand{{\D}}{\normalfont{\text{D}}}
\newcommand{{\G}}{\normalfont{\text{G}}}
\newcommand{{\U}}{\normalfont{\text{U}}}
\newcommand{{\I}}{\normalfont{\text{I}}}
\newcommand{{\per}}{\normalfont{\text{per}}}
\newcommand{{\y}}{{\boldsymbol{{y}}}}
\newcommand{\Of}{\mathsf{f}}

\newcommand{{\bc}}{{\bf c}}
\newcommand{{\bx}}{{\bf x}}
\newcommand{{\by}}{{\bf y}}
\newcommand{{\bz}}{{\bf z}}
\newcommand{{\bd}}{{\bf d}}
\renewcommand{\d}{\!\!\operatorname{d}}

% Traces
\newcommand{\traced}{\gamma_{\text{D}}}
\newcommand{\tracen}{\gamma_{\text{N}}}
\newcommand{\dual}[2]{\left \langle #1,#2 \right \rangle}
\newcommand{\dotp}[2]{\left ( #1,#2 \right )}

% Matrix trace
\newcommand{\tr}[1]{\text{tr}\left ( #1 \right )}

\newtheorem{theorem}{Theorem}[section]
\newtheorem{corollary}[theorem]{Corollary}
\newtheorem{lemma}[theorem]{Lemma}
\newtheorem{prob}[theorem]{Problem}
\newtheorem{definition}[theorem]{Definition}
\newtheorem{assumption}[theorem]{Assumption}
\newtheorem{proposition}[theorem]{Proposition}
\newtheorem{example}[theorem]{Example}
\newtheorem{problem}[theorem]{Problem}

\theoremstyle{remark}
\newtheorem{remark}{Remark}
%\newtheorem{assumption}[remark]{Assumption}
%\newtheorem{example}[remark]{Example}
%\newtheorem{problem}[thm]{Problem}


%\numberwithin{equation}{section}

\begin{document}
%\begin{talk}
%
%---------
% TITLE
%---------
%

\title{Exercise Set 2 \\ 
MATH-459--Numerical Methods for Conservation Laws \\  
Autumn Semester 2022
}

\author{Professor: Martin W. Licht \\ Assistant: Fernando Henriquez B.} 

\maketitle

\subsection*{Problem 1: Method of Characteristics}

Suppose that the flux $f(u,x,t)$ is differentiable in all variables.
Find curves along which the conservation law 
\begin{align}
	\frac{\partial u(x,t)}{\partial t}
	-
	x
	\frac{\partial f(u(x,t),x,t)}{\partial x}
	= 0
\end{align}
can be written as a collection of ordinary differential equations.

\subsection*{Problem 2: Method of Characteristics}

\begin{itemize}
\item[(i)] Consider the conservation law 
\begin{align}
	\frac{\partial u}{\partial t}
	-
	x
	\frac{\partial u}{\partial x}
	= 0
\end{align}
with initial value
\begin{align}
	u(x,0)=x.
\end{align}
Sketch the characteristics up to time $t=1$. Describe the graph of the function $u(\cdot,t)$
as $t$ increases.
\item[(ii)] Consider the conservation law
\begin{align}
	\frac{\partial u}{\partial t}
	+
	x
	\frac{\partial u}{\partial x}
	= 0
\end{align}
with initial value
\begin{align}
	u(x,0)=x.
\end{align}
Draw the characteristics and describe the graph of the function $u(\cdot,t)$ as $t$ increases.
\end{itemize}

\subsection*{Problem 3: Weak Solutions}

Show that a weak solution to the linear transport equation 
$$
	\frac{\partial u}{\partial t} 
	+ 
	a 
	\frac{\partial u}{\partial x}  = 0, 
$$
with $a\in \IR$ and initial data 
\begin{align}
	u(x,0) = 
	\left\{
	\begin{array}{ll}
	1, & \text{for} \quad x<0,\\
	0, &\text{for} \quad x > 0,
	\end{array}
	\right.
\end{align}
is given by
\begin{align} 
	u(x,t) = 
	\left\{
	\begin{array}{ll}
	1, & \text{for} \quad x<at,\\
	0, & \text{for} \quad x > at,
	\end{array}
	\right.
\end{align}




\subsection*{Problem 4: Rarefaction Waves}
Consider the initial value problem 
\begin{align}
	\frac{\partial u}{\partial t}
	+
	\frac{\partial f(u)}{\partial x}
	=0,
	\quad
	u(x,0) = u_0(x),
\end{align}
with $f(u) =\frac{u^2}{2}$, and
\begin{align}
	u_0(x)
	=
	\left \{
	\begin{array}{ll}
	2, \quad0<x<1, \\
	0, \quad\text{otherwise},
	\end{array}
	\right.
\end{align}
Here a rarefaction wave arrises at one discontinuity and a shock at the other. 
The goal of this exercise is to determine the exact solution for all $t>0$. 
In this setup, the rarefaction wave catches up with the shock at some time $T_c>0$.
\begin{itemize}
	\item[(i)]
	Draw the profile of $u_0(x)$
	and sketch the characteristics in the strip 
	$0 < t < T_c$ of the $x-t$ plane.
	\item[(ii)]
	Determine the exact solution for $0 < t < T_c$.
	\item[(iii)]
	Let $x_s(t)$ be shock's location at $t > T_c$. 
	By using the Rankine-Hugoniot jump condition construct an ODE to determine 
	$x_s(t)$ for all $t > T_c$. 
	In the sketch you drew in (i), extend the characteristic lines to $t > T_c$.
\end{itemize}
%\subsection*{Problem 3: Traffic Flow Equation}
%Consider the traffic flow equation
%\begin{align}
%	\frac{\partial u}{\partial t}
%	+
%	\frac{\partial}{\partial x}
%	\left(
%		q
%		U(q)
%	\right)
%	=
%	0,
%\end{align}
%where $q(x,t)$ is the car density, $U(q)$ is the traffic speed
%as a function of the car density. 
%Let $f(q) = q U(q)$.
%A simple model for traffic flow is obtained by taking
%\begin{align}
%	U(q)
%	=
%	u_m(1-q),
%\end{align}
%with $u_m>0$ and $0 \leq q \leq 1$.
%At zero density (empty road) and the traffic speed is $u_m$.
%As $q$ approaches 1, the speed decreases to zero. The model then reads
%\begin{align}\label{eq:traffic}
%	\frac{\partial u}{\partial t}
%	+
%	\frac{\partial}{\partial x}
%	\left(
%		u_m(q-q^2)
%	\right)
%	=
%	0,
%\end{align}
%Show that for the traffic flow equation \eqref{eq:traffic}, 
%the condition $q_l <q_r$ is required for a shock to be admissible. 
%Do this by verifying each of the following conditions:
%\begin{itemize}
%	\item[(i)]
%	The entropy condition $f'(q_l) >s f'(q_r)$.
%	\item[(ii)]
%	There exists an entropy function $\eta(q)$
%	and a corresponding entropy flux 
%	$\phi(q)$ such that
%	\begin{align}
%		s(\eta(q_r)-)
%	\end{align}
%\end{itemize}
\bibliography{ref}{}
\bibliographystyle{plain}
\end{document}
