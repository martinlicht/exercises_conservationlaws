%Typeset with LaTeX
%\documentclass{owrart}
\documentclass[10pt,reqno]{amsart}
\usepackage[a4paper,margin=1.0in]{geometry}
\UseRawInputEncoding
%% -------------------------------------------------------------------------------
%% Enter additionally required packages below this comment.
\usepackage{algorithm}
\usepackage{algpseudocode}

\makeatletter
% Reinsert missing \algbackskip
\def\algbackskip{\hskip-\ALG@thistlm}
\makeatother

\usepackage{amsmath}
\usepackage{mathtools}
%\usepackage[a4paper]{geometry}
%\usepackage{amsfonts}
\usepackage{amssymb}
\usepackage{amsthm}
\usepackage{mathrsfs}
\usepackage{bm}
\usepackage{algpseudocode}
\usepackage{upgreek}
\usepackage{subcaption}
% \usepackage{showkeys} %%%%%%%%%
% \usepackage{amscd}
% \usepackage{latexsym}
% \usepackage{nicefrac} 3
% \usepackage{cite}
 %\usepackage{epic}
% \usepackage{eepic}
% \usepackage{fancybox}
\newcommand{\bsnul}{{\boldsymbol 0}}
\newcommand{\cNN}{{\mathcal {NN}}}

\usepackage{graphicx}
\usepackage{epstopdf}

\mathtoolsset{showonlyrefs}

%\usepackage{epstopdf}
%\usepackage{graphicx} 

%\usepackage[pdf]{pstricks}

\usepackage[shortlabels]{enumitem}
%Standard operator
\newcommand{\abs}[1]{\lvert #1\rvert}
\newcommand{\supp}{Peratorname{supp}}
\newcommand{\grad}{{\mathbf{grad}}}
\newcommand{\curl}{\normalfont{\text{curl}}}
\newcommand{\Log}{\text{Log}}
\newcommand{\cc}{{\normalfont{\text{c}}}}
\newcommand{\loc}{{\normalfont{\text{loc}}}}

\DeclareMathOperator*{\argmax}{arg\,max}
\DeclareMathOperator*{\argmin}{arg\,min}

% Package for colors
\usepackage{xcolor}
\definecolor{ao}{rgb}{0.0, 0.5, 0.0}

%Edit colors
\newcommand{\CS}[1]{{\color{magenta}{#1}}}
\newcommand{\ra}[1]{{\color{blue}{#1}}}
\newcommand{\fh}[1]{{\color{cyan}{#1}}}
\newcommand{\todo}[1]{{\color{red}[#1]}}

\newcommand{\com}[1]{{\color{red} [#1]}}

%\usepackage{fancyhdr}
%\usepackage{datetime}
%
%\fancyhf{}
%\fancyfoot[L]{\fontsize{10}{12} \today \ \currenttime}
%\pagestyle{fancy} \usepackage{tikz}

\DeclareSymbolFont{extraup}{U}{zavm}{m}{n}
\DeclareMathSymbol{\varheart}{\mathalpha}{extraup}{86}
\DeclareMathSymbol{\vardiamond}{\mathalpha}{extraup}{87}

\usepackage{chngcntr}
\counterwithout{equation}{subsection}
\counterwithout{equation}{section} 

\usepackage{tikz}

\newcommand\encircle[1]{%
  \tikz[baseline=(X.base)] 
    \node (X) [draw, shape=circle, inner sep=0] {\strut #1};}

\usepackage{pifont}

\usepackage{fancyhdr}
\usepackage[us,12hr]{datetime} % `us' makes \today behave as usual in TeX/LaTeX
\fancypagestyle{plain}{
\fancyhf{}
\rhead{\footnotesize Version of \today}
\lhead{\thepage}
\renewcommand{\headrulewidth}{0pt}}
\pagestyle{plain}

\usepackage{color}
\newcommand{\half}{\frac{1}{2}}

\newcommand{\norm}[1]{\left \lVert #1 \right \rVert}
\newcommand{\snorm}[1]{\left \lvert #1 \right \rvert}

%fields, mathbb operators
\newcommand{\IR}{\mathbb{R}}
\newcommand{\IC}{\mathbb{C}}
\newcommand{\IZ}{\mathbb{Z}}
\newcommand{\IQ}{\mathbb{Q}}
\newcommand{\IN}{\mathbb{N}}
\newcommand{\IT}{\mathbb{T}}

% Boundary Integral Operators
\newcommand{\OV}{\mathsf{V}}
\newcommand{\OK}{\mathsf{K}}
\newcommand{\OW}{\mathsf{W}}
% 

%
%\DeclareMathOperator*{\argmax}{arg\,max}
%\DeclareMathOperator*{\argmin}{arg\,min}

\newcommand{\OA}{\mathsf{A}}
\newcommand{\Id}{\mathsf{Id}}
\newcommand{{\D}}{\normalfont{\text{D}}}
\newcommand{{\G}}{\normalfont{\text{G}}}
\newcommand{{\U}}{\normalfont{\text{U}}}
\newcommand{{\I}}{\normalfont{\text{I}}}
\newcommand{{\per}}{\normalfont{\text{per}}}
\newcommand{{\y}}{{\boldsymbol{{y}}}}
\newcommand{\Of}{\mathsf{f}}

\newcommand{{\bc}}{{\bf c}}
\newcommand{{\bx}}{{\bf x}}
\newcommand{{\by}}{{\bf y}}
\newcommand{{\bz}}{{\bf z}}
\newcommand{{\bd}}{{\bf d}}
\renewcommand{\d}{\!\!\operatorname{d}}

% Traces
\newcommand{\traced}{\gamma_{\text{D}}}
\newcommand{\tracen}{\gamma_{\text{N}}}
\newcommand{\dual}[2]{\left \langle #1,#2 \right \rangle}
\newcommand{\dotp}[2]{\left ( #1,#2 \right )}

% Matrix trace
\newcommand{\tr}[1]{\text{tr}\left ( #1 \right )}

\newtheorem{theorem}{Theorem}[section]
\newtheorem{corollary}[theorem]{Corollary}
\newtheorem{lemma}[theorem]{Lemma}
\newtheorem{prob}[theorem]{Problem}
\newtheorem{definition}[theorem]{Definition}
\newtheorem{assumption}[theorem]{Assumption}
\newtheorem{proposition}[theorem]{Proposition}
\newtheorem{example}[theorem]{Example}
\newtheorem{problem}[theorem]{Problem}

\theoremstyle{remark}
\newtheorem{remark}{Remark}
%\newtheorem{assumption}[remark]{Assumption}
%\newtheorem{example}[remark]{Example}
%\newtheorem{problem}[thm]{Problem}


%\numberwithin{equation}{section}

\begin{document}
%\begin{talk}
%
%---------
% TITLE
%---------
%

\title{Exercise Set 3\\ 
MATH-459--Numerical Methods for Conservation Laws \\  
Autumn Semester 2021
}

\author{Professor: Martin W. Licht \\ Assistant: Fernando Henriquez B. \\ Published on 6.10.21} 

\maketitle

\subsection*{Problem 1. Liu's Entropy Condition}
Consider the conservation law 
\begin{align}
	\frac{\partial u}{\partial t}
	+
	\frac{\partial f(u)}{\partial x}
	= 0 
\end{align}
where $f(u) = e^u$. We discuss the Riemann problem
\begin{align}
	u_0(x) = 
	\left\{
	\begin{array}{cl}
	u_l, & \text{for} \quad x<0,\\
	u_r, &\text{for} \quad x > 0
	\end{array}
	\right.
\end{align}
\begin{enumerate}
\item Show that when $u(x,t)$ is a weak solution and $\lambda > 0$, then $u(\lambda x, \lambda t)$ is a weak solution too.
\item Suppose that the solution has a shock wave with values $u_l$ and $u_r$ to the left and to the right, respectively.
Which values $u_l$ and $u_r$ are admissible on the basis of Liu's entropy condition?
\end{enumerate}
\textit{Remark: the scale invariance above suggests that "physical" solutions to the Riemann problem should be scale-invariant themselves.}


\subsection*{Problem 2. Entropy Solutions}
Consider the conservation law 
\begin{align}
	\frac{\partial u}{\partial t}
	+
	\frac{\partial f(u)}{\partial x}
	= 0,
	\quad
	u(x,0)=u_0(x),
\end{align}
and $f(u) = u^2$.
Consider the following initial conditions
\begin{align}
	\text{(a)}
	\,\,\;\;
	u_0(x) = 
	\left\{
	\begin{array}{cl}
	1, & \text{for} \quad x<-1,\\
	0, &\text{for} \quad -1<x< 1,\\
	-1, &\text{for} \quad x > 1
	\end{array}
	\right.
	\quad
	\text{and}
	\quad
	\text{(b)}
	\,\,\;\;
	u_0(x) = 
	\left\{
	\begin{array}{cl}
	-1, & \text{for} \quad x<-1,\\
	0, &\text{for} \quad -1<x < 1, \\
	1, &\text{for} \quad x > 1.
	\end{array}
	\right.
\end{align}
Draw the profile of $u_0(x)$ and sketch the characteristics
of the entropy solution $u(x,t)$ in the $x-t$ plane, and
determine $u(x,t)$ for all $t > 0$.

\subsection*{Problem 3. Weak and Entropy Solutions}
Show that for every $\alpha>1$ the function
\begin{align}
	u_\alpha(x,t)
	=
	\left\{
	\begin{array}{cl}
	-1, & 2x<(1-\alpha) t, \\
	-\alpha,& (1-\alpha)t<2x<0, \\
	\alpha, & 0<2x<(\alpha-1)t, \\
	1, & (\alpha-1) t< 2x,
	\end{array}
	\right.
\end{align}
is a weak solution of the problem
\begin{align}\label{eq:burgers_conservation_law}
	\frac{\partial u}{\partial t}
	+
	\frac{\partial f(u) }{\partial x}
	= 0,
	\quad
	u(x,0)
	=
	\left\{
	\begin{array}{cl}
	-1, & \text{for} \quad x<0,\\
	1, &\text{for} \quad x>0
	\end{array}
	\right.
\end{align}
with $f(u) = u^2/2$.
Are there any values of $\alpha$
for which $u_\alpha(x,t)$ is an entropy solution
of \eqref{eq:burgers_conservation_law}?

\subsection*{Problem 4. Traffic Flow Equation}
Consider the traffic flow equation
\begin{align}
	\frac{\partial q}{\partial t}
	+
	\frac{\partial}{\partial x}
	f(q)
	=
	0,
\end{align}
where $q(x,t)$ is the car density, $U(q)$ is the traffic speed
as a function of the car density
and $f(q) = q U(q)$.
A simple model for traffic flow is obtained by taking
\begin{align}
	U(q)
	=
	u_m
	\left(
		1-\frac{q}{q_m}
	\right),
\end{align}
with $u_m>0$ and $q_m$ being the maximum speed
and maximum car density, respectively.
If the car density is maximal, we say that the traffic is ``bumper-to-bumper''.
At zero density (empty road), the traffic speed is $u_m$.
As $q$ approaches $q_m$, the speed decreases to zero. The model then reads
\begin{align}\label{eq:traffic}
	\frac{\partial q}{\partial t}
	+
	\frac{\partial}{\partial x}
	u_m
	\left(
		q-\frac{q^2}{q_m}
	\right)
	=
	0.
\end{align}
We equip \eqref{eq:traffic} with the initial condition
\begin{align}\label{eq:init_cond}
	q(x,0)
	=
	\left\{
	\begin{array}{cl}
	q_l, & \text{for} \quad x<0,\\
	q_r, &\text{for} \quad x>0.
	\end{array}
	\right.
\end{align}
\begin{itemize}
\item[(a)] {\bf The green light problem.}
Assume that the traffic is standing at a red light, placed at $x = 0$, while the road ahead is empty.
At time $t = 0$, the traffic light turns green and we want to describe the car flow
evolution for $t>0$. To represent this situation, we set $q_l = q_m$ and $q_r =0$ in
\eqref{eq:init_cond}.
Draw the profile of $u_0(x)$ and sketch the characteristics
of the entropy solution $u(x,t)$ in the $x-t$ plane, and
determine $u(x,t)$ for all $t > 0$.
\item[(b)] {\bf Traffic jam ahead.}
We now consider the initial density profile 
with $q_l = \frac{1}{8}q_m$ and $q_r =q_m$.
For $x > 0$, the density is maximal and therefore the traffic is ``bumper-to-bumper''.
The cars on the left move with speed $U = \frac{7}{8}u_m$, so that we expect congestion.
Draw the profile of $u_0(x)$ and sketch the characteristics
of the entropy solution $u(x,t)$ in the $x-t$ plane, and
determine $u(x,t)$ for all $t > 0$.
\item[(c)] {\bf Entropy Solutions.}
Show that for the traffic flow equation \eqref{eq:traffic}, 
the condition $q_l <q_r$ is required for a shock to be admissible. 
Do this by verifying each of the following conditions:
\begin{itemize}
	\item[(i)]
	The entropy condition $f'(q_l) >s > f'(q_r)$.
	\item[(ii)]
	There exists an entropy function $\eta(q)$
	and a corresponding entropy flux 
	$\psi(q)$ such that
	\begin{align}
		s(\eta(q_r)-\eta(q_l))
		\geq
		\psi(q_r)
		-
		\psi(q_l)
	\end{align}
	holds if and only if $q_l<q_r$.
\end{itemize}
\end{itemize}
\bibliography{ref}{}
\bibliographystyle{plain}
\end{document}
