\documentclass{article}



\usepackage{../auxFiles/ExStyPac}

\newcommand{\iph}{{i + \frac{1}{2}}}
\newcommand{\imh}{{i - \frac{1}{2}}}

\input ../auxFiles/mac.tex
\graphicspath{{./figures/}}





%========================================================================
\begin{document}



\Head{11}{ENO reconstruction}



Consider the linear advection equation
\begin{gather} \label{linadv}
	v_t + a v_x = 0\ .
\end{gather}%
In Exercise 10, we had observed that a fixed stencil approach for reconstruction is not suitable for obtaining stable, non-oscillatory solutions. In the following exercises you are asked to write a number of programs which calculate different parts of the ENO reconstruction. Assume the grid is uniform.


\begin{exerciseList}


\item
Write a program that takes as an input a stencil-size $k$, and a grid function $\ol v$, and finds for each cell an appropriate stencil of size $k$. For example, for the $i$th cell $I_i=(x_{i-1/2},  x_{i+1/2})$, the function should return the output $r_i$, if the chosen stencil is $x_{i-r_i},\ldots,x_{i-r_i+k-1}$. Here, the value $\ol v_i$ of $\ol v$ at $x_i$ is interpreted as average of the approximated function $v$ in $I_i$. Notice that since the grid is assumed to be uniform, you can make your program slightly more efficient by computing the {\it undivided differences},
\begin{align}
	V\!\angb{x_{i-1/2},x_{i+1/2}}& :=V\!\sqb{x_{i-1/2},x_{i+1/2}} =\ol v_i\\[0.5em]
	V\!\angb{x_{i-1/2},\ldots,x_{i+j+1/2}} &:=
		V\!\angb{x_{i+1/2},\ldots,x_{i+j+1/2}}
		-V\!\angb{x_{i-1/2},\ldots,x_{i+j-1/2}}
\end{align}
instead of the divided differences.


\item
Write a program that calculates the coefficients
\begin{gather}%
	c_{rj}=\sum_{m=j+1}^k
		\frac{1}{\prod_{\substack{l=0 \\l\ne m}}^k\rb{m-l}}\,
			\sum_{\substack{l=0\\l\ne m}}^k\,
			\prod_{\substack{q=0\\q\ne m,l}}^k\rb{r-q+1}
		\qquad
		j=0,\ldots,k-1, \qquad r=-1,0,\ldots,k-1\ .
\end{gather}%
Explain how this expression is derived. Check that this formula generates the coefficients obtained in Exercise 10 for $k=3$.


\item
Write a program that implements the ENO reconstruction. The program should take a grid function $\ol v$, and a stencil-size $k$, and return the values
\begin{gather}%
	v_{i+1/2}^-=\sum_{j=0}^{k-1} c_{r_i,j} \ol v_{i-r_i+j}^n\ ,
	\qquad\qquad
	v_{i-1/2}^+=\sum_{j=0}^{k-1} c_{r_i-1,j}\ol v_{i-r_i+j}^n\ .
\end{gather}%
You may assume the function is periodic.

\item Write a finite volume code for \eqref{linadv}, where the interface values can be obtained by using the ENO reconstruction. Plug these values in the Godunov flux, and integrate the semi-discrete scheme using SSP-RK3. Implement the following initial conditions on the domain $[-1,1]$ with periodic BC
\begin{gather} \label{inData1}
	v_0(x)=\sin(\pi x)\ ,  
	\quad
	T_f = 5
\end{gather}%
\begin{gather} \label{inData2}
	v_0(x)=\begin{cases}
		1 & |x|<0.5\\
		-1 & |x|>0.5
	\end{cases}\ ,
	\quad
	T_f = 0.5.
\end{gather}%
Use a CFL of 0.5 to evaluate the time-step. 

\item
Run the code for $k=3$ and $a=1$. Do you recover 3rd-order convergence for \eqref{inData1}? Is the solution oscillatory for \eqref{inData2}?



\end{exerciseList}

\end{document}
