\documentclass[10pt,letterpaper]{article}
\usepackage[english]{babel}
\usepackage{graphicx}
\usepackage[margin=2cm]{geometry}

\usepackage{float}
\usepackage{subfloat}
\usepackage{caption}
\usepackage{subcaption}
\usepackage{listings}

\usepackage{mathtools}
\usepackage{amsfonts}
\usepackage{amsmath}
\usepackage{amssymb}
\usepackage{amsthm}
\newcommand{\ol}{} % TODO: What was the purpose of this macro???
\newcommand{\dif}[1][]{\mathrm{d} {#1}\,}
\newcommand{\rb}[1]{ \left(  {#1} \right) }
\newcommand{\frb}[1]{ \left(  {#1} \right) }
\newcommand{\norm}[1]{ \left\|  {#1} \right\| }
\newcommand{\jph}{{j+\frac{1}{2}}}
\newcommand{\jmh}{{j-\frac{1}{2}}}
\newcommand{\sqb}[1]{ \left(  {#1} \right) }
\newcommand{\angb}[1]{ \left\langle  {#1} \right\rangle }
\newcommand{\mat}[1]{ \begin{array}{cc}  {#1} \end{array} }
\newcommand{\myvector}[1]{ \begin{array}{c}  {#1} \end{array} }
\newcommand{\iph}{{i + \frac{1}{2}}}
\newcommand{\imh}{{i - \frac{1}{2}}}

% \usepackage[amsmath,thmmarks,standard]{ntheorem}
\newtheoremstyle{break}
{\topsep}
{\topsep}%
{\normalfont}
{}%
{\bfseries}
{}%
{\newline}
{}%
\theoremstyle{break}
\newtheorem{exercise}{Exercise}
\newtheorem*{information}{Information}
\newtheorem{mysolution}{Solution}
% \newenvironment{solution}{\begin{comment}}{\end{comment}}
\newtheorem*{solutioninformation}{Solution Information}

\usepackage{comment}
% Switch between showing and hiding solutions by commenting out either of the following lines
\newenvironment{solution}{\begin{mysolution}}{\end{mysolution}}
% \excludecomment{solution}


% \usepackage{../auxFiles/ExStyPac}

% \newcommand{\iph}{{i + \frac{1}{2}}}
% \newcommand{\imh}{{i - \frac{1}{2}}}

% \input ../auxFiles/mac.tex
\graphicspath{{./figures/}}





%========================================================================
\begin{document}



\title{ENO reconstruction}
\date{}

\maketitle


\begin{exercise}
	Consider the linear advection equation
	\begin{gather} \label{linadv}
		v_t + a v_x = 0\ .
	\end{gather}%
	In previouses exercises, we had observed that a fixed stencil approach for reconstruction is not suitable for obtaining stable non-oscillatory solutions. 
	In the following exercises you are asked to write a number of programs which calculate different parts of the ENO reconstruction. 
	Assume the grid is uniform. 
	\begin{enumerate}
		\item
		Write a program that takes as an input a stencil-size $k$, and a grid function $\ol v$, and finds for each cell an appropriate stencil of size $k$. 
		For example, for the $i$th cell $I_i=(x_{i-1/2},  x_{i+1/2})$, the function should return the output $r_i$, 
		if the chosen stencil is $x_{i-r_i},\ldots,x_{i-r_i+k-1}$. 
		Here, the value $\ol v_i$ of $\ol v$ at $x_i$ is interpreted as average of the approximated function $v$ in $I_i$. 
		Notice that since the grid is assumed to be uniform, 
		you can make your program slightly more efficient by computing the {\it undivided differences},
		\begin{align}
			V\!\angb{x_{i-1/2},x_{i+1/2}}& :=V\!\sqb{x_{i-1/2},x_{i+1/2}} =\ol v_i\\[0.5em]
			V\!\angb{x_{i-1/2},\ldots,x_{i+j+1/2}} &:=
				V\!\angb{x_{i+1/2},\ldots,x_{i+j+1/2}}
				-V\!\angb{x_{i-1/2},\ldots,x_{i+j-1/2}}
		\end{align}
		instead of the divided differences.
		\item
		Write a program that calculates the coefficients
		\begin{gather}%
			c_{rj}=\sum_{m=j+1}^k
				\frac{1}{\prod_{\substack{l=0 \\l\ne m}}^k\rb{m-l}}\,
					\sum_{\substack{l=0\\l\ne m}}^k\,
					\prod_{\substack{q=0\\q\ne m,l}}^k\rb{r-q+1}
				\qquad
				j=0,\ldots,k-1, \qquad r=-1,0,\ldots,k-1\ .
		\end{gather}%
		Explain how this expression is derived. 
		Check that this formula generates the coefficients obtained in Exercise 10 for $k=3$.
		
		\item
		Write a program that implements the ENO reconstruction. 
		The program should take a grid function $\ol v$, and a stencil-size $k$, and return the values
		\begin{gather}%
			v_{i+1/2}^-=\sum_{j=0}^{k-1} c_{r_i,j} \ol v_{i-r_i+j}^n\ ,
			\qquad\qquad
			v_{i-1/2}^+=\sum_{j=0}^{k-1} c_{r_i-1,j}\ol v_{i-r_i+j}^n\ .
		\end{gather}%
		You may assume the function is periodic.

		\item 
		Write a finite volume code for \eqref{linadv}, 
		where the interface values can be obtained by using the ENO reconstruction. 
		Plug these values in the Godunov flux, and integrate the semi-discrete scheme using SSP-RK3. 
		Implement the following initial conditions on the domain $[-1,1]$ with periodic BC
		\begin{gather} \label{inData1}
			v_0(x)=\sin(\pi x)\ ,  
			\quad
			T_f = 5
		\end{gather}%
		\begin{gather} \label{inData2}
			v_0(x)=\begin{cases}
				1 & |x|<0.5\\
				-1 & |x|>0.5
			\end{cases}\ ,
			\quad
			T_f = 0.5.
		\end{gather}%
		Use a CFL of 0.5 to evaluate the time-step. 

		\item
		Run the code for $k=3$ and $a=1$. 
		Do you recover 3rd-order convergence for \eqref{inData1}? 
		Is the solution oscillatory for \eqref{inData2}?
	\end{enumerate}
\end{exercise}


\begin{solution}
	\begin{enumerate}
		\item
		See the Matlab code attached at the end of this solution manual. The stencil selection depends on the index shift $r$, which is evaluated by the function {\tt find\_shift(U,k)}.

		\item
		The coefficients $c_{rj}$ are computed once at the beginning of the solver by the function {\tt eval\_crj(k)}, assuming a uniform grid. 

		\item Once the shift $r$ is evaluate for each cell, the solution at the left and right interfaces of the cell are computed in the function {\tt eno\_recon}.

		\item Note that the code is capable of implementing both periodic and open boundary conditions, for arbitrary order ENO. The order can be set via the variable $k$.

		\item
		Third-order convergence is obtained for the smooth initial condition. Moreover, the oscillations observed with the discontinuous initial data for fixed central reconstruction in Exercise 9, are no longer visible.
	\end{enumerate}
	%\newpage
% 	\lstinputlisting{./Code/Exercise10.m}	% TODO: include all those codes 
% 	\lstinputlisting{./Code/solver.m}		% TODO: include all those codes 
% 	\lstinputlisting{./Code/evalRHS.m}		% TODO: include all those codes 
% 	\lstinputlisting{./Code/apply_bc.m}		% TODO: include all those codes 
% 	\lstinputlisting{./Code/eval_crj.m}		% TODO: include all those codes 
% 	\lstinputlisting{./Code/eno_recon.m}	% TODO: include all those codes 
% 	\lstinputlisting{./Code/find_shift.m}	% TODO: include all those codes 
% 	\lstinputlisting{./Code/find_exact.m}	% TODO: include all those codes 
% 	\lstinputlisting{./Code/find_err.m}		% TODO: include all those codes 
\end{solution}




\end{document}
