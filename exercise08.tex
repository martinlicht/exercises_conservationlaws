\documentclass{article}



\usepackage{../auxFiles/ExStyPac}



\input ../auxFiles/mac.tex
\graphicspath{{./figures/}}
\newcommand{\iph}{{i + \frac{1}{2}}}
\newcommand{\imh}{{i - \frac{1}{2}}}




%========================================================================
\begin{document}



\Head{8}{Incremental form and Harten's lemma}



Consider the conservation law
\begin{gather} \label{conlaw}
	u_t+f(u)_x=0,
\end{gather}%
and the numerical scheme
\begin{gather} \label{genscheme}
	u^{n+1}_i = G(u^n_{i-k},...,u^n_{i+k}).
\end{gather}%
We say that the scheme \eqref{genscheme} can be put in \textit{incremental form} if there exists two incremental coefficients $C_\iph = C(u_{i-k+1}^n,...,u_{i+k}^n)$ and $D_\iph = D(u_{i-k+1}^n,...,u_{i+k}^n)$, which can be used to re-write the scheme as
\begin{gather} \label{incform}
	u^{n+1}_i = u_i^n - C_\imh \Delta^-u^n_i + D_\iph \Delta^+u^n_i,
\end{gather}%
where $\Delta^+u_i = u_{i+1} - u_i$ and $\Delta^-u_i = u_i- u_{i-1}$. Harten's lemma states that a scheme written in incremental form is TVD if i) $C_\iph \geq 0$, ii) $D_\iph \geq 0$ and iii) $C_\iph + D_\iph \leq 1$.

\begin{exerciseList}

\item Prove that any 3-point consistent, conservative scheme with numerical flux $F_\iph$ admits an incremental form with coefficients
\[
C_\iph = \frac{k}{h} \left(\frac{f(u_{i+1}) - F_\iph}{\Delta^+u_i} \right), \qquad D_\iph = \frac{k}{h} \left(\frac{f(u_{i}) - F_\iph}{\Delta^+u_i}\right).
\]


\item
Consider a conservative scheme with
\begin{itemize}
\item Lax-Friedrich flux:
\[
F^{LF}(u,v)= \frac{1}{2} \left( f(u) + f(v) - \frac{h}{k} (v-u)\right),
\]
\item Local Lax-Friedrich/ Rusanov flux:
\[
F^{LLF}(u,v)= \frac{1}{2} \left( f(u) + f(v) - \alpha (v-u)\right), \quad \alpha = \max_u|f^\prime(u)|,
\]
\item Lax-Wendroff flux:
\[
F^{LW}(u,v)= \frac{1}{2} \left( f(u) + f(v) - \frac{k}{h} f^\prime\left( \frac{u+v}{2}\right)(f(v) - f(u))\right),
\]
\item Roe flux:
\[
F^{Roe}(u,v)= \frac{1}{2} \left( f(u) + f(v) - \alpha (v-u)\right), \quad \alpha = \left|\frac{f(v) -f(u)}{v-u}\right|.
\]
\end{itemize}
\begin{enumerate}
\item Find the incremental coefficients for each flux. 
\item Check whether all three conditions of Harten's lemma are satisfied with each flux.
\item Based on your conclusions from 7.2(b), can you say whether the numerical solution obtained with a TVD scheme is guaranteed to converge to an entropy solution?
\end{enumerate}

\item
Let $f(u) = c u$. Consider a scheme with the hybrid flux,
\[
F(u,v) = \theta F^{LW} + (1-\theta) F^{LF}, \quad 0 \leq \theta \leq 1,
\]
which is nothing but a convex combination of the Lax-Friedrich and Lax-Wendroff fluxes. Assuming the usual CFL condition, can you find a $\theta$ that will lead to a TVD scheme?




\end{exerciseList}


\end{document}
