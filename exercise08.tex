\documentclass{article}



\usepackage{../auxFiles/ExStyPac}



\input ../auxFiles/mac.tex
\graphicspath{{./figures/}}
\newcommand{\iph}{{i + \frac{1}{2}}}
\newcommand{\imh}{{i - \frac{1}{2}}}




%========================================================================
\begin{document}



\Head{8}{Incremental form and Harten's lemma}



Consider the conservation law
\begin{gather} \label{conlaw}
	u_t+f(u)_x=0,
\end{gather}%
and the numerical scheme
\begin{gather} \label{genscheme}
	u^{n+1}_i = G(u^n_{i-k},...,u^n_{i+k}).
\end{gather}%
We say that the scheme \eqref{genscheme} can be put in \textit{incremental form} if there exists two incremental coefficients $C_\iph = C(u_{i-k+1}^n,...,u_{i+k}^n)$ and $D_\iph = D(u_{i-k+1}^n,...,u_{i+k}^n)$, which can be used to re-write the scheme as
\begin{gather} \label{incform}
	u^{n+1}_i = u_i^n - C_\imh \Delta^-u^n_i + D_\iph \Delta^+u^n_i,
\end{gather}%
where $\Delta^+u_i = u_{i+1} - u_i$ and $\Delta^-u_i = u_i- u_{i-1}$. Harten's lemma states that a scheme written in incremental form is TVD if i) $C_\iph \geq 0$, ii) $D_\iph \geq 0$ and iii) $C_\iph + D_\iph \leq 1$.

\begin{exerciseList}

\item Prove that any 3-point consistent, conservative scheme with numerical flux $F_\iph$ admits an incremental form with coefficients
\[
C_\iph = \frac{k}{h} \left(\frac{f(u_{i+1}) - F_\iph}{\Delta^+u_i} \right), \qquad D_\iph = \frac{k}{h} \left(\frac{f(u_{i}) - F_\iph}{\Delta^+u_i}\right).
\]


\item
Consider a conservative scheme with
\begin{itemize}
\item Lax-Friedrich flux:
\[
F^{LF}(u,v)= \frac{1}{2} \left( f(u) + f(v) - \frac{h}{k} (v-u)\right),
\]
\item Local Lax-Friedrich/ Rusanov flux:
\[
F^{LLF}(u,v)= \frac{1}{2} \left( f(u) + f(v) - \alpha (v-u)\right), \quad \alpha = \max_u|f^\prime(u)|,
\]
\item Lax-Wendroff flux:
\[
F^{LW}(u,v)= \frac{1}{2} \left( f(u) + f(v) - \frac{k}{h} f^\prime\left( \frac{u+v}{2}\right)(f(v) - f(u))\right),
\]
\item Roe flux:
\[
F^{Roe}(u,v)= \frac{1}{2} \left( f(u) + f(v) - \alpha (v-u)\right), \quad \alpha = \left|\frac{f(v) -f(u)}{v-u}\right|.
\]
\end{itemize}
\begin{enumerate}
\item Find the incremental coefficients for each flux. 
\item Check whether all three conditions of Harten's lemma are satisfied with each flux.
\item Based on your conclusions from 7.2(b), can you say whether the numerical solution obtained with a TVD scheme is guaranteed to converge to an entropy solution?
\end{enumerate}

\item
Let $f(u) = c u$. Consider a scheme with the hybrid flux,
\[
F(u,v) = \theta F^{LW} + (1-\theta) F^{LF}, \quad 0 \leq \theta \leq 1,
\]
which is nothing but a convex combination of the Lax-Friedrich and Lax-Wendroff fluxes. Assuming the usual CFL condition, can you find a $\theta$ that will lead to a TVD scheme?




\end{exerciseList}


\end{document}




\documentclass{article}



\usepackage{../auxFiles/ExStyPac}

\usepackage[framed]{../auxFiles/mcode}

\newcommand{\iph}{{i + \frac{1}{2}}}
\newcommand{\imh}{{i - \frac{1}{2}}}
\newcommand{\ipo}{{i + 1}}




\input ../auxFiles/mac.tex





%========================================================================
\begin{document}



\Head{8}{Incremental form and Harten's lemma (Solution)}





\begin{exerciseList}


\item
Consider the scheme
\begin{gather} \label{3pt_scheme}
u_i^{n+1} = u^n_i - \frac{k}{h} \left[ F_\iph - F_\imh\right].
\end{gather}
The RHS of \eqref{3pt_scheme} can be re-written as

\begin{align*}
RHS &= u^n_i - \frac{k}{h} \left[ F_\iph - F_\imh\right]\\
        &= u^n_i - \frac{k}{h} \left[ F_\iph  - f(u_i) + f(u_i) - F_\imh\right]\\
        &= u^n_i - \frac{k}{h} \left[ \frac{F_\iph  - f(u_i)}{\Delta ^+u_i^n} \right] \Delta ^+u_i^n - \frac{k}{h}\left[\frac{f(u_i) - F_\imh}{\Delta ^-u_i^n}\right]\Delta ^-u_i^n\\
        &= u^n_i +  D_\iph \Delta ^+u_i^n - C_\imh \Delta ^-u_i^n
\end{align*}
since $\Delta ^-u_i^n = \Delta ^+u_{i-1}^n$. Thus, the scheme \eqref{3pt_scheme} can be written in incremental form.

\item 

\begin{enumerate}
\item
Note that all four fluxes can be written in the form
\[
F(u,v) = \frac{1}{2} \left( f(u) + f(v) - Q(u,v) (v-u)\right)
\]
where
\begin{align*}
Q^{LF}(u,v) &= \frac{h}{k}\ ,\\
Q^{LLF}(u,v) &= \alpha = \max_u |f^\prime(u)|\ ,\\
Q^{LW}(u,v) &= \frac{k}{h} f^\prime\left( \frac{u+v}{2}\right)\left(\frac{f(v) - f(u)}{v-u}\right)\ ,\\
Q^{Roe}(u,v) &= \left|\frac{f(v) - f(u)}{v-u}\right|\ .
\end{align*}
Thus, the incremental coefficients are given by
\begin{align*}
C_\iph &= \frac{k}{2h} \left [Q(u_i,u_{i+1})+\left(\frac{f(u_{i+1}) - f(u_i)}{u_{i+1} - u_i}\right) \right]\ ,\\
D_\iph &= \frac{k}{2h} \left [Q(u_i,u_{i+1})-\left(\frac{f(u_{i+1}) - f(u_i)}{u_{i+1} - u_i}\right) \right] \ .
\end{align*}
Note that the three conditions of Harten's lemma translates to 
\begin{align}
\label{1n2n3}
\frac{h}{k} \geq Q(u_i,u_{i+1}) \geq \left|\frac{f(u_{i+1}) - f(u_i)}{u_{i+1} - u_i}\right| .
\end{align}
Thus, a three-point conservative flux satisfying \eqref{1n2n3} will lead to a TVD scheme. Note that the condition \eqref{1n2n3} also implies that $Q(u_i,u_{i+1})$ is strictly positive.
%Furthermore, we assume that the usual CFL condition holds
%\[
%\frac{k}{h} \max_u |f^\prime(u)| \leq 1.
%\]

\item
For all schemes, we assume that the usual CFL condition holds
\begin{gather} \label{cfl}
\frac{k}{h} \max_u |f^\prime(u)| \leq 1.
\end{gather}
Also note that using mean value theorem, we can find a $\xi_\iph$ such that
\begin{gather} \label{mvt}
\frac{f(u_{i+1}) - f(u_i)}{u_{i+1} - u_i} = f^\prime(\xi_\iph).
\end{gather}
\begin{itemize}

\item For the Lax-Friedrichs flux, the left inequality of \eqref{1n2n3} clearly holds. Using \eqref{cfl} and \eqref{mvt}, the right inequality of \eqref{1n2n3} is also satisfied. 

\item For the local Lax-Friedrichs flux, using \eqref{cfl} and \eqref{mvt}, both inequalities of \eqref{1n2n3} is also satisfied. 

\item For the Lax-Wendroff scheme, we can use \eqref{mvt} to get
\[
Q^{LW}(u_i, u_{i+1} ) = \frac{k}{h} f^\prime\left( \frac{u_i+u_\ipo}{2}\right)f^\prime(\xi_\iph) \ ,
\]
which may fail to be positive. Thus, the \eqref{1n2n3} need not be satisfied.

\textit{
Remark: At times, the following alternate expression for the Lax-Wendroff flux is also used
\[
\widetilde{F}^{LW} = \frac{1}{2} \left( f(u) + f(v) - \frac{k}{h} \left(\frac{f(v) - f(u)}{v-u}\right)(f(v) - f(u))\right) \ ,
\]
which leads to
\[
\widetilde{Q}^{LW}(u,v) = \frac{k}{h} \left(\frac{f(v) - f(u)}{v-u}\right)^2 \ ,
\]
which is clearly positive. Using \eqref{mvt}, we have
\[
\widetilde{Q}^{LW}(u_i, u_{i+1}) = \frac{k}{h} \left(f^\prime(\xi_\iph)\right)^2.
\]
In order to satisfy the right condition of \eqref{1n2n3}, we need
\[
\frac{k}{h}| f^\prime(\xi_\iph)| \geq 1
\]
which violates the CFL condition \eqref{cfl}.
}

\item For the Roe flux, we can once again show that \eqref{1n2n3} is satisfied by using \eqref{mvt} and \eqref{cfl}.
\end{itemize}

\item Note that the Roe flux satisfies the condition of Harten's lemma, and thus leads to a TVD scheme. But we know that the Roe scheme may give entropy violating solutions. This implies that the TVD property alone does not ensure convergence to entropy solutions.

\end{enumerate}

\item Let us consider the hybrid flux 
\[
F^\theta(u,v) = \theta F^{LW} + (1-\theta) F^{LF}, \quad 0 \leq \theta \leq 1,
\]
which can also be written as
\[
F^\theta(u,v) = \frac{1}{2} \left( f(u) + f(v) - Q^\theta(u,v) (v-u)\right) \ , 
\]
where
\[
Q^\theta(u,v) = \theta Q^{LW}(u,v) + (1-\theta) Q^{LF}(u,v)\ .
\]
Assuming the flux to be linear, i.e., $f(u) = cu$, we get the simplified expression
\begin{gather} \label{hybQ}
Q^\theta(u,v) = \theta \frac{k}{h} c^2 + (1-\theta) \frac{h}{k} = \theta\left[ \left(\frac{k}{h} c\right)^2 - 1\right]\frac{h}{k} + \frac{h}{k}\ ,
\end{gather}
while the condition \eqref{1n2n3} reduces to 
\begin{gather} \label{1n2n3_lin}
\frac{h}{k} \geq Q \geq |c|\ .
\end{gather}
If the left inequality of \eqref{1n2n3_lin} need to hold, we must have
\[
\theta\left[ \left(\frac{k}{h} c\right)^2 - 1\right]\frac{h}{k} + \frac{h}{k} \leq \frac{h}{k} \quad \iff \quad \left(\frac{k}{h} c\right)^2 \leq 1 \quad \iff \quad \frac{k}{h} |c| \leq 1\ ,
\]
which is always true due to the CFL condition \eqref{cfl}. The right inequality of \eqref{1n2n3_lin} requires
\begin{align*}
\theta\left[ \left(\frac{k}{h} c\right)^2 - 1\right]\frac{h}{k} + \frac{h}{k} \geq |c| &\quad \iff \quad \theta\left[ \left(\frac{k}{h} c\right)^2 - 1\right]\geq \frac{k}{h} |c| -1 \\
&\quad \iff \quad \theta \leq \frac{1}{\frac{k}{h} |c| + 1} = \theta^* \quad \text{(using CFL condition)}.
\end{align*}
Thus, by choosing $\theta \in [0,\theta^*]$, we can recover a TVD scheme.

\end{exerciseList}



\newpage



\end{document}
